\documentclass[a4paper,12pt]{book}

\usepackage{mystyle}
\usepackage{cite}
\usepackage{chapterbib}
\usepackage{color}


\begin{document}
\newtheorem{teo}{Teorema}[section]
\newtheorem{pro}{Proposición}[section]
\newtheorem{lem}{Lema}[section]
\newtheorem{cor}{Corolario}[section]
\newtheorem{alg}{Algoritmo}[section]
\selectlanguage{spanish}
\pagenumbering{gobble}
\begin{titlepage}
 
 
\newlength{\centeroffset}
\setlength{\centeroffset}{-0.5\oddsidemargin}
\addtolength{\centeroffset}{0.5\evensidemargin}
\thispagestyle{empty}

\noindent\hspace*{\centeroffset}\begin{minipage}{\textwidth}

\centering
\includegraphics[width=0.9\textwidth]{img/logo_ugr.jpg}\\[1.4cm]

\textsc{ \Large TRABAJO FIN DE GRADO\\[0.2cm]}
\textsc{ Doble Grado en Ingeniería Informática y Matemáticas}\\[1cm]
% Upper part of the page
% 
% Title
{\LARGE\bfseries Entendiendo la teoría de nudos mediante la simulación y la informática gráfica. 
\\
}
\noindent\rule[-1ex]{\textwidth}{3pt}\\[3.5ex]
\end{minipage}

\vspace{2.5cm}
\noindent\hspace*{\centeroffset}\begin{minipage}{\textwidth}
\centering

\textbf{Autora}\\ {Cristina Zuheros Montes.}\\[2.5ex]
\textbf{Tutores}\\
{Alejandro J. León Salas.}\\
{Antonio Martínez López.}\\
\textsc{------}\\[1cm]
Granada, Diciembre de 2016\\[3cm]
\includegraphics[width=0.3\textwidth]{img/logo_ciencias.jpg} \hfill \includegraphics[width=0.2\textwidth]{img/logo_etsiit.png}\\[0.1cm]

\end{minipage}
%\addtolength{\textwidth}{\centeroffset}
%\vspace{\stretch{2}}
\end{titlepage}

\frontmatter
\pagenumbering{gobble}

\cleardoublepage
\thispagestyle{empty}

\begin{center}
{\large\bfseries Entendiendo la teoría de nudos mediante la simulación y la informática gráfica.}\\
\end{center}
\begin{center}
Cristina Zuheros Montes.\\
\end{center}

%\vspace{0.7cm}
\noindent{\textbf{Palabras clave}: nudo, trenza, handle...PONER LAS PALABRAS CLAVE.}\\

\vspace{0.7cm}
\noindent{\textbf{Resumen}}\\

//HACER RESUMEN AQUI!
 
\cleardoublepage


\thispagestyle{empty}

\selectlanguage{USenglish}
\begin{center}
{\large\bfseries PONER TITULO EN INGLES!!!}\\
\end{center}
\begin{center}
Cristina Zuheros Montes.\\
\end{center}

%\vspace{0.7cm}
\noindent{\textbf{Keywords}: knot, braid, handle....PONER AQUI KEYWORDS!!!!!}\\

\vspace{0.7cm}
\noindent{\textbf{Abstract}}\\

//PONER RESUMEN DE MILLON DE PALABRAS INGLES.!!!!!!\\

\selectlanguage{spanish}





\tableofcontents
\listoffigures

\mainmatter
\chapter{Introducción}
\label{ch0}
\section{Contextualización.}
Dentro de la ciencia matemática nos encontramos una rama, conocida como topología, que se encarga del estudio de las propiedades de aquellos cuerpos que, mediante transformaciones continuas sobre los cuerpos, permanecen inalteradas. Aunque no se conoce una fecha exacta del origen de la topología, se suele establecer en 1735 con la resolución, por parte de Leonhard Euler, del problema de los 7 puentes de Königsberg.\\

Dentro de la topología surge la teoría de nudos que sólo tiene algo más de unos 200 años. Se puede decir que es una teoría bastante reciente y cuyos resultados más interesantes se han ido descubriendo en las últimas décadas. Es por esto que la teoría de nudos es muy llamativa tanto para investigadores matemáticos como para biólogos, físicos, químicos, marineros, expertos en supervivencia y una gran cantidad de científicos que encuentra en esta teoría una respuesta a sus planteamientos. Lo atractivo de dicha teoría es que, al ser relativamente reciente, aún tiene muchas cuestiones abiertas en las que se puede trabajar. \\

//COMPLETAR CON ALGO DE LA PARTE DE HISTORIA PONIÉNDOLO CON OTRAS PALABRAS Y TAL.....



\section{Problema abordado.}

\section{Técnicas y áreas matemáticas.}

\section{Contenido de la memoria.}

\section{Fuentes principales.}

\section{Objetivos.}

En la propuesta inicial se propusieron los siguientes objetivos:

\begin{enumerate}
\item Estudiar...//PONER LOS OBJETIVOS INICIALES
\end{enumerate}

//PONER DONDE SE CUBREN LOS OBJETIVOS

\chapter{Teoría de nudos.}
\label{ch1}
\section{Motivación y primeras definiciones.}
  ¿Alguna vez has visto una cuerda con nudos y te has preguntado si podrías deshacerlos sin necesidad de romper la cuerda? ¿Te has planteado si algo tan usual como los nudos pueden estar presentes en áreas esenciales para la vida? Es más, ¿cuál fue el motivo inicial para estudiar dicha teoría de nudos? Quizás te sorprendan las respuestas pero antes de descubrirlo, veamos que se entiende formalmente por un nudo.\\
  
\underline{\textbf{Definición:}}\\
 Un \textbf{nudo} es una curva cerrada en $\mathds{R}^{3}$ que no tiene auto-intersecciones.\\

Veamos algunos ejemplos:

  \begin{figure}[h!]
  	\includegraphics[width=4cm]{inudos/cubo1.png}
  	\includegraphics[width=4cm]{inudos/cubo2.png} 
  	\includegraphics[width=4cm]{inudos/cubo3.png}
  	\centering
  	\caption{De izquierda a derecha: nudo trivial, nudo trébol, nudo de ocho.}
  	\label{trid} 
  \end{figure}

Nos interesa saber cuándo dos nudos son equivalentes: se puede pasar de uno a otro mediante deformaciones. Veámoslo formalmente:\\

Se dice que dos nudos son equivalentes si existe un homeomorfismo de  $\mathds{R}^{3}$ que nos lleve de un nudo al otro. \\


Podemos representar un nudo en el plano visualizando su proyección. Como hay muchas formas de representar un mismo nudo, podremos tener diferentes proyecciones que representen al mismo nudo. 
 Algunos ejemplos básicos son los siguientes:\\
  \begin{figure}[h!]
  	\includegraphics[width=3cm]{inudos/1.jpg}
  	\includegraphics[width=3cm]{inudos/3f.png} 
  	\includegraphics[width=3cm]{inudos/fig8.jpg}
  	\centering
  	\caption{Proyecciones del nudo trivial, nudo trébol, nudo de ocho.}
  	\label{uno} 
  \end{figure}
  
  Podemos ver en ellos una serie de cruces. En concreto en el nudo trébol tenemos 3 cruces y el nudo de ocho tenemos 4 cruces. El nudo trivial destaca por no tener ningún cruce. \\
  
    \underline{\textbf{Definición:}}\\
     Un \textbf{enlace} es una o más curvas cerradas disjuntas en $\mathds{R}^{3}$. Cada una de sus curvas recibe el nombre de componente.\\
     Por ejemplo, el siguiente enlace tiene dos componentes:
  \begin{figure}[h!]
  	\includegraphics[width=2.7cm]{inudos/enlace.png}
  	\centering
  	\label{dos} 
  \end{figure}

    Por tanto, podemos ver un nudo como un caso particular de enlace en el que sólo tenemos una componente.\\
    
  Una de las cuestiones más interesantes en la teoría de nudos es la siguiente: \\
  ¿Dada un nudo, o alguna proyección suya, podremos saber si se trata del nudo trivial?. A lo largo de este proyecto trataremos de dar respuesta, en parte, a dicha cuestión.\\
  \label{1sub1}
\bigskip
\section{Sobre su historia y aplicaciones.}
En el siglo XIX, ciertos físicos escoceses se preguntaban por la estructura de los átomos.\\
Estos científicos tomaron como base la teoría de Descartes, que afirmaba que el \textit{éter} era un fluido que ocupaba todo el espacio y transmitía la luz (éter lumínico), para desarrollar su modelo del átomo. \\
Aunque dichos físicos conocían la existencia de los elementos y que estaban formados por átomos, no conocían la propia estructura de los átomos. \\

Científicos como Peter Guthrie Tait y Willian Thomson llegaron a la teoría de que los átomos se concebían como vórtices, que podríamos ver como remolinos tubulares, en dicho fluido. Estos vórtices se encontraban anudados y en función del tipo de anudamiento darían lugar a un tipo de elemento u otro.\\
De este modo se plantearon que los diferentes nudos corresponderían a los diferentes elementos de la naturaleza. De acuerdo con la teoría, si conociésemos todos los nudos posibles, crearíamos la tabla de elementos que reemplazaría la tabla periódica actual. \\

Para hacernos una idea más clara, para Willian Thomson el nudo trébol podría corresponder con el átomo de helio, el nudo de ocho con el átomo de oxígeno....\\

Numerosos científicos contribuyeron a dicha teoría intentando crear la tabla de nudos pero a finales de este mismo siglo, Michelson-Morley demostró que el éter lumínico no existía y por tanto la teoría de los átomos de vórtice fue descartada. \\

Tras este hecho, la teoría de nudos perdió su interés hasta que fue objeto de estudio en Topología a principios del siglo XX.\\

Posteriormente esta rama de la topología de baja dimensión destacó por su gran interés en áreas como:
  \begin{enumerate}
  	\item Química: ya hemos visto que la teoría de nudos nace en este área.
  	\item Biología: se estudia la teoría de nudos en la estructura de ADN.\\
  	Conocemos como ácido desoxirribonucleico (ADN) a aquella molécula que se encuentra en el núcleo de nuestras células y, por tanto, que contiene nuestro código genético. Se trata de un elemento esencial para la vida. Es muy conocida su forma: se puede ver como dos cuerdas enrolladas formando un doble hélice. \\
  	
  	Su forma de doble hélice puede encontrarse cerrada por los extremos de forma que nos encontraríamos con la propia forma de un nudo. Las ideas que veremos en este proyecto, junto con algunas más, se pueden aplicar a estas estructuras de ADN.\\
  	
  	Además, esta estructura de ADN puede sufrir ciertas alteraciones producidas por la encima topoisomerasa. Lo interesante es que estas alteraciones se corresponden con algunos de los movimientos que veremos para los nudos.\\
  	
  	
  	\item Criptografía \cite{12}: 
  	La fuerte relación que veremos entre la teoría de nudos y la teoría de trenzas, hace que podamos establecer cierta relación entre la teoría de nudos y la criptografía. \\
  	
  	Haciendo uso de la criptografía tratamos de encontrar métodos para que el intercambio de información no sea comprensible por terceras personas.\\
  	
  	Algunos métodos para encriptar dicha información están inspirados en la teoría de trenzas. En concreto el problema de la conjugación de trenzas, que trata de estudiar la igualdad de dos trenzas haciendo uso de una tercera trenza, nos permitirá encriptar la información de forma segura. \\
  \end{enumerate}
\label{1sub2}
\section{Componiendo nudos.}\label{seccion3}
Supongamos que tenemos dos proyecciones J y K de nudos. Podemos definir un nuevo nudo a partir de ellos eliminando un arco de cada una de las proyecciones y conectando los 4 extremos finales de dos en dos mediante otros arcos de modo que no se añadan ni eliminen cruces.\\
A este nudo resultante le llamaremos \textbf{suma conexa} o composición de los dos nudos y se denotará como \textbf{J\#K}. A los nudos originales J y K les llamaremos \textbf{nudos factores}. \\

Por ejemplo, consideremos como nudos factores el nudo trébol y el nudo de ocho. 
\begin{figure}[h!]
	\subfigure[J]{\includegraphics[width=2.5cm]{inudos/conexion1.jpg}} 
	\subfigure[K]{\includegraphics[width=2.5cm]{inudos/fig8.jpg}}
	\centering
	\caption{Nudo trébol y nudo de ocho.}
	\label{comp1} 
\end{figure}

Haciendo la suma conexa de ambos nudos obtenemos el nudo composición:
\begin{figure}[h!]
	\subfigure[Haciendo la composición]{\includegraphics[width=7cm]{inudos/conexion2.jpg} }
	\subfigure[J\#K]{\includegraphics[width=7cm]{inudos/conexion3.jpg}}
	\centering
	\caption{Composición de nudo trébol y nudo de ocho.}
	\label{comp2} 
\end{figure}


El nudo trivial es un elemento identidad para la suma conexa: si hacemos la composición de un nudo cualquiera J con el nudo trivial, vamos a obtener el propio nudo J. Por ejemplo, seguimos considerando J como el nudo trébol y el nudo trivial como K. \\
\begin{figure}[h!]
	\centering
	\subfigure[J]{\includegraphics[width=2.5cm]{inudos/conexion1.jpg} }
	\subfigure[K]{\includegraphics[width=2.5cm]{inudos/1.jpg}}
	\caption{Nudo trébol y nudo trivial.}
	\label{comp3} 
\end{figure}

Su suma conexa nos seguiría dando el nudo factor J, es decir, el nudo trébol.\\

\begin{figure}[h!]
	\subfigure[Haciendo la composición]{\includegraphics[width=8cm]{inudos/conexion4.jpg} }
	\subfigure[J\#K]{\includegraphics[width=4cm]{inudos/conexion1.jpg}}
	\centering
	\caption{Composición de nudo trébol y nudo trivial.}
	\label{comp4} 
\end{figure}

\underline{\textbf{ Definición:}}\\
Diremos que un \textbf{nudo es primo} si no puede ser expresado como la suma conexa de dos nudos, a menos que uno de ellos sea el nudo trivial. \\

\underline{ \textbf{ Definición:}}\\
Diremos que un \textbf{nudo es compuesto} si no es el nudo trivial ni es un nudo primo.\\

Por ejemplo, los nudos trébol y nudo de ocho de la figura \ref{comp1} son nudos primos mientras que el nudo de la figura \ref{comp2} es un nudo compuesto. \\ 

Hay una gran variedad de nudos primos. Cualquier nudo puede ser expresado singularmente como suma conexa de nudos primos. En la tabla \ref{comp6}, extraida de \cite{10}, podemos ver los diferentes nudos primos que tienen menos de 8 cruces.\\
\begin{figure}[h!]
	\includegraphics[width=14cm]{inudos/tableknot.png}
	\centering
	\caption{Tabla de nudos primos.}
	\label{comp6} 
\end{figure}




Finalmente, es importante destacar el hecho de que la elección que hacemos de los arcos que eliminamos de cada uno de los nudos factores afecta al nudo composición. Por tanto, es posible construir dos nudos composición diferentes a partir del mismo par de nudos factores. Veamos esta idea con más detalle, para ello necesitamos:\\

\underline{\textbf{ Definición:}}\\
\textbf{ Un nudo orientado} es un nudo al que se le ha asignado una orientación, es decir, es un nudo que dispone de una dirección de viaje sobre él mismo. Esta orientación se indica mediante flechas en la proyección. \\

\underline{\textbf{ Definición:}}\\
\textbf{ Un nudo es invertible} si es equivalente a sí mismo con la orientación opuesta. \\

El problema de determinar si un nudo cualquiera es o no invertible no es para nada trivial.

Como ejemplo de nudo invertible nos podemos encontrar el nudo trébol, que vemos en la imagen \ref{comp5}.\\
\begin{figure}[h!]
	\includegraphics[width=3.5cm]{inudos/3fcon1.png}
	\includegraphics[width=3.5cm]{inudos/3fcon2.png}
	\centering
	\caption{Ambas orientaciones del nudo trébol.}
	\label{comp5} 
\end{figure}

Sean los dos nudos factores J y K a los que se asignamos una orientación. Tendremos dos formas posibles de hacer la composición: conectar con las orientaciones emparejadas o no emparejadas. \\
Todas las composiciones de los nudos cuyas orientaciones emparejan al componer, darán el mismo nudo composición. Todas las composiciones de los nudos cuyas orientaciones no emparejan al componer, también darán el mismo nudo composición. Sin embargo, es posible que la composición de los nudos cuyas orientaciones emparejen no de lugar al mismo nudo que haciendo la composición de los nudos cuyas orientaciones no emparejen. Serán el mismo si uno de los nudos factores es invertible.\\

Veamos un caso en el que la composición de dos mismos factores, genera nudos diferentes. Consideramos el nudo de la imagen \ref{comp7}.\\

\begin{figure}[h!]
	\includegraphics[width=4cm]{inudos/817con.png}
	\centering
	\caption{Nudo factor J y K.}
	\label{comp7} 
\end{figure}
Si componemos el nudo consigo mismo conectando las orientaciones emparejadas y desemparejadas obtenemos nudos que no son equivalentes. Lo podemos comprobar visualmente en la Figura \ref{comp8}.

\begin{figure}[h!]
	\includegraphics[width=7cm]{inudos/817def1.png}
	\includegraphics[width=7cm]{inudos/817def2.png}
	\centering
	\caption{Las composiciones no son equivalentes.}
	\label{comp8} 
\end{figure}\label{1sub3}
\bigskip
\section{Equivalencia de nudos: movimientos de Reidemeister.}\label{seccion4}
Dos nudos K1 y K2 serán equivalentes (K1 $\thicksim$ K2) si podemos distorsionar uno de ellos en el otro sin hacer ningún corte. \\

Para ver si dos proyecciones corresponden a nudos equivalentes, usaremos el conepto de isotopía plana. Más precisamente, definimos una \textbf{isotopía plana} de las proyecciones P1 y P2 de nudos como la aplicación continua $F: \mathds{R}^{2}$ x $[0,1] \rightarrow \mathds{R}^{2}$ tal que $F_{0}=identidad$, $F_{1}(P1) = P2$ y $F_{t}$ es un homeomorfismo $\forall t$.\\

Los movimientos de Reidemeister que vamos a ver a continuación nos permiten cambiar la proyección de un nudo de modo que se cambie la relación entre los cruces pero que no cambie el nudo al que representa la proyección. \\
Cada uno de estos movimientos es una isotopía:\\

\textbf{Primer movimiento de Reidemeister - R1}\\
En cualquier zona de la proyección nos permite añadir o eliminar un giro tal y como vemos en la figura \ref{movi1}.
  \begin{figure}[h!]
  	\includegraphics[width=5cm]{inudos/movi1.png}
  	\includegraphics[width=5cm]{inudos/movi2.png}
  	\centering
  	\caption{Primer movimiento Reidemeister.}
  	\label{movi1} 
  \end{figure}
  
	\textbf{Segundo movimiento de Reidemeister - R2.}\\
Nos permite añadir o eliminar dos cruces del nudo como se ve en la figura \ref{movi2}.
    \begin{figure}[h!]
    	\includegraphics[width=7cm]{inudos/movi3.png}
    	\includegraphics[width=7cm]{inudos/movi4.png}
    	\centering
    	\caption{Segundo movimiento de Reidemeister.}
    	\label{movi2} 
    \end{figure}
    
	\textbf{Tercer movimiento de Reidemeister - R3.}\\
Nos permite deslizar una hebra del nudo de un lado de un cruce al otro lado del cruce. Veamos la figura \ref{movi3} para aclarar la idea.
      \begin{figure}[h!]
      	\includegraphics[width=7cm]{inudos/movi5.png}
      	\includegraphics[width=7cm]{inudos/movi6.png}
      	\centering
      	\caption{Tercer movimiento de Reidemeister.}
      	\label{movi3} 
      \end{figure}
  

\begin{teo} \textbf{Teorema de Reidemeister.} Sean P1 y P2 las proyecciones que representan a dos nudos K1 y K2, respectivamente. Entonces, K1 $\thicksim$ K2 si, y solo si, P1 y P2 están conectados por una secuencia finita de movimientos de Reidemeister e isotopías planas.
\end{teo}

Veamos un ejemplo en el que vemos la equivalencia de dos proyecciones, que en un primer momento podrían no parecernos equivalentes:
  \begin{figure}[h!]
  	\subfigure[P1]{\includegraphics[width=3cm]{inudos/3f.png}}
  	\subfigure[R1]{\includegraphics[width=2cm]{inudos/flecha.png}}
  	\includegraphics[width=3cm]{inudos/3fseg.png}
  	\subfigure[R3]{\includegraphics[width=2cm]{inudos/flecha.png}}
  	\includegraphics[width=3cm]{inudos/fase3.png}
  	\subfigure[Isotopia]{\includegraphics[width=2cm]{inudos/flecha.png}}
  	\includegraphics[width=3cm]{inudos/fase4.png}
  	\subfigure[R3]{\includegraphics[width=2cm]{inudos/flecha.png}}
  	\includegraphics[width=3cm]{inudos/fase5.png}
  	\subfigure[R1]{\includegraphics[width=2cm]{inudos/flecha.png}}
  	\includegraphics[width=3cm]{inudos/fase6.png}
  	\subfigure[Isotopia]{\includegraphics[width=2cm]{inudos/flecha.png}}
  	\includegraphics[width=3cm]{inudos/fase7.png}
  	\subfigure[Isotopia]{\includegraphics[width=2cm]{inudos/flecha.png}}
  	\subfigure[P2]{\includegraphics[width=3cm]{inudos/fase8.png}}
  	\centering
  	\caption{Equivalencia de dos proyecciones de nudos.}
  	\label{algosj} 
  \end{figure}

Gracias a dicho teorema podremos estudiar si dos proyecciones representan el mismo nudo. Para ello tendremos que encontrar una secuencia de movimientos de Reidemeister que nos lleve de una proyección a la otra. Sin embargo, este proceso puede no tener el número de movimientos intermedios limitado por lo que no tiene mucho sentido implementarlo.\\
 
 Aunque este teorema no nos permita ver de una forma cómoda la equivalencia entre dos nudos en la práctica (por la fuerte complejidad) si que nos permite obtener una conclusión esencial:\\
 
 Si una propiedad de un nudo no cambia al aplicarle cualquiera de estos tres movimientos de Reidemeister, entonces esta propiedad no va a cambiar por muchas deformaciones que se le hagan al nudo. En definitiva, si un nudo cumple cierta propiedad y otro nudo no la cumple, esos nudos no podrán ser equivalentes. Incidiremos en esta idea en la siguiente Sección \ref{seccion5}.
 \label{1sub4}
\bigskip
\section{Algunos invariantes.}
Ya conocemos cuáles son los movimientos básicos que nos permiten determinar si dos trenzas dadas son equivalentes. Pero al igual que ocurría con los nudos, llevar esta idea a la práctica no es factible.\\

En esta sección vamos a ver algunos invariantes de trenzas que nos permitirán determinar de forma relativamente rápida si dos trenzas no son equivalentes.\\

\underline{\textbf{Definición:}} \\
Un \textbf{invariante} de una trenza es una propiedad que no cambia cuando la trenza sufre deformaciones en el espacio. \\

Vamos a ver algunos invariantes de trenzas que usaremos posteriormente en la práctica.

\bigskip
\subsection{Exponente:}\label{invtren1}
\textbf{\underline{Definición:}}\\
Sea $\beta \in B_{n}$ representada como $\beta $= $\sigma_{i_{1}}^{e_{1}} \sigma_{i_{2}}^{e_{2}} ... \sigma_{i_{m}}^{e_{m}}$ donde $e_{i} \in \{-1,1\}$, 1 $\le i_{1}, i_{2},..,i_{m} \le$ n-1.\\
Llamamos \textbf{exponente} de $\beta$ al entero $ \sum_{i=1}^{m} e_{i} $. Se denotará como exp($\beta$).\\

El exponente de la trenza $\beta = \sigma3\sigma2^{-1}\sigma3^{-1}$ de la figura \ref{exp1} es $+1-1-1=-1$.\\
   \begin{figure}[h!]
   	\centering
   	\includegraphics[width=4.3cm]{itrenzas/4c3.png}
   	\caption{$\sigma3\sigma2^{-1}\sigma3^{-1}$}
   	\label{exp1} 
   \end{figure}

\begin{pro}
	El exponente de una trenza $\beta \in B_{n}$ es un invariante. 
	\begin{proof}
		Veamos que dadas las trenzas $\beta1,\beta2 \in B_{n}$ tales que $\beta1 \sim \beta2$, se verifica que exp($\beta1$)=exp($\beta2$).\\
		
		Al ser $\beta1 \sim \beta2$, por la definición \ref{defequi}, sabemos que existe la secuencia de trenzas equivalentes tal que: 
		\begin{center}
			$ \beta1 = \beta1_{0} \rightarrow \beta1_{1} \rightarrow ... \rightarrow \beta1_{m}=\beta2$ 
		\end{center}
	    donde cada par de trenzas $ \beta_{j}, \beta_{j+1}, j=0,..,m-1, $ está relacionado por un movimiento elemental. Estos movimientos elementales se pueden reducir a las relaciones las  \ref{rel1} y \ref{rel2} que definen al grupo $ B_{n} $ . Por tanto,para verificar que el exponente de cada par $ \beta_{j}, \beta_{j+1}$ es igual, tenemos que ver que estas relaciones de igualdad tienen el mismo exponente:
	    
	    \begin{enumerate}
	    \item $\sigma_{i+1}\sigma_{i}\sigma_{i+1} =\sigma_{i}\sigma_{i+1}\sigma_{i} $ siendo $i=2,..,n-2 $:\\
	    exp($ \sigma_{i+1}\sigma_{i}\sigma_{i+1})$ = exp$(\sigma_{i}\sigma_{i+1}\sigma_{i} $).
	    \item $\sigma_{i}\sigma_{j}=\sigma_{j}\sigma_{i}$ siendo $1 \le i < j \le n-1 $, $j-i \geq 2$:\\
	    exp($\sigma_{i}\sigma_{j})$ = exp$(\sigma_{j}\sigma_{i}$).  	
	    \end{enumerate}
	\end{proof}
\end{pro}

Haciendo uso de este invariante podremos ver muy fácilmente si dos trenzas dadas tienen distintos exponente y por tanto no son equivalentes. Podemos ver un ejemplo de dos trenzas no equivalentes en la figura \ref{exp2}. La trenza $\beta1 = \sigma2\sigma1\sigma2^{-1}$ tiene exponente +1, mientras que la trenza $\beta1 = \sigma2^{-1}\sigma1^{-1}\sigma2$ tiene exponente -1.\\ 
	\begin{figure}[h!]
		\centering
		\subfigure[exp($ \beta1 $) = +1]{\includegraphics[width=3.5cm]{itrenzas/6c1.png}}
		\subfigure[exp($ \beta2 $) = -1]{\includegraphics[width=3.5cm]{itrenzas/6c4.png}}
		\caption{Trenzas no equivalentes.}
		\label{exp2} 
	\end{figure}
	
Si tuviesen igual exponente tendríamos que estudiar otros invariantes para ver si las trenzas son iguales. Por ejemplo, en la figura \ref{exp3} vemos que las trenzas $\beta1 = \sigma2\sigma2\sigma3^{-1}$ y $\beta2 = \sigma3^{-1}\sigma2\sigma1$ tienen el mismo exponente (+1) luego no podremos saber si son o no equivalentes. Al estudiar el invariante de la seguiente sección veremos que no lo son.  \\
	\begin{figure}[h!]
		\centering
		\subfigure[exp($ \beta1 $) = +1]{\includegraphics[width=3.5cm]{itrenzas/7c1.png}}
		\subfigure[exp($ \beta2 $) = +1]{\includegraphics[width=3.5cm]{itrenzas/7c2.png}}
		\caption{Trenzas con mismo exponente.}
		\label{exp3} 
	\end{figure}

\bigskip
\subsection{Permutación:}\label{invtren2}
\textbf{\underline{Definición:}}\\
Sea $\beta \in B_{n}$. Consideramos la cadena i que conecta el punto inicial $ A_{i}$ con el punto final $B_{j_{i}}$. Llamaremos \textbf{permutación} asociada a la trenza a la matriz:\\
\[\pi(\beta)=\begin{bmatrix}
1 & 2 & .. & n\\
j_{1} & j_{2} & .. & j_{n} \\
\end{bmatrix}\]
Por simplicidad podremos denotar la permutación de la trenza como el vector $j_{1} j_{2} ... j_{n}$.\\

La permutación de la trenza $\beta = \sigma3\sigma2^{-1}\sigma3^{-1}$ de la figura \ref{perm1} es \[\pi(\beta)=\begin{bmatrix}
1 & 2 & 3 & 4\\
1 & 4 & 3 & 2 \\
\end{bmatrix}\] o bien directamente la permutación: 1 4 3 2.\\
\begin{figure}[h!]
	\centering
	\includegraphics[width=5.5cm]{itrenzas/4c3.png}
	\caption{$\sigma3\sigma2^{-1}\sigma3^{-1}$}
	\label{perm1} 
\end{figure}

\begin{pro}
    La permutación de una trenza $\beta \in B_{n}$ es un invariante. 
    \begin{proof}
    	Veamos que dadas las trenzas $\beta1,\beta2 \in B_{n}$ tales que $\beta1 \sim \beta2$, se verifica que $\pi(\beta1$)=$\pi(\beta2$).\\
    	
    	Por la definición \ref{defequi}, existirá la secuencia de trenzas equivalentes tal que: 
    	\begin{center}
    		$ \beta1 = \beta1_{0} \rightarrow \beta1_{1} \rightarrow ... \rightarrow \beta1_{m}=\beta2$ 
    	\end{center}
    	donde cada par de trenzas $ \beta_{j}, \beta_{j+1}, j=0,..,m-1, $ está relacionado por un movimiento elemental. Los movimientos elementales, por definición, no permiten intercambiar los extremos a los que están sujetas las cuerdas de las trenzas. Por este motivo las permutaciones tienen que ser iguales. \\
    \end{proof}
\end{pro}

Haciendo uso de este invariante podemos ver un ejemplo de dos trenzas que no son equivalentes en la figura \ref{perm2}: la trenza $\beta1 = \sigma2\sigma2\sigma3^{-1}$ tiene permutación 1 2 4 3 mientras que la trenza $\beta2 = \sigma3^{-1}\sigma2\sigma1$ tiene permutación 2 3 4 1.\\ 
	\begin{figure}[h!]
		\centering
		\subfigure[$\pi( \beta1 $) = 1 2 4 3]{\includegraphics[width=4.4cm]{itrenzas/7c1.png}}
		\subfigure[$\pi( \beta2 $) = 2 3 4 1]{\includegraphics[width=4.8cm]{itrenzas/7c2.png}}
		\caption{Trenzas no equivalentes.}
		\label{perm2} 
	\end{figure}


\bigskip
\subsection{Polinomio de Alexander:}\label{invtren3}
En la sección \ref{alenudos} estudiamos el polinomio de Alexander como un invariante para nudos. Por el teorema \ref{teoMarkov}, podremos hacer uso de las trenzas como una base para obtener los invariantes de nudos, en concreto, para estudiar este invariante que implementaremos posteriormente. \\

Para poder obtener el polinomio de Alexander de una trenza, tendremos que ver unos conceptos previos, \cite{3}:

\textbf{\underline{Definición:}}\\
Sea la trenza $\beta $= $\sigma_{i_{1}}^{e_{1}} \sigma_{i_{2}}^{e_{2}} ... \sigma_{i_{m}}^{e_{m}}$ donde $e_{i} \in \{-1,1\}$, 1 $\le i_{1}, i_{2},..,i_{m} \le$ n-1. Podemos definir el homomorfismo
\begin{center}
	 $ \phi_{n} : B_{n}  \rightarrow  M(n,\mathds{Z}[t,t^{-1}])$	 
\end{center}
\[ \phi_{n} ( \sigma_{i}) = \begin{bmatrix}
I_{i-1} &  &  & \\
 & 1-t & t &  \\
 & 1 & 0 &  \\
 &  &  & I_{n-i-1} \\
\end{bmatrix}\]
donde $ I_{i} $ representa a la matriz $ ixi $ identidad y los espacios en blanco representan matrices nulas. LLamaremos a esta representación como \textbf{representación de Burau}.\\

En consecuencia tenemos 
	\[ \phi_{n} ( \sigma_{i}^{-1}) = \phi_{n} ( \sigma_{i})^{-1}= \begin{bmatrix}
	I_{i-1} &  &  & \\
	& 0 & 1 &  \\
	& t^{-1} & 1-t^{-1} &  \\	
	&  &  & I_{n-i-1} \\
	\end{bmatrix}\]


\bigskip
Veamos la matriz de Burau de la trenza $\beta \in B_{4}$ representada como $\beta = \sigma2\sigma3^{-1}$.\\
Matriz Burau de $ \sigma2 $:
	\[ \phi_{4} ( \sigma2) = \begin{bmatrix}
	1 & 0 & 0 & 0 \\
	0 & 1-t & t & 0  \\
	0 & 1 & 0 & 0 \\
	0 & 0 & 0 & 1 \\
	\end{bmatrix}\]

Matriz Burau de $ \sigma3^{-1} $:
\[ \phi_{4} ( \sigma3^{-1}) = \begin{bmatrix}
	1 & 0 & 0 & 0 \\
	0 & 1 & 0 & 0 \\
	0 & 0 & 0 & 1  \\	
	0 & 0 & t^{-1} & 1-t^{-1} \\
\end{bmatrix}\]
 
 Por tanto, la matriz Burau de $\beta = \sigma2\sigma3^{-1}$ es:
 \[ \phi_{4} (\sigma2\sigma3^{-1}) = \phi_{4} (\sigma2) \phi_{4}(\sigma3^{-1}) = \begin{bmatrix}
 1 & 0 & 0 & 0 \\
 0 & 1-t & 0 & t \\
 0 & 1 & 0 & 0  \\	
 0 & 0 & t^{-1} & 1-t^{-1} \\
 \end{bmatrix}\]\\
 
 
\begin{teo}\label{teoalex}
	Sea la trenza $\beta \in B_{n}$ con representación de Burau $\phi_{n}(\beta)$. Supongamos que su trenza cerrada genera al nudo K.\\
    Entonces $\exists k \in \mathds{Z}$ tal que el polinomio ($ \pm $ $ t^{k} )det[\phi_{n}(\beta)$ - $ I_{n} $$ ]_{1,1} $ es un invariante de K. A este polinomio, conocido como polinomio de Alexander, se le denota como $ \triangle_{k} $(t).
\end{teo}

Nota: Sea la matriz cuadrada A de dimensión n y A' la matriz de dimensión n-1 que se obtiene al suprimir en la matriz A la fila i y la columna i. 
Se tiene $ det[A]_{i,i}=det[A'] $.\\ 

Gracias a este teorema podemos afirmar que si dos trenzas $\beta1 \in B_{n}, \beta2 \in B_{m}$ cerradas generan el mismo nudo entonces se verificará la igualdad:\\
det[$\phi_{n}(\beta1)$ - $ I_{n} $$ ]_{1,1}$ = $ \pm t^{k} $det[$\phi_{m}(\beta2)$ - $ I_{m} $$ ]_{1,1}$.\\

Para ver la demostración necesitamos un lema previo y los siguientes conceptos:\\

\newpage
Sea $\phi_{n}(\beta)$ = $M$ = $||a_{ij}||$ $1 \le i,j \le n$, donde $\beta \in B_{n}$. Definimos las siguientes matrices de dimensión nxn:\\
  \[ S = \begin{bmatrix}
  1 & 1 & 1 &... & 1 & 1 \\
  0 & 1 & 1 &... & 1 & 1\\
  0 & 0 & 1 &... & 1 & 1 \\
  ... & ... & ...& ... & 1 & 1 \\	
  0 & 0 & 0 & ... & 1 & 1\\	
  0 & 0 & 0 & ... & 0 & 1\\
  \end{bmatrix} 
  S^{-1} = \begin{bmatrix}
  1 & -1 & 0 & ... & 0 & 0 \\
  0 & 1 & -1 & ... & 0 & 0\\
  0 & 0 & 1 & ... & 0  & 0\\
  ... & ... & ...& ... & 0 & 0  \\	
  0 & 0 & 0 &...& 1 & -1  \\	
  0 & 0 & 0 & ... & 0 & 1 \\
  \end{bmatrix}\]\\
  
 Podemos considerar el producto de matrices $S^{-1}MS$
   \[ M S = \begin{bmatrix}
   a_{1,1} & a_{1,2} & a_{1,3} &... & a_{1,n} \\
   a_{2,1} & a_{2,2} & a_{2,3} &...  & a_{2,n}\\
   ... & ... & ...& ... & ... \\		
   a_{n,1} & a_{n,2} & a_{n,3} & ... & a_{n,n}\\
   \end{bmatrix} 
   \begin{bmatrix}
   1 & 1 & 1 &... & 1  \\
   0 & 1 & 1 &... & 1 \\
   ... & ... & ...& ... & ... \\	
   0 & 0 & 0 & ... & 1\\
   \end{bmatrix}\]
   
   \[= \begin{bmatrix}
   a_{1,1} & a_{1,1}+a_{1,2} & ... & a_{1,1} + a_{1,2}+...+a_{1,n} = 1  \\
   a_{2,1} & a_{2,1}+a_{2,2} &... & a_{2,1} + a_{2,2}+...+a_{2,n} = 1  \\
   ... & ... & ... & ... \\	
   a_{n,1} & a_{n,1}+a_{n,2} & ... & a_{n,1} + a_{n,2}+...+a_{n,n} = 1 \\
   \end{bmatrix}\]\\
  
  Luego la matriz  $S^{-1}MS$ tendrá la siguiente forma:\\
     \[ S^{-1}MS = \begin{bmatrix}
     1 & -1 & 0 &... & 0 \\
     0 & 1 & -1 &...  & 0\\
     ... & ... & ...& ... & ... \\		
     0 & 0 & 0 & ... & 1\\
     \end{bmatrix} 
     \begin{bmatrix}
     a_{1,1} & a_{1,1}+a_{1,2} & ... & a_{1,1} + a_{1,2}+...+a_{1,n} = 1  \\
     a_{2,1} & a_{2,1}+a_{2,2} &... & a_{2,1} + a_{2,2}+...+a_{2,n} = 1  \\
     ... & ... & ... & ... \\	
     a_{n,1} & a_{n,1}+a_{n,2} & ... & a_{n,1} + a_{n,2}+...+a_{n,n} = 1 \\
     \end{bmatrix}\]
     
     \[= \begin{bmatrix}
     a_{1,1}-a_{2,1} & a_{1,1}+a_{1,2}-a_{2,1}-a_{2,2} & ... & 1-1 = 0  \\
     a_{2,1}-a_{3,1} & a_{2,1}+a_{2,2}-a_{3,1}-a_{3,2}&... & 1 - 1=0  \\
     ... & ... & ... & ... \\	
     a_{n,1} & a_{n,1}+a_{n,2} & ... & 1 \\
     \end{bmatrix}\]\\
     
     De un modo más simple, vamos a denotar a dicha matriz del siguiente modo:\\
     \[ S^{-1}MS = \left[\begin{array}{r|r}
     \Lambda (t) & 0 \\ \hline	
     * ... * & 1\\
     \end{array}\right]\]\\
  Nota:  Es claro que $ det[S^{-1}(M-I_{n})S]_{n,n} = det(\Lambda(t)-I_{n-1}) $.\\
  
  \newpage
 \begin{lem}
 	Se verifica la siguiente igualdad:
	 \begin{center}
 		det $(\Lambda(t) - I_{n-1})$ = (1+t+..+$ t^{n-1} $) det$[ M - I_{n}] _{1,1} $
	 \end{center}
	 \begin{proof}
	 	Consideramos la matriz de dimensión (n-1)x(n-1):
	            \begin{center}
	            	$ \Lambda(t) - I_{n-1}= $ 
	            \end{center}	           
	           \[\begin{bmatrix}
	           a_{1,1}-a_{2,1}-1 & a_{1,1}+a_{1,2}-a_{2,1}-a_{2,2} & ... &  a_{1,1}+...+a_{1,n-1}-a_{2,1}-...-a_{2,n-1} \\
	           a_{2,1}-a_{3,1} & a_{2,1}+a_{2,2}-a_{3,1}-a_{3,2}-1&... & a_{2,1}+...+a_{2,n-1}-a_{3,1}-...-a_{3,n-1}  \\
	           ... & ... & ... & ... \\	
	           a_{n-1,1}-a_{n,1} & a_{n-1,1}+a_{n-1,2}-a_{n,1}-a_{n,2} & ... & a_{n-1,1}+...+a_{n-1,n-1}-a_{n,1}-...-a_{n,n-1} \\
	           \end{bmatrix}\]\\
	           
	           Es claro que det $(\Lambda(t) - I_{n-1})$ = det$(S^{T}(\Lambda(t) - I_{n-1})S^{-1})$ donde $S^{T}$ es la matriz traspuesta de dimensión (n-1)x(n-1) de la matriz $S$. Por tanto, vamos a trabajar con la matriz $(S^{T}(\Lambda(t) - I_{n-1})S^{-1})$:
	           
	           \[ (\Lambda(t) - I_{n-1})S^{-1} = (\Lambda(t) - I_{n-1}) \begin{bmatrix}
	           1 & -1 & 0 &... & 0 \\
	           0 & 1 & -1 &...  & 0\\
	           ... & ... & ...& ... & ... \\		
	           0 & 0 & 0 & ... & 1\\
	           \end{bmatrix}=\]
	                
	           \[= \begin{bmatrix}
	           a_{1,1}-a_{2,1}-1 & a_{1,2}-a_{2,2}+1 & ... & a_{1,n-1}-a_{2,n-1}  \\
	           a_{2,1}-a_{3,1} & a_{2,2}-a_{3,2}-1&... &  a_{2,n-1}-a_{3,n-1} \\
	           ... & ... & ... & ... \\	
	           a_{n-1,1}-a_{n,1} & a_{n-1,2}-a_{n,2}&... &  a_{n-1,n-1}-a_{n,n-1} \\
	           \end{bmatrix}\]\\
	           
	           Luego:	          
	           \begin{center}
	            	$  S^{T}(\Lambda(t) - I_{n-1})S^{-1} =$
	           \end{center}
	           \[ = \begin{bmatrix}
	           1 & 0 & 0 &... & 0 \\
	           1 & 1 & 0 &...  & 0\\
	           ... & ... & ...& ... & ... \\		
	           1 & 1 & 1 & ... & 1\\
	           \end{bmatrix}
	           \begin{bmatrix}
	           a_{1,1}-a_{2,1}-1 & a_{1,2}-a_{2,2}+1 & ... & a_{1,n-1}-a_{2,n-1}  \\
	           a_{2,1}-a_{3,1} & a_{2,2}-a_{3,2}-1&... &  a_{2,n-1}-a_{3,n-1} \\
	           ... & ... & ... & ... \\	
	           a_{n-1,1}-a_{n,1} & a_{n-1,2}-a_{n,2}&... &  a_{n-1,n-1}-a_{n,n-1} \\
	           \end{bmatrix}=\]\\
	           
	           \[ = \begin{bmatrix}
	           a_{1,1}-a_{2,1}-1 & a_{1,2}-a_{2,2}+1 & ... & a_{1,n-1}-a_{2,n-1}  \\
	           a_{1,1}-a_{3,1}-1 & a_{1,2}-a_{3,2}&... &  a_{1,n-1}-a_{3,n-1} \\
	           ... & ... & ... & ... \\	
	           a_{1,1}-a_{n,1}-1 & a_{1,2}-a_{n,2}&... &  a_{1,n-1}-a_{n,n-1} \\
	           \end{bmatrix} =
	           \begin{bmatrix}
	           A_{1} - A_{2} \\
	           A_{1} - A_{3} \\
	           ... \\	
	           A_{1} - A_{n} \\
	           \end{bmatrix}\]\\
	           
	           Donde los vectores $A_{i} $ $ 1 \le i \le n $ son las filas de la matriz $[M-I_{n}]_{0,n}$, es decir, tienen la siguiente forma:
			   \begin{center}
		           $ A_{1} = [a_{1,1}-1, a_{1,2},...,a_{1,n-1}] $\\
	               $ A_{2} = [a_{2,1}, a_{2,2}-1,...,a_{2,n-1}] $\\
	               ...\\
	               $ A_{n} = [a_{n,1}, a_{n,2},...,a_{n,n-1}] $\\
		       \end{center}
	           
	           Por tanto det $(\Lambda(t) - I_{n-1})$ = det$(S^{T}(\Lambda(t) - I_{n-1})S^{-1})$ = 
	           \[=det\begin{bmatrix}
	           A_{1} - A_{2} \\
	           A_{1} - A_{3} \\
	           ... \\	
	           A_{1} - A_{n} \\
	           \end{bmatrix}=\]
	           \[=det\begin{bmatrix}
	           A_{1} \\
	           A_{1} - A_{3} \\
	           ... \\	
	           A_{1} - A_{n} \\
	           \end{bmatrix} + det\begin{bmatrix}
	           - A_{2} \\
	           A_{1} - A_{3} \\
	           ... \\	
	           A_{1} - A_{n} \\
	           \end{bmatrix} =det\begin{bmatrix}
	           A_{1} \\
	           - A_{3} \\
	           ... \\	
	           - A_{n} \\
	           \end{bmatrix} + det\begin{bmatrix}
	           - A_{2} \\
	           A_{1} - A_{3} \\
	           ... \\	
	           A_{1} - A_{n} \\
	           \end{bmatrix}=...=\]
	           
	           \[=det\begin{bmatrix}
	           A_{1}  \\
	           - A_{3} \\
	           ... \\	
	           - A_{n} \\
	           \end{bmatrix}+det\begin{bmatrix}
	           -A_{2}  \\
	           A_{1} \\
	           ... \\	
	           - A_{n} \\
	           \end{bmatrix}+...+det\begin{bmatrix}
	           -A_{2}  \\
	           -A_{3} \\
	           ... \\
	           -A_{k}\\
	           A_{1}\\
	           -A_{k+2}
	           ...\\	
	           - A_{n} \\
	           \end{bmatrix}+...+det\begin{bmatrix}
	           -A_{2}  \\
	           - A_{3} \\
	           ... \\	
	           - A_{n-1} \\
	           A_{1}\\
	           \end{bmatrix}+det\begin{bmatrix}
	           - A_{2}  \\
	           - A_{3} \\
	           ... \\	
	           - A_{n} \\
	           \end{bmatrix}\]
	           
	           Para obtener el valor de estos determinantes vamos a hacer uso de la siguiente igualdad:
	           \begin{center}
	           	$ det[M-I_{n}]_{p,q} = (-1)^{p-q}t^{p-1}det[M-I_{n}]_{1,1}	 $ $1\le p,q \le n$  
	           \end{center}    
	           
	           De modo que tomando $ p=k+1 $ y $ q=n $, se tiene:     
	           
	           \[det\begin{bmatrix}
	           -A_{2}  \\
	           -A_{3} \\
	           ... \\
	           -A_{k}\\
	           A_{1}\\
	           -A_{k+2}
	           ...\\	
	           - A_{n} \\
	           \end{bmatrix}=(-1)^{n-k-1}det[M-I_{n}]_{k+1,n} = t^{k}det[M-I_{n}]_{1,1}\]
	           
	           Por tanto, det $(\Lambda(t) - I_{n-1})$ = (1+t+..+$ t^{n-1} $) det$[ M - I_{n}] _{1,1} $.
	 \end{proof}
 \end{lem}
 

 \begin{cor}\label{corlem}
 		Se verifica la siguiente igualdad:
 		\begin{center}
 			$ det[S^{-1}(M-I_{n})S]_{n,n} = det(\Lambda(t)-I_{n-1}) $ = (1+t+..+$ t^{n-1} $) det$[ M - I_{n}] _{1,1} $
 		\end{center}
 \end{cor}
 
 \bigskip
 Ya estamos en condiciones de demostrar el teorema \ref{teoalex}:
 \begin{proof}
 	Por el teorema \ref{teoMarkov} sabemos que es suficiente con probar las siguientes igualdades, donde $ \gamma, \beta \in B_{n} $ :
 	\begin{itemize}
 		\item Movimiento elemental M1: \\
 		$ det[\phi_{n}(+\sigma_{i}-\sigma_{i}) - I_{n}]_{1,1} = det[\phi_{n}(-\sigma_{i}+\sigma_{i}) - I_{n}]_{1,1}$, siendo $i<n$
 		\item Movimiento elemental M2: \\
 		$ det[\phi_{n}(\sigma_{i}\sigma_{i+1}\sigma_{i}) - I_{n}]_{1,1} = det[\phi_{n}(\sigma_{i+1}\sigma_{i}\sigma_{i+1}) - I_{n}]_{1,1}$, siendo $ i+1<n $.
 		\item Movimiento elemental M3:\\
 		$ det[\phi_{n}(\sigma_{i}\sigma_{j}) - I_{n}]_{1,1} = det[\phi_{n}(\sigma_{j}\sigma_{i}) - I_{n}]_{1,1}$, siendo $i<n, |i-j| > 1$
 		\item Verificando conjugación Mv1: $ det[\phi_{n}(\gamma\beta\gamma^{-1}) - I_{n}]_{1,1} = det[\phi_{n}(\beta) - I_{n}]_{1,1}$ 
 		\item Verificando estabilización Mv2: $ det[\phi_{n+1}(\beta\sigma_{n}) - I_{n+1}]_{1,1} = det[\phi_{n}(\beta) - I_{n}]_{1,1}$   
  	\end{itemize}
	 Para demostrar las igualdades referentes a los movimientos elementales basta con probar las siguientes igualdades:
	 \begin{itemize}
	 	 \item Para el movimiento elemental M1: \\
	 	 $ \phi_{n}(+\sigma_{i}-\sigma_{i}) = \phi_{n}(-\sigma_{i}+\sigma_{i})$, siendo $i<n$.\\
	 	 Esta igualdad es clara pues por definición $-\sigma_{i}$ es la matriz inversa de $\sigma_{i}$  luego $ \phi_{n}(+\sigma_{i}-\sigma_{i}) = \phi_{n}(+\sigma_{i})\phi_n(-\sigma_{i}) = \phi_{n}(-\sigma_{i})\phi_n(+\sigma_{i}) = \phi_{n}(-\sigma_{i}+\sigma_{i})$
	 	 
	 	 \item Para el movimiento elemental M2: \\
	  	 $ \phi_{n}(\sigma_{i}\sigma_{i+1}\sigma_{i}) = \phi_{n}(\sigma_{i+1}\sigma_{i}\sigma_{i+1})$, siendo $ i+1<n $.\\
	  	 Sin pérdida de generalidad podemos suponer $i=1, n=3$ de modo que:
	  	 	\[ \phi_{3}(\sigma_{1}\sigma_{2}\sigma_{1}) = 
	  	 	\begin{bmatrix}
	  	 	  1-t & t & 0  \\
	  	 	  1 & 0 & 0 \\
	  	 	  0 & 0 & 1 \\
	  	 	\end{bmatrix}\begin{bmatrix}
	  	 	1 & 0 & 0 \\
	  	 	0 & 1-t & t \\
	  	 	0 & 1 & 0 \\
	  	 	\end{bmatrix}\begin{bmatrix}
	  	 	1-t & t & 0  \\
	  	 	1 & 0 & 0 \\
	  	 	0 & 0 & 1 \\
	  	 	\end{bmatrix}=\]
	  	 	
	  	 	\[ = 
	  	 	\begin{bmatrix}
	  	 	1-t & t & 0  \\
	  	 	1 & 0 & 0 \\
	  	 	0 & 0 & 1 \\
	  	 	\end{bmatrix}\begin{bmatrix}
	  	 	1-t & t & 0  \\
	  	 	1-t & 0 & t \\
	  	 	1 & 0 & 0 \\
	  	 	\end{bmatrix}=
	  	 	\begin{bmatrix}
	  	 	(1-t)^{2}+(1-t)t = 1-t & (1-t)t & t^{2}  \\
	  	 	1-t & t & 0 \\
	  	 	1 & 0 & 0 \\
	  	 	\end{bmatrix}=\]
	  	 	
	  	 	\[\begin{bmatrix}
	  	 	1-t& (1-t)t & t^{2}  \\
	  	 	1-t & t & 0 \\
	  	 	1 & 0 & 0 \\
	  	 	\end{bmatrix} = 
	  	 	\begin{bmatrix}
	  	 	1 & 0 & 0 \\
	  	 	0 & 1-t & t \\
	  	 	0 & 1 & 0 \\
	  	 	\end{bmatrix}\begin{bmatrix}
	  	 	1-t & (1-t)t & t^{2}  \\
	  	 	1 & 0 & 0 \\
	  	 	0 & 1 & 0 \\
	  	 	\end{bmatrix}= \]
	  	 	\[\begin{bmatrix}
	  	 	1 & 0 & 0 \\
	  	 	0 & 1-t & t \\
	  	 	0 & 1 & 0 \\
	  	 	\end{bmatrix}\begin{bmatrix}
	  	 	1-t & t & 0  \\
	  	 	1 & 0 & 0 \\
	  	 	0 & 0 & 1 \\
	  	 	\end{bmatrix}\begin{bmatrix}
	  	 	1 & 0 & 0 \\
	  	 	0 & 1-t & t \\
	  	 	0 & 1 & 0 \\
	  	 	\end{bmatrix}= \phi_{3}(\sigma_{2}\sigma_{1}\sigma_{2})\]
	  	 	
	  	 
	 	 \item Para el movimiento elemental M3:\\
	 	 $ \phi_{n}(\sigma_{i}\sigma_{j}) = \phi_{n}(\sigma_{j}\sigma_{i})$, siendo $i<n, |i-j| > 1$\\
	 	 Sin pérdida de generalidad podemos suponer $ i+1<j $ de modo que:
	 	 \[ \phi_{n} (\sigma_{i}) \phi_{n} (\sigma_{j}) = \begin{bmatrix}
	 	 I_{i-1} &  &  & \\
	 	 & 1-t & t &  \\
	 	 & 1 & 0 &  \\
	 	 &  &  & I_{j-i-2} \\
	 	 & & & & 1-t & t &  \\
	 	 & & & & 1 & 0 &  \\
	 	 & & & & & & I_{n-j-1}  \\
	 	 \end{bmatrix}= \phi_{n} (\sigma_{j}) \phi_{n} (\sigma_{i})\]
 	 \end{itemize}
 	 
 	 Veamos ahora que se verifican las igualdades referentes a los movimientos de Markov.
 	 \begin{itemize}
 	 	\item 
 	 Veamos que se verifica la igualdad para el primer movimiento de Markov, es decir, veamos que se verifica $ det[\phi_{n}(\gamma\beta\gamma^{-1}) - I_{n}]_{1,1} = det[\phi_{n}(\beta) - I_{n}]_{1,1}$:\\
 	 Consideramos las matrices $ S $ y $ S^{-1} $ que definimos anteriormente y definimos los productos de matrices     
 	 \[S^{-1}\phi_{n}(\gamma)S = \left[\begin{array}{r|r}
 	 \Lambda (\gamma) & 0 \\ \hline	
 	 * ... * & 1\\
 	 \end{array}\right];
 	 S^{-1}\phi_{n}(\beta)S = \left[\begin{array}{r|r}
 	 \Lambda (\beta) & 0 \\ \hline	
 	 * ... * & 1\\
 	 \end{array}\right];
 	 S^{-1}\phi_{n}(\gamma^{-1})S = \left[\begin{array}{r|r}
 	 \Lambda (\gamma)^{-1} & 0 \\ \hline	
 	 * ... * & 1\\
 	 \end{array}\right]\]\\
 	 
 	 De modo que  	 
 	 \[ S^{-1}\phi_{n}(\gamma\beta\gamma^{-1})S = \left[\begin{array}{r|r}
 	 \Lambda (\gamma)\Lambda (\beta)\Lambda (\gamma)^{-1} & 0 \\ \hline	
 	 * ...................... * & 1\\
 	 \end{array}\right]\]\\
	 Por el corolario \ref{corlem} sabemos que \\
	 $ (1+t+..+ t^{n-1} ) det[ \phi_{n}(\gamma\beta\gamma^{-1}) - I_{n}] _{1,1} = det[S^{-1}(\phi_{n}(\gamma\beta\gamma^{-1}))S-I_{n}]_{n,n}$\\
	 
	 Desarrollando este segundo término tenemos:
	 \[det[S^{-1}(\phi_{n}(\gamma\beta\gamma^{-1}))S-I_{n}]_{n,n} = det[\Lambda (\gamma)\Lambda (\beta)\Lambda (\gamma)^{-1}-I_{n-1}] =\]
	 \[
	 det[\Lambda (\gamma)(\Lambda (\beta)-I_{n-1})\Lambda (\gamma)^{-1}] =
	 det[\Lambda (\beta)-I_{n-1}] \]
	 
	 Aplicando de nuevo el corolario \ref{corlem} tenemos la siguiente igualdad\\
	 $det[\Lambda (\beta)-I_{n-1}]  = (1+t+..+ t^{n-1} ) det[ \phi_{n}(\beta) - I_{n}] _{1,1} $\\
	 
	 Luego obtenemos la igualdad:\\
	  $ (1+t+..+ t^{n-1} ) det[ \phi_{n}(\gamma\beta\gamma^{-1}) - I_{n}] _{1,1} = (1+t+..+ t^{n-1} ) det[ \phi_{n}(\beta) - I_{n}] _{1,1} $\\
	  
	  y podemos concluir
	  $ det[ \phi_{n}(\gamma\beta\gamma^{-1}) - I_{n}] _{1,1} = det[ \phi_{n}(\beta) - I_{n}] _{1,1} $\\
	  
	  \item 
	  Por último vamos a ver que se verifica la igualdad para el segundo movimiento de Markov, es decir, veamos que se verifica $ det[\phi_{n+1}(\beta\sigma_{n}) - I_{n+1}]_{1,1} = det[\phi_{n}(\beta) - I_{n}]_{1,1}$ .\\
	  
	  Llamemos $ M = \phi_{n}(\beta) $ de modo que 
 	 \[ \phi_{n+1}(\beta) = \left[\begin{array}{r|r}
 	  M & 0 \\ \hline	
 	  0 & 1\\
 	 \end{array}\right]\]\\	  
 	 
 	 Por otra parte sabemos que 
 	 \[ \phi_{n+1} (\sigma_{n}) = \begin{bmatrix}
 	 I_{n-1} &  &  \\
 	 & 1-t & t  \\
 	 & 1 & 0  \\
 	 \end{bmatrix}\]\\
 	 
 	 Por tanto, 
 	 \[ \phi_{n+1} (\beta\sigma_{n}) = \begin{bmatrix}
 	 & & & a_{1,n}(1-t) & a_{1,n}t\\
 	 & M_{n-1,n-1} & & ... & ...\\
 	 & & & a_{n-1,n}(1-t) & a_{n-1,n}t\\
 	 a_{n,1}& ... & a_{n,n-1} & a_{n,n}(1-t) & a_{n,n}t\\
 	 & & & 1 & 0\\
 	 \end{bmatrix}\]\\
	  
	 De este modo se tiene 
 	 \[det[\phi_{n+1}(\beta\sigma_{n}) - I_{n+1}]_{n,n} = det \left[\begin{array}{r|r}
    	M_{n-1,n-1} - I_{n-1} & \\ \hline	
 	    & -1\\
 	 \end{array}\right]\] 
 	 \begin{center}
 	 	$ = -det[M_{n-1,n-1}-I_{n-1}] = -det[\phi_{n}(\beta)-I_{n}]_{n,n} $ 
 	 \end{center} 
 	 
 	 Obteniendo la igualdad
 	  	 \begin{center}
 	  	 	$ det[\phi_{n+1}(\beta\sigma_{n}) - I_{n+1}]_{1,1} = det[\phi_{n}(\beta)-I_{n}]_{1,1} $ 
 	  	 \end{center}
      \end{itemize}
 \end{proof}


De este modo, el polinomio de Alexander de la trenza $\beta1 = \sigma1$ se obtendría del siguiente modo:\\
Por definición tenemos que  
\[ \phi_{2} (\beta1) = \begin{bmatrix}
1-t & t  \\
1 & 0 \\
\end{bmatrix}\]
luego 
\[ \phi_{2} (\beta1) - I_{2}= \begin{bmatrix}
-t & t  \\
1 & -1 \\
\end{bmatrix}\]
Finalmente tenemos 
\[ det(\phi_{2} (\beta1) - I_{2})_{1,1} = det(\begin{bmatrix}
-1 \\
\end{bmatrix}) = -1\].

Veamos ahora el polinomio de Alexander de una trenza más compleja, en concreto de la trenza $\beta2 = \sigma2\sigma3^{-1}$. Ya sabemos que 
 \[ \phi_{4} (\beta2) = \begin{bmatrix}
 1 & 0 & 0 & 0 \\
 0 & 1-t & 0 & t \\
 0 & 1 & 0 & 0  \\	
 0 & 0 & t^{-1} & 1-t^{-1} \\
 \end{bmatrix}\]
 luego
  \[ \phi_{4} (\beta2) - I_{4} = \begin{bmatrix}
  0 & 0 & 0 & 0 \\
  0 & -t & 0 & t \\
  0 & 1 & -1 & 0  \\	
  0 & 0 & t^{-1} & t^{-1} \\
  \end{bmatrix}\].
  
  Finalmente tenemos 
    \[ det(\phi_{4} (\beta2) - I_{n})_{1,1} = det(\begin{bmatrix}
    -t & 0 & t \\
     1 & -1 & 0  \\	
     0 & t^{-1} & t^{-1} \\
    \end{bmatrix}) = -1+1 = 0.\].
    
Por tanto tenemos que las trenzas $\beta1$ y $\beta2$ no pueden generar nudos equivalentes pues sus polinomios de Alexander son distintos.\\
    
\label{1sub5}
\bigskip
\section{Notación de nudos.}
Ya sabemos qué son los nudos y algunas nociones esenciales sobre los mismos, pero aún no sabemos asociar una notación concreta a un nudo. En esta sección vamos a ver algunas de las notaciones más comunes de los nudos. 

\begin{center}
	\item \subsection{Notación de Dowker:}
\end{center}
Se trata de una notación muy sencilla para describir la proyección de un nudo. La notación en sí es una secuencia de números enteros, veamos cómo se obtiene:\\

Consideramos una orientación en la proyección de n cruces y asignamos el valor 1 al primer cruce que nos encontremos. Continuamos por la proyección y asignamos el valor 2 al siguiente cruce. Vamos repitiendo el proceso hasta pasar por cada cruce un par de veces (una vez por el undercrossing y otra por el overcrossing). Como resultado tendremos un número par y un número impar por cada cruce de la proyección. \\

Finalmente es necesario asignar los signos a cada uno de estos $2n$ números. Los números pares que correspondan a un overcrossing tendrán signo negativo. Veamos un ejemplo con el nudo trébol que podemos ver en la figura \ref{dow1}:\\
   \begin{figure}[h!]
   	\centering
   	\includegraphics[width=4cm]{inudos/3fcon2dow.png}
   	\caption{Numeración de cruces-Dowker.}
   	\label{dow1} 
   \end{figure}
   
  Tendríamos los pares (1,-4), (3,-6) y (5,2). La notación de Dowker sería -4 -6 2 pues nos quedamos únicamente con los números pares en el orden que indican los números impares.\\

\begin{center}
	\item \subsection{Notación de Gauss:}
\end{center}
La notación de Gauss es una notación parecida a la de Dowker. Consideremos de nuevo una orientación en la proyección de n cruces de un nudo.\\

En este caso cada vez que pasamos por un cruce, tendremos un solo número asignado. De este modo la secuencia de números de la notación se compone de $2n$ elementos con valores desde 1 hasta n, cada uno de ellos repetido dos veces.\\

Consideramos una orientación en la proyección de n cruces y asignamos el valor 1 al primer cruce que nos encontremos. Realizamos el siguiente proceso: continuamos por la proyección hasta el siguiente cruce. Si ya hemos pasado por él, anotamos el número de cruce que tenga asociado. Si no hemos pasado anteriormente por él, asignamos el siguiente número de cruce.\\

Finalmente es necesario asignar los signos a cada uno de esos $2n$ números. Los números que representen uncrossings tendrán signo negativo. Veamos la notación de Gauss para el nudo trébol.\\
   \begin{figure}[h!]
   	\centering
   	\includegraphics[width=4cm]{inudos/3fcon2gaus.png}
   	\caption{Numeración de cruces-Gauss.}
   	\label{gaus1} 
   \end{figure} 
Si vamos haciendo el recorrido partiendo desde el punto grueso indicado tendríamos la secuencia de números: -1 2 -3 1 -2 3. Esta sería su notación de Gauss.

\begin{center}
	\item \subsection{Notación de Conway:}
\end{center}
Por último vamos a ver una notación que puede resultar algo más compleja pero que tiene gran uso e interés, sobre todo en la tería del ADN. \\

\underline{\textbf{Definición:}}\\
Un \textbf{enredo} o tangle de una proyección de un enlace es una región de la proyección rodeada por una bola de modo que las dos cuerdas enlazadas de la proyección tocan la bola exactamente en cuatro puntos. A estos puntos los denotaremos como NO, NE, SO y SE.\\

En la figura \ref{conw1} vemos un enredo general y un ejemplo particular de un enredo.\\
   \begin{figure}[h!]
   	\centering
   	\subfigure[Enredo general]{\includegraphics[width=4.5cm]{inudos/en1.png}}
   	\subfigure[Ejemplo de enredo]{\includegraphics[width=4.5cm]{inudos/en2.png}}
   	\caption{}
   	\label{conw1} 
   \end{figure} 

Pero la idea es trabajar con enredos más sencillos como son los que vemos en la figura \ref{conw2}. La notación se corresponde con el número de cruces que tiene el enredo con signo positivo y negativo según se ve en las imágenes.\\
   \begin{figure}[h!]
   	\centering
   	\subfigure[Enredo $\infty$]{\includegraphics[width=3cm]{inudos/en3.png}}
   	\subfigure[Enredo 0]{\includegraphics[width=3cm]{inudos/3enrot.png}}
   	
   	\subfigure[Enredo 4]{\includegraphics[width=8cm]{inudos/en4.png}}
   	
   	\subfigure[Enredo -4]{\includegraphics[width=8cm]{inudos/en4conneg.png}}
   	\caption{Tipos básicos de enredos.}
   	\label{conw2} 
   \end{figure} 

Haciendo uso de estos enredos básicos podemos construir nuevos enredos uniendo, respectivamente, los extremos NE y SE de un enredo con los extremos NO y SO del otro enredo. A esta operación se le conoce como suma de enredos.\\
Además, disponemos de una operación de multiplicación: reflejaremos el primer enredo y hacemos la operación de suma.\\
A los enredos construidos con estas operaciones se les conoce con \textbf{enredos racionales}. Podemos ver un esquema básico estas dos operaciones en la imagen \ref{conw3}:\\
   \begin{figure}[h!]
   	\centering
   	\subfigure[Operacion suma]{\includegraphics[width=8cm]{inudos/ensum.png}}
   	\subfigure[Operacion multiplicacion]{\includegraphics[width=8cm]{inudos/enmul.png}}
   	\caption{Operaciones con enredos.}
   	\label{conw3} 
   \end{figure}

En la figura \ref{conw4} podemos ver un ejemplo de un enlace más complejo:
\begin{figure}[h!]
	\centering
	\includegraphics[width=4cm]{inudos/en5fin.png}
	\caption{Enlace -2 3 2}
	\label{conw4} 
\end{figure}

A partir de estos enredos podemos construir proyecciones de nudos enlazando los puntos NO con NE y los puntos SO con SE. La notación del nudo corresponde con la notación que le damos a su enredo.\\ 

Nos interesa ver si dos proyecciones de nudos representan al mismo nudo, luego nos interesa ver si dos enredos son equivalentes. Veamos qué quiere decir que sean equivalentes:\\

\underline{\textbf{Definición:}}\\
Diremos que dos enredos son equivalentes si podemos pasar de un enredo al otro mediante los movimientos de Reidemeister, que vimos en la sección \ref{seccion4}, manteniendo los cuatro extremos fijos en la bola imaginaria. \\

Ver si dos enredos son equivalentes por definición no es viable de modo que vamos a aplicar otro método: consiste en calcular la fracción continua asociada a cada enredo.\\

\underline{\textbf{Definición:}}\\
Sea un enredo con notación $a_{n}...a_{1}a_{0}$. Su\textbf{ fracción continua} es una expresión del tipo:
\begin{equation}
    a_{0} + \frac{1}{a_{1} + \frac{1}{a_{2} + \frac{1}{...a_{n}}}}
\end{equation}
siendo $a_{i} \in \mathds{Z},  \forall i = 0,1,..,n.$\\

\begin{teo}
Dos enredos racionales son equivalentes si y solo si sus fracciones continuas toman el mismo valor. 
\end{teo}

Veamos un ejemplo. Consideramos los enredos racionales -2 3 2 y 3 -2 3 que vemos en la figura \ref{conw5}.\\
\begin{figure}[h!]
	\centering
	\subfigure[Enlace -2 3 2]{\includegraphics[width=4cm]{inudos/en5fin.png}}
	\space
	\subfigure[Enlace 3 -2 3]{\includegraphics[width=4cm]{inudos/en6fin.png}}
	\caption{Enlaces equivalentes}
	\label{conw5} 
\end{figure}

A simple vista resulta difícil confirmar que sean equivalentes. Vamos a obtener sus fracciones continuas asociadas:\\
$-2 \hspace{1cm} 3 \hspace{1cm} 2 \hspace{1cm}$ tiene fracción continua
\begin{equation}
 2 + \frac{1}{3 + \frac{1}{-2}} = \frac{12}{5}
\end{equation}

$3 \hspace{1cm} -2 \hspace{1cm} 3 \hspace{1cm}$ tiene fracción continua
\begin{equation}
3 + \frac{1}{-2 + \frac{1}{3}} = \frac{12}{5}
\end{equation}
Ambas fracciones continuas son iguales, luego los enredos -2 3 2 y 3 -2 3 son equivalentes. 
\label{1sub6}
\section{Conexión con distintas teorías.}\label{seccion7}
\begin{center}
	\item \subsection{Teoría de grafos:}
\end{center}

\underline{\textbf{Definición:}}\\
Un \textbf{grafo} es un par $ (V,A) $ de conjuntos, junto con la aplicación 
\begin{center}
	 $  \gamma :A \rightarrow$ \{\{$u,v$\} / $u,v \in V$\}  
\end{center}
Al conjunto de puntos V se se llama conjunto de vértices y al conjunto A le llamaremos conjunto de aristas. \\

\underline{\textbf{Definición:}}\\
Un \textbf{grafo plano} $ G $ es un grafo que permanece en el plano.\\

Podemos ver varios ejemplos en la figura \ref{graf1}.\\
\begin{figure}[h!]
	\centering
	\includegraphics[width=3.5cm]{inudos/grafo.png}
	\includegraphics[width=4cm]{inudos/pgrafo.png}
	\caption{Ejemplos de grafos planos.}
	\label{graf1} 
\end{figure}

A partir de la proyección de un nudo (o de un enlace en general) podemos generar su grafo plano asociado. Para ello tendremos que realizar el siguiente proceso:\\

Sombreamos las regiones de la proyección que estén de modo que la región externa al nudo se quede sin sombrear y situamos un vértice en cada zona. Unimos los vértices con aristas que pasan por los cruces de la proyección. Ya tendríamos el grafo plano. Además, si el nudo tiene asignada una orientación, podremos asignarle el tipo de cruce (positivo o negativo) a cada arista. Podemos ver un ejemplo en la figura \ref{graf2}.\\
\begin{figure}[h!]
	\centering
	\includegraphics[width=5cm]{inudos/pgrafo3.png}
	\includegraphics[width=5cm]{inudos/pgrafo2.png}
	\includegraphics[width=5cm]{inudos/pgrafo1.png}
	\caption{De proyección a grafo}
	\label{graf2} 
\end{figure}

Finalmente, para ver que los problemas de nudos se pueden ver como problemas de grafos y viceversa, vamos a ver el procedimiento inverso. Dado un grafo plano, podremos obtener la proyección del nudo asociado. Veamos cuál sería el procedimiento:\\

Partiendo del grafo plano con los signos asociados en cada vértice, marcamos cada una de las aristas. Uniremos cada una de estas marcas con aquellas marcas que estén en las aristas que conectan con los vértices de la arista que tiene la marca. A continuación, sombreamos las zonas que contienen a cada vértice. Finalmente, establecemos los cruces conforme a los signos del grafo plano. Podemos ver un ejemplo en la figura \ref{graf3}.\\
\begin{figure}[h!]
	\centering
	\includegraphics[width=4.5cm]{inudos/grafo1.png}
	\includegraphics[width=4.5cm]{inudos/grafo2.png}
	\includegraphics[width=4.5cm]{inudos/grafo3.png}
	\includegraphics[width=4.5cm]{inudos/grafo4.png}
	\caption{De grafo a proyección.}
	\label{graf3} 
\end{figure}



\begin{center}
	\item \subsection{Teoría de trenzas:}
\end{center}
En esta sección vamos a introducir la relación que hay entre teoría de nudos y teoría de trenzas, teoría que estudiaremos con mayor detalle en el próximo tema.\\

Vamos a ver la idea general de lo que se entiende por el término trenza y veremos una definición más precisa más adelante. Podemos pensar en una trenza como un conjunto de $n$ cadenas que son atadas a un tope imaginario arriba y abajo. Podemos ver algunos ejemplos de trenzas en la figura \ref{ntren1}.\\
   \begin{figure}[h!]
   	\centering
   	\includegraphics[width=3.5cm]{itrenzas/t4.png}
   	\space
   	\includegraphics[width=6cm]{itrenzas/t7.png}
   	\caption{Ejemplos de trenzas}
   	\label{ntren1} 
   \end{figure} 

A partir de una trenza, podemos obtener su nudo o enlace correspondiente. Simplemente tendremos que unir en orden  los topes superiores de las cadenas con los inferiores. Esta trenza cerrada será el nudo al que representa la trenza. Podemos ver algunos ejemplos en la sección \ref{Markov}.\\

Para ver el proceso inverso haremos uso del siguiente teorema:
\begin{teo}Teorema de Alexander.\\
	Todo nudo puede ser representado como una trenza cerrada.
\end{teo}

Para ver la demostración de dicho teorema podemos inspirarnos en varias ideas \cite{13}, \cite{14}. En este caso vamos a ver el algoritmo de Yamada-Vogel, que transforma un nudo en una trenza cerrada. Este proceso se puede realizar a cualquier nudo y el resultado es siempre una trenza cerrada. 
\begin{proof}
	Dada nudo orientado K, realizaremos los siguientes pasos:
	\begin{enumerate}
		\item A partir de la proyección D del nudo K, vamos a obtener su imagen de Seifert S realizando el siguiente proceso:\\
		Sabemos que en cada cruce de la proyección D nos encontramos dos hebras entrando al cruce y dos hebras salientes. Vamos a eliminar el cruce conectando cada una de las hebras que entran al cruce con las hebras adyacentes que salen del mismo. Podemos ver la idea en la figura \ref{prueale1}.\\
		
		\begin{figure}[h!]
			\centering
			\includegraphics[width=12cm]{inudos/ima8.png}
			\caption{Transformando la proyección de un cruce.}
			\label{prueale1} 
		\end{figure}
		
		
		Como resultado, vamos a obtener un conjunto de círculos en el plano a los que se les conoce como círculos de Seifert. \\
		Para no perder el tipo de cruce, vamos a conservar la conexión de los cruces y asignaremos el símbolo + a los cruces positivos y el símbolo - a los cruces negativos. De este modo, conseguimos conservar toda la información sobre la proyección del nudo. 
		
		\item Antes de ver cómo realizar este segundo paso, necesitamos unos conceptos previos. \\
	
		Sean dos círculos de Seifert C1 y C2. Diremos que son incoherentes si no existe un arco que los conecte.\\
		Definimos la altura de la proyección D (h(D)) como el número de pares de círculos de Seifert incoherentes.\\
		
		Si h(D)=0, entonces D ya representa una trenza cerrada, finalizo.\\
		Si h(D)$ > $0, podemos encontrar un arco que una dos pares de círculos de Seifert incoherentes (se conoce como arco de reducción). Se puede de ver cómo representaremos un arco de reducción entre dos círculos de Seifert en la figura \ref{prueale2}.
		\begin{figure}[h!]
			\centering
			\includegraphics[width=11cm]{inudos/ima10.png}
			\caption{Arco de reducción.}
			\label{prueale2} 
		\end{figure}
		
		 Este arco de reducción nos sirve de guía para realizar un movimiento de reducción sobre ambos círculos de Seifert de modo que obtenemos la primera proyección de la figura \ref{prueale3}. Con las siguientes imágenes de la figura vemos que efectivamente se conservan los círculos de Seifert originales pero hemos conseguido que ahora sean coherentes. 
		\begin{figure}[h!]
			\centering
			\includegraphics[width=13cm]{inudos/ima11.png}
			\caption{Movimiento de reducción.}
			\label{prueale3} 
		\end{figure}
	
		
		\item Continuamos realizando movimientos de reducción hasta obtener h(D)=0.\\
		
		Se puede demostrar \cite{13} que dicho algoritmo finaliza haciendo uso del siguiente lema:\\
		Supongamos que realizamos un movimiento de reducción al a proyección D, obteniendo la proyección D'. En ese caso, h(D')=h(D)-1.
		
	\end{enumerate}
\end{proof} 

En la figura \ref{prueale4} aplicamos el algoritmo a un nudo particular. 
		\begin{figure}[h!]
			\centering
			\subfigure[Proyección nudo]{\includegraphics[width=4.5cm]{inudos/ima1.png}}
			\subfigure[Círculos Seifert]{\includegraphics[width=4cm]{inudos/ima3.png}}
			\caption{Movimiento de reducción.}		
			\subfigure[Círculos Seifert]{\includegraphics[width=4cm]{inudos/ima4.png}}
			\subfigure[Arco reducción]{\includegraphics[width=4cm]{inudos/ima5.png}}
			
			\subfigure[Movimiento reducción]{\includegraphics[width=7.8cm]{inudos/ima6.png}}
			\subfigure[Movimiento reducción]{\includegraphics[width=7.8cm]{inudos/ima7.png}}
			\subfigure[Final]{\includegraphics[width=4.5cm]{inudos/ima9.png}}
			\caption{Algoritmo Yamada-Vogel.}
			\label{prueale4} 
		\end{figure}
\label{1sub7}

\chapter{Teoría de trenzas.}
\label{ch2}
\section{Nudos y trenzas.}\label{t2sec1}
En la sección \ref{seccion7} dimos una idea general de lo que se entiende por una trenza. Veamos su definición más formal:\\

\underline{\textbf{Definición:}}\\
Consideremos el cubo $\mathds{D} = \{(x,y,z) / 0 \leq x,y,z \leq 1\}$ y situamos $A_{i}$ puntos en su cara superior y $B_{i}$ puntos en la cara inferior, siendo $i \in \mathds{N}$. Unimos cada punto $A_{i}$ con un cierto punto $B_{k}$, $i \geq k \in \mathds{N}$, mediante arcos simples $d_{i}$ de modo que:
\begin{enumerate}
	\item $ d_{1}, d_{2},...,d_{n} $ sean disjuntos.
	\item Los arcos $ d_{i} $ no pueden conectar puntos $A_{i}$ o $B_{i}$ entre sí.
	\item Al cortar por cualquier plano horizontal, cada arco $ d_{i} $ toca en un sólo punto al plano. 
\end{enumerate}
A cada uno de estos arcos simples $ d_{i} $ les llamaremos cadenas y al conjunto de las n-cadenas se le conoce como \textbf{trenza}.\\

Podemos ver algunos ejemplos de trenzas en la figura \ref{tren1}.\\
\begin{figure}[h!]
	\centering
	\subfigure[$-\sigma2+\sigma1-\sigma2+\sigma1+\sigma3$]{\includegraphics[width=5cm]{itrenzas/t1cubo.png}}
	\space
	\subfigure[$+\sigma1-\sigma2+\sigma1-\sigma2$]{\includegraphics[width=5cm]{itrenzas/t2cubo.png}}
	\caption{Ejemplos de trenzas}
	\label{tren1} 
\end{figure} 

Al igual que hacíamos con los nudos, podremos representar una trenza en el plano visualizando su proyección. En la figura \ref{tren2} se pueden ver las proyección de las trenzas representadas en la figura \ref{tren1}.\\
\begin{figure}[h!]
	\centering
	\subfigure[$-\sigma2+\sigma1-\sigma2+\sigma1+\sigma3$]{\includegraphics[width=3.5cm]{itrenzas/t1pro.png}}
	\space
	\subfigure[$+\sigma1-\sigma2+\sigma1-\sigma2$]{\includegraphics[width=3cm]{itrenzas/t2pro.png}}
	\caption{Proyección de trenzas}
	\label{tren2} 
\end{figure}  

Anteriormente vimos que a cada trenza le corresponde un nudo o un enlace particular. Se obtendrá uniendo los extremos superiores con los extremos inferiores de las cadenas en el mismo orden. A este nudo se le conocerá como \textbf{trenza cerrada}. \\

Denotaremos como $\mathscr{B}_{n}$ al conjunto de todas las trenzas de n cadenas.\\


\bigskip
\begin{center}
	\subsection{Notación de trenzas:}
\end{center}
Para poder trabajar de forma cómoda con las trenzas vamos a darle la siguiente notación:\\
Sean los segmentos que unen las posiciones $i$ con la $i+1$ y las posiciones $i+1$ con la $i$. Al producir un intercambio de posiciones de estos segmentos se producirá un \textbf{cruce}. Este cruce puede realizarse de dos formas: 
\begin{itemize}
	\item El segmento que parte de la posición $i$ cruza por delante al segmento que inicialmente parte en la posición $i+1$. En este caso el cruce se denota como $-\sigma(i)$ y se conoce como un cruce negativo.
	\item  El segmento que parte de la posición $i$ cruza por detrás al segmento que inicialmente parte en la posición $i+1$. En este caso el cruce se denota como $+\sigma(i)$ y se conoce como un cruce positivo.
\end{itemize}
Podemos verlo más claro en la figura \ref{tren4}.\\
\begin{figure}[h!]
	\centering
	\subfigure[$-\sigma(i)$]{\includegraphics[width=3.5cm]{itrenzas/t5.png}}
	\space
	\subfigure[$+\sigma(i)$]{\includegraphics[width=3.4cm]{itrenzas/t6.png}}
	\caption{Signo cruce.}
	\label{tren4} 
\end{figure}

La \textbf{n-trenza trivial} se define como la n-trenza que no realiza ningún cruce. La denotaremos como $1_{n}.$ \\

Cualquier trenza no trivial tendrá de una serie de cruces. En cada plano horizontal podremos tener como mucho un cruce. Notaremos a la trenza con la secuencia de cruces que tenga, empezando por la parte superior de la trenza. A esta secuencia se le conoce como \textbf{palabra} que representa a la trenza. Podemos ver un ejemplo en la figura \ref{tren5}.\\
\begin{figure}[h!]
	\centering
	\includegraphics[width=6cm]{itrenzas/t7.png}
	\caption{Trenza $+\sigma3-\sigma2+\sigma4$.}
	\label{tren5} 
\end{figure}


\bigskip
\begin{center}
	\subsection{Equivalencia de trenzas:}
\end{center}
Intuitivamente diremos que dos trenzas son equivalentes si podemos deformar las cadenas de las trenzas de forma que ambas trenzas se vean iguales. Las trenzas de la figura \ref{tren3} son equivalentes.\\
\begin{figure}[h!]
	\centering
	\subfigure[$-\sigma1+\sigma2-\sigma2$]{\includegraphics[width=3.9cm]{itrenzas/t3.png}}
	\space
	\subfigure[$-\sigma1-\sigma2+\sigma2$]{\includegraphics[width=3.9cm]{itrenzas/t4.png}}
	\caption{Trenzas equivalentes.}
	\label{tren3} 
\end{figure}

\textbf{\underline{Definición:}}\\
Consideremos la cadena $d$ de una trenza situada en el cubo $\mathds{D} = \{(x,y,z) / 0 \leq x,y,z \leq 1\}$. Sea AB un segmento de dicha cadena y C un punto en el cubo de forma que el triángulo $\triangle ABC$ no corta a ninguna otra cadena de la trenza y sólo toca a la cadena $d$ en el segmento AB. Supongamos además que los segmentos AC y CB cortar a cualquier plano horizontal del cubo en un sólo punto como mucho. Podemos quedarnos con una representación poligonal de las cadenas. Visualizamos estas condiciones en la primera imagen de la figura \ref{elem}.\\
Bajo estas condiciones definimos un \textbf{movimiento elemental} como la operación $ \Omega $ que intercambia el segmento AB por los segmentos AC $ \cup $ CB.\\

La operación inversa $ \Omega^{-1} $ que intercambia los segmentos AC $\cup$  CB, que formen parte de una cadena, por el segmento AB de forma que el triángulo $\triangle ABC$ no corte a ninguna otra cadena, también es considerada un movimiento elemental. \\
Podemos ver la representación de ambos movimientos en la figura \ref{elem}.\\
\begin{figure}[h!]
	\centering
	\includegraphics[width=6.5cm]{itrenzas/elemental.png}
	\caption{}
	\label{elem} 
\end{figure}


\textbf{\underline{Definición 2.1:}}\label{defequi}\\
Sean dos trenzas $\beta$, $\beta'$. Diremos que son \textbf{equivalentes} ($\beta \sim \beta'$) si existe una cadena finita de trenzas $ \beta = \beta_{0}$, $\beta_{1},...,\beta_{m}=\beta'$ tal que cada par de trenzas $ \beta_{i}, \beta_{i+1}, i=0,..,m-1, $ está relacionado por un movimiento elemental. A esta cadena de trenzas equivalentes la representaremos del siguiente modo:
\begin{center}
	$ \beta = \beta_{0} \rightarrow \beta_{1} \rightarrow ... \rightarrow \beta_{m}=\beta'$
\end{center}

Si dos trenzas $\beta$, $\beta'$ no son equivalentes, lo denotaremos como $\beta \not \sim \beta'$.\\

Denotaremos como $ \textbf{B}_{n} $ al conjunto de todas las trenzas de n cadenas no equivalentes entre sí. Es decir, ${B}_{n}$ = $\mathscr{B}_{n}$/$ \sim $.\\\label{2sub1}
\section{El grupo de las trenzas.}\label{grupotrenzas}
Hemos visto una definición que nos permite saber cuándo dos trenzas son equivalentes, pero es demasiado general como para poder llevarla a la práctica. En esta sección vamos a hacer un estudio más profundo de la teoría de trenzas. Para ello tenemos que empezar viendo que el conjunto ${B}_{n}$, dotado del producto de trenzas que veremos a continuación, tiene estructura de grupo no abeliano.

\begin{center}
     \subsection{Estructura de grupo no abeliano:}
\end{center}

\textbf{\underline{Definición 2.2:}}\label{defpro}\\
Sean las trenzas $\beta$, $\beta' \in \mathscr{B}_{n}$. Definimos su \textbf{producto} $\beta \beta'$ como la n-trenza que se crea al unir los extremos finales de las cuerdas de $\beta$ con los extremos iniciales de las cuerdas de $\beta'$.\\

En la figura \ref{grupo0} se puede ver un ejemplo del producto de dos trenzas: la trenza (c) es el producto de las trenzas (a) y (b).\\
   \begin{figure}[h!]
   	\centering
   	\subfigure[$\sigma3^{-1}\sigma1$]{\includegraphics[width=5.5cm]{itrenzas/1c1.png}}
   	\subfigure[$\sigma2\sigma3$]{\includegraphics[width=5.5cm]{itrenzas/1c2.png}}
   	\space
   	\subfigure[$\sigma3^{-1}\sigma1\sigma2\sigma3$]{\includegraphics[width=5cm]{itrenzas/1c3.png}}  	
   	\caption{Producto trenzas}
   	\label{grupo0} 
   \end{figure}

\begin{pro}\label{prod1}
	Sean las trenzas $\beta1$, $\beta1'$, $\beta2$, $\beta2' \in \mathscr{B}_{n}$ verificando las equivalencias $\beta1 \sim \beta1'$ y $\beta2 \sim \beta2'$. Entonces se verifica la equivalencia $\beta1\beta2 \sim \beta1'\beta2'$.
	\begin{proof}
		Por hipótesis tenemos $\beta1 \sim \beta1'$ y $\beta2 \sim \beta2'$. Por la definición \ref{defequi} sabemos que existen las secuencias de trenzas equivalentes tales que: 
		\begin{center}
			$ \beta1 = \beta1_{0} \rightarrow \beta1_{1} \rightarrow ... \rightarrow \beta1_{m}=\beta1'$ 
		\end{center}
		\begin{center}
			$ \beta2 = \beta2_{0} \rightarrow \beta2_{1} \rightarrow ... \rightarrow \beta2_{m}=\beta2'$ 
		\end{center}
		Mediante la primera igualdad tenemos 
		\begin{center}
			$ \beta1\beta2 = \beta1_{0}\beta2 \rightarrow \beta1_{1}\beta2 \rightarrow ... \rightarrow \beta1_{m}\beta2=\beta1'\beta2$ luego $ \beta1\beta2 \sim \beta1'\beta2$.
		\end{center}
		Mediante la segunda igualdad tenemos
		\begin{center}
			$ \beta1'\beta2 = \beta1'\beta2_{0} \rightarrow \beta1'\beta2_{1} \rightarrow ... \rightarrow \beta1'\beta2_{m}=\beta1'\beta2'$ 
			luego $ \beta1'\beta2 \sim \beta1'\beta2'$.
		\end{center}
		Por la transitividad de la equivalencia de trenzas se tiene que
\begin{center}
			 $  \beta1\beta2 \sim \beta1'\beta2 \sim \beta1'\beta2'$	
\end{center}	
				
	\end{proof}
\end{pro}

   \begin{figure}[h!]
   	\centering
   	\subfigure[$\sigma3^{-1}\sigma2^{-1}\sigma3^{-1}$]{\includegraphics[width=3.4cm]{itrenzas/3c1.png}}
   	\includegraphics[width=1.2cm]{itrenzas/flechac.png}
   	\subfigure[$\sigma2^{-1}\sigma3^{-1}\sigma2^{-1}$]{\includegraphics[width=3.6cm]{itrenzas/3c2.png}}
   	\subfigure[$\sigma3^{-1}\sigma1\sigma3^{-1}$]{\includegraphics[width=3.6cm]{itrenzas/3c3.png}}
   	\includegraphics[width=1.2cm]{itrenzas/flechac.png}
   	\subfigure[$\sigma1\sigma3^{-1}\sigma3^{-1}$]{\includegraphics[width=3.1cm]{itrenzas/3c4.png}}

	\subfigure[$\sigma3^{-1}\sigma2^{-1}\sigma3^{-1}\sigma3^{-1}\sigma1\sigma3^{-1}$]{\includegraphics[width=4cm]{itrenzas/3c5.png}}
	\includegraphics[width=1.5cm]{itrenzas/flechac.png}
	\subfigure[$\sigma2^{-1}\sigma3^{-1}\sigma2^{-1}\sigma1\sigma3^{-1}\sigma3^{-1}$]{\includegraphics[width=3.2cm]{itrenzas/3c6.png}}   	
   	\caption{Equivalencia producto}
   	\label{grupo1} 
   \end{figure}
   
   Veamos el ejemplo de la figura \ref{grupo1}: Las trenzas (a) y (b) son equivalentes luego la parte superior de las trenzas (e) y (f) es equivalente. Las trenzas (c) y (d) son equivalentes, luego la parte inferior de las trenzas (e) y (f) es equivalente. La unión de estas dos partes que mencionamos en las trenzas (e) y (f) no supone la pérdida de equivalencia. \\

\begin{pro} Producto de trenzas \textbf{asociativo}.\label{prodaso}\\
	Sean las trenzas $\beta1$, $\beta2$, $\beta3 \in \mathscr{B}_{n}$. Se verifica $(\beta1 \beta2) \beta3 \sim \beta1 (\beta2 \beta3)$.

	\begin{proof}	
		
		Por la definición \ref{defpro} sabemos que el producto $(\beta1 \beta2) \beta3$ une los extremos finales $ \beta1 $ con los extremos iniciales de $\beta2 $ y posteriormente une los extremos finales de $\beta2 $ con los extremos iniciales de $ \beta3 $. \\
		
		Por otra parte el producto $\beta1 (\beta2 \beta3)$ une los extremos finales $ \beta2 $ con los extremos iniciales de $\beta3 $ y posteriormente une los extremos finales de $\beta1 $ con los extremos iniciales de $ \beta2 $. \\
		
		En definitiva, es claro que las trenzas $(\beta1 \beta2) \beta3 $ y $ \beta1 (\beta2 \beta3)$ son equivalentes.
	\end{proof}
\end{pro}

En la figura \ref{grupo2} podemos ver que, efectivamente, el producto de las trenzas dadas $\beta1, \beta2 y \beta3$ es asociativo: las trenzas (e) y (g) son iguales.\\

   \begin{figure}[h!]
   	\centering
	\subfigure[$\beta1 =\sigma3^{-1}\sigma1$]{\includegraphics[width=3.5cm]{itrenzas/1c1.png}}
	\subfigure[$\beta2 = \sigma2\sigma3$]{\includegraphics[width=3.5cm]{itrenzas/1c2.png}}  
	\subfigure[$\beta3 = \sigma3$]{\includegraphics[width=2.2cm]{itrenzas/2c1.png}}  	
	
	\subfigure[$\beta1\beta2$]{\includegraphics[width=4cm]{itrenzas/1c3.png}}  	
	\subfigure[$(\beta1\beta2)\beta3$]{\includegraphics[width=3.3cm]{itrenzas/2c2.png}} 
	\subfigure[$\beta2\beta3$]{\includegraphics[width=4.6cm]{itrenzas/2c3.png}}  	
	\subfigure[$\beta1(\beta2\beta3$)]{\includegraphics[width=3.3cm]{itrenzas/2c2.png}} 
   	\caption{Asociatividad producto}
   	\label{grupo2} 
   \end{figure}


\begin{pro} Producto de trenzas \textbf{no conmutativo}.\label{prodnocon}\\
	Sean las trenzas $\beta1$, $\beta2 \in \mathscr{B}_{n}$. No se tiene porqué verificarse la equivalencia $\beta1 \beta2 \sim \beta2 \beta1$.

	\begin{proof}	
		
		Por la definición \ref{defpro} sabemos que:\\
		El producto $\beta1 \beta2$ une los extremos finales $ \beta1 $ con los extremos iniciales de $\beta2 $. \\
		El producto $\beta2 \beta1$ une los extremos finales $ \beta2 $ con los extremos iniciales de $\beta1 $. \\
		
	    Es claro que las trenzas resultantes no tienen porqué ser equivalentes.
	\end{proof}
\end{pro}
Podemos ver un ejemplo de dos trenzas no conmutativas en la figura \ref{grupo3}. Efectivamente las trenzas (c) y (e) no son iguales.\\
   \begin{figure}[h!]
   	\centering
   	\subfigure[$\beta1 = +\sigma3-\sigma2$]{\includegraphics[width=3.5cm]{itrenzas/4c1.png}}
   	\subfigure[$\beta2 = -\sigma3$]{\includegraphics[width=4.5cm]{itrenzas/4c2.png}} 
   	
   	\subfigure[$\beta1\beta2$]{\includegraphics[width=5.2cm]{itrenzas/4c3.png}}
   	\space
   	\subfigure[$\beta2\beta1$]{\includegraphics[width=4.5cm]{itrenzas/4c4.png}} 
   	\includegraphics[width=1.2cm]{itrenzas/flechac.png}
   	\subfigure[$\beta2\beta1$]{\includegraphics[width=5.5cm]{itrenzas/4c5.png}} 
   	\caption{No conmutatividad producto}
   	\label{grupo3} 
   \end{figure}

\begin{pro}  \textbf{Elemento neutro}.\label{prodneutro}\\
	Sea la trenza $\beta \in \mathscr{B}_{n}$. Se verifica:
	\begin{center}
		 $\beta 1_{n} \sim \beta \sim 1_{n} \beta$.
	\end{center}
	
	\begin{proof}	
		Es claro por la definición de n-trenza trivial $ 1_{n} $ (las cadenas de $ 1_{n} $ no tienen cruces luego no afectan a las cadenas de la trenza $ \beta $). 
	\end{proof}
\end{pro}

\begin{pro}  \textbf{Elemento inverso}.\label{prodinverso}\\
	Sea la trenza $\beta \in \mathscr{B}_{n}$. Existirá una trenza $\beta^{-1} \in \mathscr{B}_{n}$ verificando:
	\begin{center}
		$\beta \beta^{-1} \sim 1_{n} \sim \beta^{-1} \beta$.
	\end{center}
	A esta trenza $\beta^{-1}$ se le conoce como trenza inversa.
	
	\begin{proof} 
		Sea la trenza $\beta$ a la que notamos como $\beta = \sigma_{i_{1}}^{\pm 1} \sigma_{i_{2}}^{\pm 1} ... \sigma_{i_{m}}^{\pm 1}$\\

		 Construimos la trenza $\beta'$ a la que notamos como $\beta' = \sigma_{i_{m}}^{\mp 1} ...\sigma_{i_{2}}^{\mp 1} \sigma_{i_{1}}^{\mp 1}$. Veamos que esta trenza es la trenza inversa:
			\begin{center}
			 Por una parte, $\beta \beta'$ = $\sigma_{i_{1}}^{\pm 1} \sigma_{i_{2}}^{\pm 1} ... \sigma_{i_{m}}^{\pm 1} \sigma_{i_{m}}^{\mp 1} ...\sigma_{i_{2}}^{\mp 1} \sigma_{i_{1}}^{\mp 1}$ = $\sigma_{i_{1}}^{\pm 1} \sigma_{i_{2}}^{\pm 1} ... \sigma_{i_{m-1}}^{\pm 1} \sigma_{i_{m-1}}^{\mp 1} ...\sigma_{i_{2}}^{\mp 1} \sigma_{i_{1}}^{\mp 1}$ = $ 1_{n} $. Luego $\beta \beta' \sim 1_{n}$.\\
			\end{center}		 
			\begin{center}
			 Por otra parte, $ 1_{n} = \sigma_{i_{m}}^{\mp 1} \sigma_{i_{m}}^{\pm 1} = \sigma_{i_{m}}^{\mp 1} ...\sigma_{i_{2}}^{\mp 1} \sigma_{i_{2}}^{\pm 1} ... \sigma_{i_{m}}^{\pm 1} = \sigma_{i_{m}}^{\mp 1} ...\sigma_{i_{2}}^{\mp 1} \sigma_{i_{1}}^{\mp 1} \sigma_{i_{1}}^{\pm 1} \sigma_{i_{2}}^{\pm 1} ... \sigma_{i_{m}}^{\pm 1} = \beta' \beta$. Luego $1_{n} \sim \beta \beta'$.\\
			\end{center}
		 Por tanto $\beta' = \beta^{-1}$.\\
		 
		 Nota: estas igualdades son ciertas porque $\sigma_{i}^{\mp 1} \sigma_{i}^{\pm 1} = 1_{2}$ Se puede ver claro en la figura \ref{demo1}.
	      	\begin{figure}[h!]
			\centering
			\includegraphics[width=7cm]{itrenzas/M1.png}
			\caption{Primer movimiento}
			\label{demo1} 
			\end{figure}	     
				
	\end{proof}
\end{pro}

En la figura \ref{grupo4} podemos ver un ejemplo de una trenza (a) y su trenza inversa (b). En la secuencia de trenzas (c)-(f) vemos que efectivamente su producto genera la trenza trivial.\\
   \begin{figure}[h!]
   	\centering
   	\subfigure[$\beta = -\sigma4-\sigma1+\sigma2$]{\includegraphics[width=5cm]{itrenzas/5c1.png}}
   	\subfigure[$\beta^{-1} = -\sigma2+\sigma1+\sigma4$]{\includegraphics[width=4.7cm]{itrenzas/5c2.png}}
   	
   	\subfigure[$\beta\beta^{-1}$]{\includegraphics[width=2cm]{itrenzas/5c3.png}}
   	\includegraphics[width=1.2cm]{itrenzas/flechac.png}
   	\subfigure[$\beta\beta^{-1}$]{\includegraphics[width=2cm]{itrenzas/5c4.png}} 
   	\includegraphics[width=1.2cm]{itrenzas/flechac.png}
   	\subfigure[$\beta\beta^{-1}$]{\includegraphics[width=2cm]{itrenzas/5c5.png}} 
   	\includegraphics[width=1.2cm]{itrenzas/flechac.png}
   	\subfigure[$\beta\beta^{-1}$]{\includegraphics[width=2cm]{itrenzas/5c6.png}} 
   	\caption{Trenzas inversas}
   	\label{grupo4} 
   \end{figure}

\begin{teo}
	El conjunto ${B}_{n}$, dotado del producto de trenzas, es un grupo. El grupo se conoce como el grupo de n-trenzas o el \textbf{grupo de n-trenzas de Artin}. 
	\begin{proof} Sea la trenza $\beta \in \mathscr{B}_{n}$, denotamos su clase de equivalencia como $[\beta] \in {B}_{n}$. Veamos que  ${B}_{n}$ es un grupo: \\
		\begin{enumerate}
			\item 
			Por la proposición \ref{prod1} sabemos que el producto de trenzas está bien definido: Sean $[\beta1],[\beta2] \in {B}_{n}$, se tiene que $[\beta1][\beta2] = [\beta1\beta2] $
			\item 
			Por la proposición \ref{prodaso} sabemos que el producto de trenzas es asociativo.
			\item 
			Por la proposición \ref{prodneutro} sabemos que la n-trenza trivial es el elemento identidad para el producto de trenzas.
			\item 
			Por la proposición \ref{prodinverso} sabemos que el elemento inverso de  $[\beta] \in {B}_{n}$ es $[\beta^{-1}]$, luego  $[\beta]^{-1}$ =  $[\beta^{-1}]$.
		\end{enumerate}
	\end{proof}
\end{teo}

\bigskip
	\subsection{Trenzas equivalentes:}\label{trenzasequi}
	
	Hemos definido $\mathscr{B}_{n}$ como el conjunto de todas las n-trenzas, siendo estas n-trenzas representadas por palabras. En concreto si tenemos la trenza $\beta \in \mathscr{B}_{n}$, que consta de m-cruces, la podremos representar como:
    \begin{center}
    	$\beta = \sigma_{i_{1}}^{\pm 1} \sigma_{i_{2}}^{\pm 1} ... \sigma_{i_{m}}^{\pm 1} $, 1 $\le i_{1}, i_{2},..,i_{m} \le$ n-1.
    \end{center}
	
	Además, hemos visto que el conjunto $B_{n} $ = $\mathscr{B}_{n}$/$ \sim $, de todas las n-trenzas no equivalentes entre sí, tiene estructura de grupo al dotarle del producto de trenzas.\\
	
	En esta sección vamos a ver cuáles son los movimientos que nos permitirán estudiar la equivalencia de dos n-trenzas y posteriormente vamos a reflejar estos movimientos como el conjunto de relaciones que representan al conjunto $B_{n}$.\\
	
	Para analizar cuándo dos trenzas son equivalentes tendremos que ver cuándo las palabras que representan a dichas trenzas son equivalentes. Dos palabras serán equivalentes si y sólo si podemos pasar de una palabra a otra mediante un secuencia de estos tres movimientos:
	\begin{enumerate}
		\item \underline{Primer movimiento - M1:} \\
		Podemos añadir o eliminar $\sigma_{i}\sigma_{i}^{-1}$ o $\sigma_{i}^{-1}\sigma_{i}$ en cualquier palabra. Es claro que la palabra inicial y la palabra final representan a la misma trenza. Podemos verlo en la figura \ref{tren6}.
		\begin{figure}[h!]
			\centering
			\includegraphics[width=5.1cm]{itrenzas/M1.png}
			\caption{Primer movimiento.}
			\label{tren6} 
		\end{figure}
		
		\item \underline{Segundo movimiento - M2:} \\
		Las palabras $\sigma_{i} \sigma_{i+1} \sigma_{i}$ y $\sigma_{i+1} \sigma_{i} \sigma_{i+1}$ son equivalentes (o bien con cruces negativos). Se puede ver en la figura \ref{tren7}.
		\begin{figure}[h!]
			\centering
			\includegraphics[width=8cm]{itrenzas/M2.png}
			\caption{Segundo movimiento.}
			\label{tren7} 
		\end{figure}
		
		
		\item \underline{Tercer movimiento - M3:} \\
		Las palabras $\sigma_{i} \sigma_{j}$ y $\sigma_{j} \sigma_{i}$ son equivalentes si se verifica que $|i-j| > 1$ (o bien con cruces negativos). Se ve claro en la figura \ref{tren8}.
		\begin{figure}[h!]
			\centering
			\includegraphics[width=10cm]{itrenzas/M3.png}
			\caption{Tercer movimiento.}
			\label{tren8} 
		\end{figure}
		
	\end{enumerate}
	En la figura \ref{tren9} podemos ver los movimientos que nos demuestran que dos trenzas dadas más complejas son equivalentes. \\
	Partimos de la trenza $\sigma3^{-1}\sigma1\sigma2^{-1}\sigma3^{-1}$ y aplicamos M3 a los dos primeros cruces obteniendo la trenza $\sigma1\sigma3^{-1}\sigma2^{-1}\sigma3^{-1}$ que vemos en la segunda imagen. \\
	Por último, aplicamos el movimiento M2 a los tres últimos cruces obteniendo la trenza $\sigma1\sigma2^{-1}\sigma3^{-1}\sigma2^{-1}$ que vemos en la última imagen.\\
	\begin{figure}[h!]
		\centering
		\includegraphics[width=10cm]{itrenzas/movi.png}
		\caption{Trenzas equivalentes.}
		\label{tren9} 
	\end{figure}
	
	\begin{teo}	    \label{teoBn}
	    Defino las relaciones:
	    \begin{enumerate}
	    	\item $ \sigma_{i+1}\sigma_{i}\sigma_{i+1} =\sigma_{i}\sigma_{i+1}\sigma_{i} $ siendo $i=2,..,n-2 $ \label{rel1}
	    	\item $ \sigma_{i}\sigma_{j}=\sigma_{j}\sigma_{i} $ siendo $1 \le i < j \le n-1 $, $j-i \geq 2$	 \label{rel2}   	
	    \end{enumerate}
	    El grupo $B_{n}$ tiene la siguiente representación:
        \begin{center}
			$B_{n} = <\sigma1, \sigma2,..,\sigma_{n-1} /$ las relaciones \ref{rel1} y \ref{rel2} se verifican$>$
        \end{center}
	   
	\end{teo}
	
	A partir de estas relaciones de equivalencia base podremos construir nuevas relaciones de equivalencia. Vamos a mostrar algunas relaciones de equivalencia que usaremos posteriormente:
	\begin{enumerate}
		\item $ \sigma_{i+1}\sigma_{i}\sigma_{i+1}^{-1} =\sigma_{i}^{-1}\sigma_{i+1}\sigma_{i} $ siendo $i=2,..,n-2 $		
		\item $ \sigma_{i+1}\sigma_{i}^{-1}\sigma_{i+1}^{-1} =\sigma_{i}^{-1}\sigma_{i+1}^{-1}\sigma_{i} $ siendo $i=2,..,n-2 $
	\end{enumerate}
	
    \underline{	Demostración:}
    \begin{enumerate}
    	\item $ \sigma_{i+1}\sigma_{i}\sigma_{i+1}^{-1} = \sigma_{i}^{-1}\sigma_{i}\sigma_{i+1}\sigma_{i}\sigma_{i+1}^{-1} = \sigma_{i}^{-1}\sigma_{i+1}\sigma_{i}\sigma_{i+1}\sigma_{i+1}^{-1} =  \sigma_{i}^{-1}\sigma_{i+1}\sigma_{i} $ 		
    	\item $ \sigma_{i+1}\sigma_{i}^{-1}\sigma_{i+1}^{-1} = \sigma_{i+1}\sigma_{i}^{-1}\sigma_{i+1}^{-1}\sigma_{i}^{-1}\sigma_{i}=\sigma_{i+1}\sigma_{i+1}^{-1}\sigma_{i}^{-1}\sigma_{i+1}^{-1}\sigma_{i} = \sigma_{i}^{-1}\sigma_{i+1}^{-1}\sigma_{i}$
    \end{enumerate}
	

	En la figura \ref{tren14} podemos visualizar las equivalencias.\\
	\begin{figure}[h!]
		\centering
		\includegraphics[width=3cm]{itrenzas/6c1.png}
		\includegraphics[width=0.7cm]{itrenzas/flechac.png}
		\includegraphics[width=2.6cm]{itrenzas/6c2.png}
		\space	
		\includegraphics[width=2.7cm]{itrenzas/6c3.png}
		\includegraphics[width=0.7cm]{itrenzas/flechac.png}
		\includegraphics[width=2.9cm]{itrenzas/6c4.png}
		\caption{Trenzas equivalentes.}
		\label{tren14} 
	\end{figure}


\newpage
\begin{center}
	\subsection{Trenzas Markov-equivalentes:}
\end{center}
Es claro que si tenemos dos trenzas que son equivalentes, sus trenzas cerradas nos darán nudos que serán equivalentes. Pero...¿puede darse el caso de tener dos trenzas no sean equivalentes y que sus trenzas cerradas sí lo sean? Veamos, mediante el ejemplo de la figura \ref{tren10} que sí es posible: las trenzas de partida no son equivalentes (tenemos distinto número de cadenas), sin embargo sus trenzas cerradas sí lo son. Lo demostraremos en la figura \ref{tren13}.\\
\begin{figure}[h!]
	\centering
	\includegraphics[width=3cm]{itrenzas/t1pro.png}
	\includegraphics[width=2.5cm]{itrenzas/t2pro.png}
	
	\includegraphics[width=3.5cm]{itrenzas/t1probien.png}
	\includegraphics[width=3.5cm]{itrenzas/t2probien.png}
	\caption{Trenzas Markov-equivalentes.}
	\label{tren10} 
\end{figure}

\underline{\textit{Definición:}}\\
Se dice que dos trenzas son \textbf{Markov-equivalentes} si sus cierres producen el mismo enlace. En este caso tendremos que los nudos representados por las trenzas son equivalentes. \\

\begin{teo} \textbf{Teorema de Markov.} \label{teoMarkov}\\
	Dos trenzas son Markov-equivalentes si y solo si podemos pasar de una trenza a otra mediante una secuencia de las tres operaciones que hemos visto en la subsección \ref{trenzasequi} y los movimientos de Markov.
\end{teo}

Veamos los movimientos de Markov:
\begin{enumerate}
	\item \underline{Conjugación - Mv1:} \\
	Las palabras $\beta$ y $\sigma_{i} \beta \sigma_{i}^{-1}$ (o bien  $\sigma_{i}^{-1} \beta \sigma_{i}$) generan trenzas cerradas equivalentes. Se puede ver en la figura \ref{tren11}.
	\begin{figure}[h!]
		\centering
		\includegraphics[width=8cm]{itrenzas/M4.png}
		\caption{Conjugación Markov.}
		\label{tren11} 
	\end{figure}
	
	\item \underline{Estabilización - Mv2:} \\
	Esta operación nos permite modificar el número de cadenas de las trenzas. 
	Sea la palabra $\beta$ de n cadenas. Esta palabra genera una trenza cerrada equivalente a $\beta \sigma_{n}$ o bien $\sigma_{n} \beta $. Se puede ver en la figura \ref{tren12}.
	\begin{figure}[h!]
		\centering
		\includegraphics[width=7cm]{itrenzas/M5.png}
		\caption{Estabilización Markov.}
		\label{tren12} 
	\end{figure}
	
\end{enumerate}

Podemos ver un ejemplo de dos trenzas Markov-equivalentes en la figura \ref{tren13}.
\begin{figure}[h!]
	\centering
	\subfigure[$\sigma2^{-1}\sigma1\sigma2^{-1}\sigma1\sigma3$]{\includegraphics[width=3.5cm]{itrenzas/t1probien.png}}
	\subfigure[Mv2]{\includegraphics[width=1cm]{itrenzas/flechac.png}}
	\subfigure[$\sigma2^{-1}\sigma1\sigma2^{-1}\sigma1$]{\includegraphics[width=3.5cm]{itrenzas/t3probien.png}}
	\subfigure[Mv1]{\includegraphics[width=1cm]{itrenzas/flechac.png}}
	\subfigure[$\sigma1\sigma2^{-1}\sigma1$ $\sigma2^{-1}\sigma1\sigma1^{-1}$]{\includegraphics[width=2.4cm]{itrenzas/t4probien.png}}
	\subfigure[M1]{\includegraphics[width=1cm]{itrenzas/flechac.png}}
	\subfigure[$\sigma1\sigma2^{-1}\sigma1\sigma2^{-1}$]{\includegraphics[width=3.5cm]{itrenzas/t2probien.png}}
	\caption{Trenzas Markov-equivalentes.}
	\label{tren13} 
\end{figure}

\bigskip
\section{Algunos invariantes.}
Ya conocemos cuáles son los movimientos básicos que nos permiten determinar si dos trenzas dadas son equivalentes. Pero al igual que ocurría con los nudos, llevar esta idea a la práctica no es factible.\\

En esta sección vamos a ver algunos invariantes de trenzas que nos permitirán determinar de forma relativamente rápida si dos trenzas no son equivalentes.\\

\underline{\textbf{Definición:}} \\
Un \textbf{invariante} de una trenza es una propiedad que no cambia cuando la trenza sufre deformaciones en el espacio. \\

Vamos a ver algunos invariantes de trenzas que usaremos posteriormente en la práctica.

\bigskip
\subsection{Exponente:}\label{invtren1}
\textbf{\underline{Definición:}}\\
Sea $\beta \in B_{n}$ representada como $\beta $= $\sigma_{i_{1}}^{e_{1}} \sigma_{i_{2}}^{e_{2}} ... \sigma_{i_{m}}^{e_{m}}$ donde $e_{i} \in \{-1,1\}$, 1 $\le i_{1}, i_{2},..,i_{m} \le$ n-1.\\
Llamamos \textbf{exponente} de $\beta$ al entero $ \sum_{i=1}^{m} e_{i} $. Se denotará como exp($\beta$).\\

El exponente de la trenza $\beta = \sigma3\sigma2^{-1}\sigma3^{-1}$ de la figura \ref{exp1} es $+1-1-1=-1$.\\
   \begin{figure}[h!]
   	\centering
   	\includegraphics[width=4.3cm]{itrenzas/4c3.png}
   	\caption{$\sigma3\sigma2^{-1}\sigma3^{-1}$}
   	\label{exp1} 
   \end{figure}

\begin{pro}
	El exponente de una trenza $\beta \in B_{n}$ es un invariante. 
	\begin{proof}
		Veamos que dadas las trenzas $\beta1,\beta2 \in B_{n}$ tales que $\beta1 \sim \beta2$, se verifica que exp($\beta1$)=exp($\beta2$).\\
		
		Al ser $\beta1 \sim \beta2$, por la definición \ref{defequi}, sabemos que existe la secuencia de trenzas equivalentes tal que: 
		\begin{center}
			$ \beta1 = \beta1_{0} \rightarrow \beta1_{1} \rightarrow ... \rightarrow \beta1_{m}=\beta2$ 
		\end{center}
	    donde cada par de trenzas $ \beta_{j}, \beta_{j+1}, j=0,..,m-1, $ está relacionado por un movimiento elemental. Estos movimientos elementales se pueden reducir a las relaciones las  \ref{rel1} y \ref{rel2} que definen al grupo $ B_{n} $ . Por tanto,para verificar que el exponente de cada par $ \beta_{j}, \beta_{j+1}$ es igual, tenemos que ver que estas relaciones de igualdad tienen el mismo exponente:
	    
	    \begin{enumerate}
	    \item $\sigma_{i+1}\sigma_{i}\sigma_{i+1} =\sigma_{i}\sigma_{i+1}\sigma_{i} $ siendo $i=2,..,n-2 $:\\
	    exp($ \sigma_{i+1}\sigma_{i}\sigma_{i+1})$ = exp$(\sigma_{i}\sigma_{i+1}\sigma_{i} $).
	    \item $\sigma_{i}\sigma_{j}=\sigma_{j}\sigma_{i}$ siendo $1 \le i < j \le n-1 $, $j-i \geq 2$:\\
	    exp($\sigma_{i}\sigma_{j})$ = exp$(\sigma_{j}\sigma_{i}$).  	
	    \end{enumerate}
	\end{proof}
\end{pro}

Haciendo uso de este invariante podremos ver muy fácilmente si dos trenzas dadas tienen distintos exponente y por tanto no son equivalentes. Podemos ver un ejemplo de dos trenzas no equivalentes en la figura \ref{exp2}. La trenza $\beta1 = \sigma2\sigma1\sigma2^{-1}$ tiene exponente +1, mientras que la trenza $\beta1 = \sigma2^{-1}\sigma1^{-1}\sigma2$ tiene exponente -1.\\ 
	\begin{figure}[h!]
		\centering
		\subfigure[exp($ \beta1 $) = +1]{\includegraphics[width=3.5cm]{itrenzas/6c1.png}}
		\subfigure[exp($ \beta2 $) = -1]{\includegraphics[width=3.5cm]{itrenzas/6c4.png}}
		\caption{Trenzas no equivalentes.}
		\label{exp2} 
	\end{figure}
	
Si tuviesen igual exponente tendríamos que estudiar otros invariantes para ver si las trenzas son iguales. Por ejemplo, en la figura \ref{exp3} vemos que las trenzas $\beta1 = \sigma2\sigma2\sigma3^{-1}$ y $\beta2 = \sigma3^{-1}\sigma2\sigma1$ tienen el mismo exponente (+1) luego no podremos saber si son o no equivalentes. Al estudiar el invariante de la seguiente sección veremos que no lo son.  \\
	\begin{figure}[h!]
		\centering
		\subfigure[exp($ \beta1 $) = +1]{\includegraphics[width=3.5cm]{itrenzas/7c1.png}}
		\subfigure[exp($ \beta2 $) = +1]{\includegraphics[width=3.5cm]{itrenzas/7c2.png}}
		\caption{Trenzas con mismo exponente.}
		\label{exp3} 
	\end{figure}

\bigskip
\subsection{Permutación:}\label{invtren2}
\textbf{\underline{Definición:}}\\
Sea $\beta \in B_{n}$. Consideramos la cadena i que conecta el punto inicial $ A_{i}$ con el punto final $B_{j_{i}}$. Llamaremos \textbf{permutación} asociada a la trenza a la matriz:\\
\[\pi(\beta)=\begin{bmatrix}
1 & 2 & .. & n\\
j_{1} & j_{2} & .. & j_{n} \\
\end{bmatrix}\]
Por simplicidad podremos denotar la permutación de la trenza como el vector $j_{1} j_{2} ... j_{n}$.\\

La permutación de la trenza $\beta = \sigma3\sigma2^{-1}\sigma3^{-1}$ de la figura \ref{perm1} es \[\pi(\beta)=\begin{bmatrix}
1 & 2 & 3 & 4\\
1 & 4 & 3 & 2 \\
\end{bmatrix}\] o bien directamente la permutación: 1 4 3 2.\\
\begin{figure}[h!]
	\centering
	\includegraphics[width=5.5cm]{itrenzas/4c3.png}
	\caption{$\sigma3\sigma2^{-1}\sigma3^{-1}$}
	\label{perm1} 
\end{figure}

\begin{pro}
    La permutación de una trenza $\beta \in B_{n}$ es un invariante. 
    \begin{proof}
    	Veamos que dadas las trenzas $\beta1,\beta2 \in B_{n}$ tales que $\beta1 \sim \beta2$, se verifica que $\pi(\beta1$)=$\pi(\beta2$).\\
    	
    	Por la definición \ref{defequi}, existirá la secuencia de trenzas equivalentes tal que: 
    	\begin{center}
    		$ \beta1 = \beta1_{0} \rightarrow \beta1_{1} \rightarrow ... \rightarrow \beta1_{m}=\beta2$ 
    	\end{center}
    	donde cada par de trenzas $ \beta_{j}, \beta_{j+1}, j=0,..,m-1, $ está relacionado por un movimiento elemental. Los movimientos elementales, por definición, no permiten intercambiar los extremos a los que están sujetas las cuerdas de las trenzas. Por este motivo las permutaciones tienen que ser iguales. \\
    \end{proof}
\end{pro}

Haciendo uso de este invariante podemos ver un ejemplo de dos trenzas que no son equivalentes en la figura \ref{perm2}: la trenza $\beta1 = \sigma2\sigma2\sigma3^{-1}$ tiene permutación 1 2 4 3 mientras que la trenza $\beta2 = \sigma3^{-1}\sigma2\sigma1$ tiene permutación 2 3 4 1.\\ 
	\begin{figure}[h!]
		\centering
		\subfigure[$\pi( \beta1 $) = 1 2 4 3]{\includegraphics[width=4.4cm]{itrenzas/7c1.png}}
		\subfigure[$\pi( \beta2 $) = 2 3 4 1]{\includegraphics[width=4.8cm]{itrenzas/7c2.png}}
		\caption{Trenzas no equivalentes.}
		\label{perm2} 
	\end{figure}


\bigskip
\subsection{Polinomio de Alexander:}\label{invtren3}
En la sección \ref{alenudos} estudiamos el polinomio de Alexander como un invariante para nudos. Por el teorema \ref{teoMarkov}, podremos hacer uso de las trenzas como una base para obtener los invariantes de nudos, en concreto, para estudiar este invariante que implementaremos posteriormente. \\

Para poder obtener el polinomio de Alexander de una trenza, tendremos que ver unos conceptos previos, \cite{3}:

\textbf{\underline{Definición:}}\\
Sea la trenza $\beta $= $\sigma_{i_{1}}^{e_{1}} \sigma_{i_{2}}^{e_{2}} ... \sigma_{i_{m}}^{e_{m}}$ donde $e_{i} \in \{-1,1\}$, 1 $\le i_{1}, i_{2},..,i_{m} \le$ n-1. Podemos definir el homomorfismo
\begin{center}
	 $ \phi_{n} : B_{n}  \rightarrow  M(n,\mathds{Z}[t,t^{-1}])$	 
\end{center}
\[ \phi_{n} ( \sigma_{i}) = \begin{bmatrix}
I_{i-1} &  &  & \\
 & 1-t & t &  \\
 & 1 & 0 &  \\
 &  &  & I_{n-i-1} \\
\end{bmatrix}\]
donde $ I_{i} $ representa a la matriz $ ixi $ identidad y los espacios en blanco representan matrices nulas. LLamaremos a esta representación como \textbf{representación de Burau}.\\

En consecuencia tenemos 
	\[ \phi_{n} ( \sigma_{i}^{-1}) = \phi_{n} ( \sigma_{i})^{-1}= \begin{bmatrix}
	I_{i-1} &  &  & \\
	& 0 & 1 &  \\
	& t^{-1} & 1-t^{-1} &  \\	
	&  &  & I_{n-i-1} \\
	\end{bmatrix}\]


\bigskip
Veamos la matriz de Burau de la trenza $\beta \in B_{4}$ representada como $\beta = \sigma2\sigma3^{-1}$.\\
Matriz Burau de $ \sigma2 $:
	\[ \phi_{4} ( \sigma2) = \begin{bmatrix}
	1 & 0 & 0 & 0 \\
	0 & 1-t & t & 0  \\
	0 & 1 & 0 & 0 \\
	0 & 0 & 0 & 1 \\
	\end{bmatrix}\]

Matriz Burau de $ \sigma3^{-1} $:
\[ \phi_{4} ( \sigma3^{-1}) = \begin{bmatrix}
	1 & 0 & 0 & 0 \\
	0 & 1 & 0 & 0 \\
	0 & 0 & 0 & 1  \\	
	0 & 0 & t^{-1} & 1-t^{-1} \\
\end{bmatrix}\]
 
 Por tanto, la matriz Burau de $\beta = \sigma2\sigma3^{-1}$ es:
 \[ \phi_{4} (\sigma2\sigma3^{-1}) = \phi_{4} (\sigma2) \phi_{4}(\sigma3^{-1}) = \begin{bmatrix}
 1 & 0 & 0 & 0 \\
 0 & 1-t & 0 & t \\
 0 & 1 & 0 & 0  \\	
 0 & 0 & t^{-1} & 1-t^{-1} \\
 \end{bmatrix}\]\\
 
 
\begin{teo}\label{teoalex}
	Sea la trenza $\beta \in B_{n}$ con representación de Burau $\phi_{n}(\beta)$. Supongamos que su trenza cerrada genera al nudo K.\\
    Entonces $\exists k \in \mathds{Z}$ tal que el polinomio ($ \pm $ $ t^{k} )det[\phi_{n}(\beta)$ - $ I_{n} $$ ]_{1,1} $ es un invariante de K. A este polinomio, conocido como polinomio de Alexander, se le denota como $ \triangle_{k} $(t).
\end{teo}

Nota: Sea la matriz cuadrada A de dimensión n y A' la matriz de dimensión n-1 que se obtiene al suprimir en la matriz A la fila i y la columna i. 
Se tiene $ det[A]_{i,i}=det[A'] $.\\ 

Gracias a este teorema podemos afirmar que si dos trenzas $\beta1 \in B_{n}, \beta2 \in B_{m}$ cerradas generan el mismo nudo entonces se verificará la igualdad:\\
det[$\phi_{n}(\beta1)$ - $ I_{n} $$ ]_{1,1}$ = $ \pm t^{k} $det[$\phi_{m}(\beta2)$ - $ I_{m} $$ ]_{1,1}$.\\

Para ver la demostración necesitamos un lema previo y los siguientes conceptos:\\

\newpage
Sea $\phi_{n}(\beta)$ = $M$ = $||a_{ij}||$ $1 \le i,j \le n$, donde $\beta \in B_{n}$. Definimos las siguientes matrices de dimensión nxn:\\
  \[ S = \begin{bmatrix}
  1 & 1 & 1 &... & 1 & 1 \\
  0 & 1 & 1 &... & 1 & 1\\
  0 & 0 & 1 &... & 1 & 1 \\
  ... & ... & ...& ... & 1 & 1 \\	
  0 & 0 & 0 & ... & 1 & 1\\	
  0 & 0 & 0 & ... & 0 & 1\\
  \end{bmatrix} 
  S^{-1} = \begin{bmatrix}
  1 & -1 & 0 & ... & 0 & 0 \\
  0 & 1 & -1 & ... & 0 & 0\\
  0 & 0 & 1 & ... & 0  & 0\\
  ... & ... & ...& ... & 0 & 0  \\	
  0 & 0 & 0 &...& 1 & -1  \\	
  0 & 0 & 0 & ... & 0 & 1 \\
  \end{bmatrix}\]\\
  
 Podemos considerar el producto de matrices $S^{-1}MS$
   \[ M S = \begin{bmatrix}
   a_{1,1} & a_{1,2} & a_{1,3} &... & a_{1,n} \\
   a_{2,1} & a_{2,2} & a_{2,3} &...  & a_{2,n}\\
   ... & ... & ...& ... & ... \\		
   a_{n,1} & a_{n,2} & a_{n,3} & ... & a_{n,n}\\
   \end{bmatrix} 
   \begin{bmatrix}
   1 & 1 & 1 &... & 1  \\
   0 & 1 & 1 &... & 1 \\
   ... & ... & ...& ... & ... \\	
   0 & 0 & 0 & ... & 1\\
   \end{bmatrix}\]
   
   \[= \begin{bmatrix}
   a_{1,1} & a_{1,1}+a_{1,2} & ... & a_{1,1} + a_{1,2}+...+a_{1,n} = 1  \\
   a_{2,1} & a_{2,1}+a_{2,2} &... & a_{2,1} + a_{2,2}+...+a_{2,n} = 1  \\
   ... & ... & ... & ... \\	
   a_{n,1} & a_{n,1}+a_{n,2} & ... & a_{n,1} + a_{n,2}+...+a_{n,n} = 1 \\
   \end{bmatrix}\]\\
  
  Luego la matriz  $S^{-1}MS$ tendrá la siguiente forma:\\
     \[ S^{-1}MS = \begin{bmatrix}
     1 & -1 & 0 &... & 0 \\
     0 & 1 & -1 &...  & 0\\
     ... & ... & ...& ... & ... \\		
     0 & 0 & 0 & ... & 1\\
     \end{bmatrix} 
     \begin{bmatrix}
     a_{1,1} & a_{1,1}+a_{1,2} & ... & a_{1,1} + a_{1,2}+...+a_{1,n} = 1  \\
     a_{2,1} & a_{2,1}+a_{2,2} &... & a_{2,1} + a_{2,2}+...+a_{2,n} = 1  \\
     ... & ... & ... & ... \\	
     a_{n,1} & a_{n,1}+a_{n,2} & ... & a_{n,1} + a_{n,2}+...+a_{n,n} = 1 \\
     \end{bmatrix}\]
     
     \[= \begin{bmatrix}
     a_{1,1}-a_{2,1} & a_{1,1}+a_{1,2}-a_{2,1}-a_{2,2} & ... & 1-1 = 0  \\
     a_{2,1}-a_{3,1} & a_{2,1}+a_{2,2}-a_{3,1}-a_{3,2}&... & 1 - 1=0  \\
     ... & ... & ... & ... \\	
     a_{n,1} & a_{n,1}+a_{n,2} & ... & 1 \\
     \end{bmatrix}\]\\
     
     De un modo más simple, vamos a denotar a dicha matriz del siguiente modo:\\
     \[ S^{-1}MS = \left[\begin{array}{r|r}
     \Lambda (t) & 0 \\ \hline	
     * ... * & 1\\
     \end{array}\right]\]\\
  Nota:  Es claro que $ det[S^{-1}(M-I_{n})S]_{n,n} = det(\Lambda(t)-I_{n-1}) $.\\
  
  \newpage
 \begin{lem}
 	Se verifica la siguiente igualdad:
	 \begin{center}
 		det $(\Lambda(t) - I_{n-1})$ = (1+t+..+$ t^{n-1} $) det$[ M - I_{n}] _{1,1} $
	 \end{center}
	 \begin{proof}
	 	Consideramos la matriz de dimensión (n-1)x(n-1):
	            \begin{center}
	            	$ \Lambda(t) - I_{n-1}= $ 
	            \end{center}	           
	           \[\begin{bmatrix}
	           a_{1,1}-a_{2,1}-1 & a_{1,1}+a_{1,2}-a_{2,1}-a_{2,2} & ... &  a_{1,1}+...+a_{1,n-1}-a_{2,1}-...-a_{2,n-1} \\
	           a_{2,1}-a_{3,1} & a_{2,1}+a_{2,2}-a_{3,1}-a_{3,2}-1&... & a_{2,1}+...+a_{2,n-1}-a_{3,1}-...-a_{3,n-1}  \\
	           ... & ... & ... & ... \\	
	           a_{n-1,1}-a_{n,1} & a_{n-1,1}+a_{n-1,2}-a_{n,1}-a_{n,2} & ... & a_{n-1,1}+...+a_{n-1,n-1}-a_{n,1}-...-a_{n,n-1} \\
	           \end{bmatrix}\]\\
	           
	           Es claro que det $(\Lambda(t) - I_{n-1})$ = det$(S^{T}(\Lambda(t) - I_{n-1})S^{-1})$ donde $S^{T}$ es la matriz traspuesta de dimensión (n-1)x(n-1) de la matriz $S$. Por tanto, vamos a trabajar con la matriz $(S^{T}(\Lambda(t) - I_{n-1})S^{-1})$:
	           
	           \[ (\Lambda(t) - I_{n-1})S^{-1} = (\Lambda(t) - I_{n-1}) \begin{bmatrix}
	           1 & -1 & 0 &... & 0 \\
	           0 & 1 & -1 &...  & 0\\
	           ... & ... & ...& ... & ... \\		
	           0 & 0 & 0 & ... & 1\\
	           \end{bmatrix}=\]
	                
	           \[= \begin{bmatrix}
	           a_{1,1}-a_{2,1}-1 & a_{1,2}-a_{2,2}+1 & ... & a_{1,n-1}-a_{2,n-1}  \\
	           a_{2,1}-a_{3,1} & a_{2,2}-a_{3,2}-1&... &  a_{2,n-1}-a_{3,n-1} \\
	           ... & ... & ... & ... \\	
	           a_{n-1,1}-a_{n,1} & a_{n-1,2}-a_{n,2}&... &  a_{n-1,n-1}-a_{n,n-1} \\
	           \end{bmatrix}\]\\
	           
	           Luego:	          
	           \begin{center}
	            	$  S^{T}(\Lambda(t) - I_{n-1})S^{-1} =$
	           \end{center}
	           \[ = \begin{bmatrix}
	           1 & 0 & 0 &... & 0 \\
	           1 & 1 & 0 &...  & 0\\
	           ... & ... & ...& ... & ... \\		
	           1 & 1 & 1 & ... & 1\\
	           \end{bmatrix}
	           \begin{bmatrix}
	           a_{1,1}-a_{2,1}-1 & a_{1,2}-a_{2,2}+1 & ... & a_{1,n-1}-a_{2,n-1}  \\
	           a_{2,1}-a_{3,1} & a_{2,2}-a_{3,2}-1&... &  a_{2,n-1}-a_{3,n-1} \\
	           ... & ... & ... & ... \\	
	           a_{n-1,1}-a_{n,1} & a_{n-1,2}-a_{n,2}&... &  a_{n-1,n-1}-a_{n,n-1} \\
	           \end{bmatrix}=\]\\
	           
	           \[ = \begin{bmatrix}
	           a_{1,1}-a_{2,1}-1 & a_{1,2}-a_{2,2}+1 & ... & a_{1,n-1}-a_{2,n-1}  \\
	           a_{1,1}-a_{3,1}-1 & a_{1,2}-a_{3,2}&... &  a_{1,n-1}-a_{3,n-1} \\
	           ... & ... & ... & ... \\	
	           a_{1,1}-a_{n,1}-1 & a_{1,2}-a_{n,2}&... &  a_{1,n-1}-a_{n,n-1} \\
	           \end{bmatrix} =
	           \begin{bmatrix}
	           A_{1} - A_{2} \\
	           A_{1} - A_{3} \\
	           ... \\	
	           A_{1} - A_{n} \\
	           \end{bmatrix}\]\\
	           
	           Donde los vectores $A_{i} $ $ 1 \le i \le n $ son las filas de la matriz $[M-I_{n}]_{0,n}$, es decir, tienen la siguiente forma:
			   \begin{center}
		           $ A_{1} = [a_{1,1}-1, a_{1,2},...,a_{1,n-1}] $\\
	               $ A_{2} = [a_{2,1}, a_{2,2}-1,...,a_{2,n-1}] $\\
	               ...\\
	               $ A_{n} = [a_{n,1}, a_{n,2},...,a_{n,n-1}] $\\
		       \end{center}
	           
	           Por tanto det $(\Lambda(t) - I_{n-1})$ = det$(S^{T}(\Lambda(t) - I_{n-1})S^{-1})$ = 
	           \[=det\begin{bmatrix}
	           A_{1} - A_{2} \\
	           A_{1} - A_{3} \\
	           ... \\	
	           A_{1} - A_{n} \\
	           \end{bmatrix}=\]
	           \[=det\begin{bmatrix}
	           A_{1} \\
	           A_{1} - A_{3} \\
	           ... \\	
	           A_{1} - A_{n} \\
	           \end{bmatrix} + det\begin{bmatrix}
	           - A_{2} \\
	           A_{1} - A_{3} \\
	           ... \\	
	           A_{1} - A_{n} \\
	           \end{bmatrix} =det\begin{bmatrix}
	           A_{1} \\
	           - A_{3} \\
	           ... \\	
	           - A_{n} \\
	           \end{bmatrix} + det\begin{bmatrix}
	           - A_{2} \\
	           A_{1} - A_{3} \\
	           ... \\	
	           A_{1} - A_{n} \\
	           \end{bmatrix}=...=\]
	           
	           \[=det\begin{bmatrix}
	           A_{1}  \\
	           - A_{3} \\
	           ... \\	
	           - A_{n} \\
	           \end{bmatrix}+det\begin{bmatrix}
	           -A_{2}  \\
	           A_{1} \\
	           ... \\	
	           - A_{n} \\
	           \end{bmatrix}+...+det\begin{bmatrix}
	           -A_{2}  \\
	           -A_{3} \\
	           ... \\
	           -A_{k}\\
	           A_{1}\\
	           -A_{k+2}
	           ...\\	
	           - A_{n} \\
	           \end{bmatrix}+...+det\begin{bmatrix}
	           -A_{2}  \\
	           - A_{3} \\
	           ... \\	
	           - A_{n-1} \\
	           A_{1}\\
	           \end{bmatrix}+det\begin{bmatrix}
	           - A_{2}  \\
	           - A_{3} \\
	           ... \\	
	           - A_{n} \\
	           \end{bmatrix}\]
	           
	           Para obtener el valor de estos determinantes vamos a hacer uso de la siguiente igualdad:
	           \begin{center}
	           	$ det[M-I_{n}]_{p,q} = (-1)^{p-q}t^{p-1}det[M-I_{n}]_{1,1}	 $ $1\le p,q \le n$  
	           \end{center}    
	           
	           De modo que tomando $ p=k+1 $ y $ q=n $, se tiene:     
	           
	           \[det\begin{bmatrix}
	           -A_{2}  \\
	           -A_{3} \\
	           ... \\
	           -A_{k}\\
	           A_{1}\\
	           -A_{k+2}
	           ...\\	
	           - A_{n} \\
	           \end{bmatrix}=(-1)^{n-k-1}det[M-I_{n}]_{k+1,n} = t^{k}det[M-I_{n}]_{1,1}\]
	           
	           Por tanto, det $(\Lambda(t) - I_{n-1})$ = (1+t+..+$ t^{n-1} $) det$[ M - I_{n}] _{1,1} $.
	 \end{proof}
 \end{lem}
 

 \begin{cor}\label{corlem}
 		Se verifica la siguiente igualdad:
 		\begin{center}
 			$ det[S^{-1}(M-I_{n})S]_{n,n} = det(\Lambda(t)-I_{n-1}) $ = (1+t+..+$ t^{n-1} $) det$[ M - I_{n}] _{1,1} $
 		\end{center}
 \end{cor}
 
 \bigskip
 Ya estamos en condiciones de demostrar el teorema \ref{teoalex}:
 \begin{proof}
 	Por el teorema \ref{teoMarkov} sabemos que es suficiente con probar las siguientes igualdades, donde $ \gamma, \beta \in B_{n} $ :
 	\begin{itemize}
 		\item Movimiento elemental M1: \\
 		$ det[\phi_{n}(+\sigma_{i}-\sigma_{i}) - I_{n}]_{1,1} = det[\phi_{n}(-\sigma_{i}+\sigma_{i}) - I_{n}]_{1,1}$, siendo $i<n$
 		\item Movimiento elemental M2: \\
 		$ det[\phi_{n}(\sigma_{i}\sigma_{i+1}\sigma_{i}) - I_{n}]_{1,1} = det[\phi_{n}(\sigma_{i+1}\sigma_{i}\sigma_{i+1}) - I_{n}]_{1,1}$, siendo $ i+1<n $.
 		\item Movimiento elemental M3:\\
 		$ det[\phi_{n}(\sigma_{i}\sigma_{j}) - I_{n}]_{1,1} = det[\phi_{n}(\sigma_{j}\sigma_{i}) - I_{n}]_{1,1}$, siendo $i<n, |i-j| > 1$
 		\item Verificando conjugación Mv1: $ det[\phi_{n}(\gamma\beta\gamma^{-1}) - I_{n}]_{1,1} = det[\phi_{n}(\beta) - I_{n}]_{1,1}$ 
 		\item Verificando estabilización Mv2: $ det[\phi_{n+1}(\beta\sigma_{n}) - I_{n+1}]_{1,1} = det[\phi_{n}(\beta) - I_{n}]_{1,1}$   
  	\end{itemize}
	 Para demostrar las igualdades referentes a los movimientos elementales basta con probar las siguientes igualdades:
	 \begin{itemize}
	 	 \item Para el movimiento elemental M1: \\
	 	 $ \phi_{n}(+\sigma_{i}-\sigma_{i}) = \phi_{n}(-\sigma_{i}+\sigma_{i})$, siendo $i<n$.\\
	 	 Esta igualdad es clara pues por definición $-\sigma_{i}$ es la matriz inversa de $\sigma_{i}$  luego $ \phi_{n}(+\sigma_{i}-\sigma_{i}) = \phi_{n}(+\sigma_{i})\phi_n(-\sigma_{i}) = \phi_{n}(-\sigma_{i})\phi_n(+\sigma_{i}) = \phi_{n}(-\sigma_{i}+\sigma_{i})$
	 	 
	 	 \item Para el movimiento elemental M2: \\
	  	 $ \phi_{n}(\sigma_{i}\sigma_{i+1}\sigma_{i}) = \phi_{n}(\sigma_{i+1}\sigma_{i}\sigma_{i+1})$, siendo $ i+1<n $.\\
	  	 Sin pérdida de generalidad podemos suponer $i=1, n=3$ de modo que:
	  	 	\[ \phi_{3}(\sigma_{1}\sigma_{2}\sigma_{1}) = 
	  	 	\begin{bmatrix}
	  	 	  1-t & t & 0  \\
	  	 	  1 & 0 & 0 \\
	  	 	  0 & 0 & 1 \\
	  	 	\end{bmatrix}\begin{bmatrix}
	  	 	1 & 0 & 0 \\
	  	 	0 & 1-t & t \\
	  	 	0 & 1 & 0 \\
	  	 	\end{bmatrix}\begin{bmatrix}
	  	 	1-t & t & 0  \\
	  	 	1 & 0 & 0 \\
	  	 	0 & 0 & 1 \\
	  	 	\end{bmatrix}=\]
	  	 	
	  	 	\[ = 
	  	 	\begin{bmatrix}
	  	 	1-t & t & 0  \\
	  	 	1 & 0 & 0 \\
	  	 	0 & 0 & 1 \\
	  	 	\end{bmatrix}\begin{bmatrix}
	  	 	1-t & t & 0  \\
	  	 	1-t & 0 & t \\
	  	 	1 & 0 & 0 \\
	  	 	\end{bmatrix}=
	  	 	\begin{bmatrix}
	  	 	(1-t)^{2}+(1-t)t = 1-t & (1-t)t & t^{2}  \\
	  	 	1-t & t & 0 \\
	  	 	1 & 0 & 0 \\
	  	 	\end{bmatrix}=\]
	  	 	
	  	 	\[\begin{bmatrix}
	  	 	1-t& (1-t)t & t^{2}  \\
	  	 	1-t & t & 0 \\
	  	 	1 & 0 & 0 \\
	  	 	\end{bmatrix} = 
	  	 	\begin{bmatrix}
	  	 	1 & 0 & 0 \\
	  	 	0 & 1-t & t \\
	  	 	0 & 1 & 0 \\
	  	 	\end{bmatrix}\begin{bmatrix}
	  	 	1-t & (1-t)t & t^{2}  \\
	  	 	1 & 0 & 0 \\
	  	 	0 & 1 & 0 \\
	  	 	\end{bmatrix}= \]
	  	 	\[\begin{bmatrix}
	  	 	1 & 0 & 0 \\
	  	 	0 & 1-t & t \\
	  	 	0 & 1 & 0 \\
	  	 	\end{bmatrix}\begin{bmatrix}
	  	 	1-t & t & 0  \\
	  	 	1 & 0 & 0 \\
	  	 	0 & 0 & 1 \\
	  	 	\end{bmatrix}\begin{bmatrix}
	  	 	1 & 0 & 0 \\
	  	 	0 & 1-t & t \\
	  	 	0 & 1 & 0 \\
	  	 	\end{bmatrix}= \phi_{3}(\sigma_{2}\sigma_{1}\sigma_{2})\]
	  	 	
	  	 
	 	 \item Para el movimiento elemental M3:\\
	 	 $ \phi_{n}(\sigma_{i}\sigma_{j}) = \phi_{n}(\sigma_{j}\sigma_{i})$, siendo $i<n, |i-j| > 1$\\
	 	 Sin pérdida de generalidad podemos suponer $ i+1<j $ de modo que:
	 	 \[ \phi_{n} (\sigma_{i}) \phi_{n} (\sigma_{j}) = \begin{bmatrix}
	 	 I_{i-1} &  &  & \\
	 	 & 1-t & t &  \\
	 	 & 1 & 0 &  \\
	 	 &  &  & I_{j-i-2} \\
	 	 & & & & 1-t & t &  \\
	 	 & & & & 1 & 0 &  \\
	 	 & & & & & & I_{n-j-1}  \\
	 	 \end{bmatrix}= \phi_{n} (\sigma_{j}) \phi_{n} (\sigma_{i})\]
 	 \end{itemize}
 	 
 	 Veamos ahora que se verifican las igualdades referentes a los movimientos de Markov.
 	 \begin{itemize}
 	 	\item 
 	 Veamos que se verifica la igualdad para el primer movimiento de Markov, es decir, veamos que se verifica $ det[\phi_{n}(\gamma\beta\gamma^{-1}) - I_{n}]_{1,1} = det[\phi_{n}(\beta) - I_{n}]_{1,1}$:\\
 	 Consideramos las matrices $ S $ y $ S^{-1} $ que definimos anteriormente y definimos los productos de matrices     
 	 \[S^{-1}\phi_{n}(\gamma)S = \left[\begin{array}{r|r}
 	 \Lambda (\gamma) & 0 \\ \hline	
 	 * ... * & 1\\
 	 \end{array}\right];
 	 S^{-1}\phi_{n}(\beta)S = \left[\begin{array}{r|r}
 	 \Lambda (\beta) & 0 \\ \hline	
 	 * ... * & 1\\
 	 \end{array}\right];
 	 S^{-1}\phi_{n}(\gamma^{-1})S = \left[\begin{array}{r|r}
 	 \Lambda (\gamma)^{-1} & 0 \\ \hline	
 	 * ... * & 1\\
 	 \end{array}\right]\]\\
 	 
 	 De modo que  	 
 	 \[ S^{-1}\phi_{n}(\gamma\beta\gamma^{-1})S = \left[\begin{array}{r|r}
 	 \Lambda (\gamma)\Lambda (\beta)\Lambda (\gamma)^{-1} & 0 \\ \hline	
 	 * ...................... * & 1\\
 	 \end{array}\right]\]\\
	 Por el corolario \ref{corlem} sabemos que \\
	 $ (1+t+..+ t^{n-1} ) det[ \phi_{n}(\gamma\beta\gamma^{-1}) - I_{n}] _{1,1} = det[S^{-1}(\phi_{n}(\gamma\beta\gamma^{-1}))S-I_{n}]_{n,n}$\\
	 
	 Desarrollando este segundo término tenemos:
	 \[det[S^{-1}(\phi_{n}(\gamma\beta\gamma^{-1}))S-I_{n}]_{n,n} = det[\Lambda (\gamma)\Lambda (\beta)\Lambda (\gamma)^{-1}-I_{n-1}] =\]
	 \[
	 det[\Lambda (\gamma)(\Lambda (\beta)-I_{n-1})\Lambda (\gamma)^{-1}] =
	 det[\Lambda (\beta)-I_{n-1}] \]
	 
	 Aplicando de nuevo el corolario \ref{corlem} tenemos la siguiente igualdad\\
	 $det[\Lambda (\beta)-I_{n-1}]  = (1+t+..+ t^{n-1} ) det[ \phi_{n}(\beta) - I_{n}] _{1,1} $\\
	 
	 Luego obtenemos la igualdad:\\
	  $ (1+t+..+ t^{n-1} ) det[ \phi_{n}(\gamma\beta\gamma^{-1}) - I_{n}] _{1,1} = (1+t+..+ t^{n-1} ) det[ \phi_{n}(\beta) - I_{n}] _{1,1} $\\
	  
	  y podemos concluir
	  $ det[ \phi_{n}(\gamma\beta\gamma^{-1}) - I_{n}] _{1,1} = det[ \phi_{n}(\beta) - I_{n}] _{1,1} $\\
	  
	  \item 
	  Por último vamos a ver que se verifica la igualdad para el segundo movimiento de Markov, es decir, veamos que se verifica $ det[\phi_{n+1}(\beta\sigma_{n}) - I_{n+1}]_{1,1} = det[\phi_{n}(\beta) - I_{n}]_{1,1}$ .\\
	  
	  Llamemos $ M = \phi_{n}(\beta) $ de modo que 
 	 \[ \phi_{n+1}(\beta) = \left[\begin{array}{r|r}
 	  M & 0 \\ \hline	
 	  0 & 1\\
 	 \end{array}\right]\]\\	  
 	 
 	 Por otra parte sabemos que 
 	 \[ \phi_{n+1} (\sigma_{n}) = \begin{bmatrix}
 	 I_{n-1} &  &  \\
 	 & 1-t & t  \\
 	 & 1 & 0  \\
 	 \end{bmatrix}\]\\
 	 
 	 Por tanto, 
 	 \[ \phi_{n+1} (\beta\sigma_{n}) = \begin{bmatrix}
 	 & & & a_{1,n}(1-t) & a_{1,n}t\\
 	 & M_{n-1,n-1} & & ... & ...\\
 	 & & & a_{n-1,n}(1-t) & a_{n-1,n}t\\
 	 a_{n,1}& ... & a_{n,n-1} & a_{n,n}(1-t) & a_{n,n}t\\
 	 & & & 1 & 0\\
 	 \end{bmatrix}\]\\
	  
	 De este modo se tiene 
 	 \[det[\phi_{n+1}(\beta\sigma_{n}) - I_{n+1}]_{n,n} = det \left[\begin{array}{r|r}
    	M_{n-1,n-1} - I_{n-1} & \\ \hline	
 	    & -1\\
 	 \end{array}\right]\] 
 	 \begin{center}
 	 	$ = -det[M_{n-1,n-1}-I_{n-1}] = -det[\phi_{n}(\beta)-I_{n}]_{n,n} $ 
 	 \end{center} 
 	 
 	 Obteniendo la igualdad
 	  	 \begin{center}
 	  	 	$ det[\phi_{n+1}(\beta\sigma_{n}) - I_{n+1}]_{1,1} = det[\phi_{n}(\beta)-I_{n}]_{1,1} $ 
 	  	 \end{center}
      \end{itemize}
 \end{proof}


De este modo, el polinomio de Alexander de la trenza $\beta1 = \sigma1$ se obtendría del siguiente modo:\\
Por definición tenemos que  
\[ \phi_{2} (\beta1) = \begin{bmatrix}
1-t & t  \\
1 & 0 \\
\end{bmatrix}\]
luego 
\[ \phi_{2} (\beta1) - I_{2}= \begin{bmatrix}
-t & t  \\
1 & -1 \\
\end{bmatrix}\]
Finalmente tenemos 
\[ det(\phi_{2} (\beta1) - I_{2})_{1,1} = det(\begin{bmatrix}
-1 \\
\end{bmatrix}) = -1\].

Veamos ahora el polinomio de Alexander de una trenza más compleja, en concreto de la trenza $\beta2 = \sigma2\sigma3^{-1}$. Ya sabemos que 
 \[ \phi_{4} (\beta2) = \begin{bmatrix}
 1 & 0 & 0 & 0 \\
 0 & 1-t & 0 & t \\
 0 & 1 & 0 & 0  \\	
 0 & 0 & t^{-1} & 1-t^{-1} \\
 \end{bmatrix}\]
 luego
  \[ \phi_{4} (\beta2) - I_{4} = \begin{bmatrix}
  0 & 0 & 0 & 0 \\
  0 & -t & 0 & t \\
  0 & 1 & -1 & 0  \\	
  0 & 0 & t^{-1} & t^{-1} \\
  \end{bmatrix}\].
  
  Finalmente tenemos 
    \[ det(\phi_{4} (\beta2) - I_{n})_{1,1} = det(\begin{bmatrix}
    -t & 0 & t \\
     1 & -1 & 0  \\	
     0 & t^{-1} & t^{-1} \\
    \end{bmatrix}) = -1+1 = 0.\].
    
Por tanto tenemos que las trenzas $\beta1$ y $\beta2$ no pueden generar nudos equivalentes pues sus polinomios de Alexander son distintos.\\
    
\label{2sub3a}
\section{El problema de las palabras.}\label{deh}
En esta sección vamos a tratar de ver si dos trenzas dadas son equivalentes entre sí sin hacer uso de invariantes. Vamos a apoyarnos en la idea que vimos en la sección \ref{grupotrenzas}:\\

Consideramos el grupo $ B_{n} $ con la representación que vimos en el teorema \ref{teoBn}. El problema de ver si dos trenzas dadas son equivalentes se puede ver como el problema de las palabras del grupo $ B_{n} $.\\

\textbf{Problema de las palabras del grupo de las trenzas:}\\
Dadas dos palabras de trenzas $\beta1, \beta2 \in B_{n}$ tratamos de encontrar algún método que nos permita confirmar si son o no equivalentes. \\

Ver si dos palabras (de trenzas) dadas $\beta1, \beta2 \in B_{n}$ son equivalentes se pude ver cómo el problema de ver si la palabra $\beta1\beta2^{-1}$ es equivalente a la palabra vacía (trenza trivial). Por tanto el problema de las palabras se puede reducir a distinguir si una palabra es equivalente o no a la palabra vacía. \\

En este proyecto vamos a ver el método de Patrick Dehornoy que nos permite reducir las palabras de trenzas dadas hasta una forma específica que nos permitirá ver si la palabra es igual a la cadena vacía. Para verlo con más detalle necesitamos ver algunas ideas previas.\\

\underline{\textbf{Definición:}}\\
Diremos que una palabra es \textbf{libremente reducida} si no contiene secuencias de la forma $\sigma_{i}^{-1}\sigma_{i} $ ni $\sigma_{i}\sigma_{i}^{-1}$.\\

Por ejemplo, la palabra $\sigma2\sigma1^{-1}\sigma1\sigma3^{-1}$ que representa a la trenza de la figura \ref{deh1} no es libremente reducida pues contiene la palabra $\sigma1^{-1}\sigma1$.\\

	\begin{figure}[h!]
		\centering
		\includegraphics[width=3cm]{itrenzas/deh1.png}
		\caption{Trenza no libremente reducida.}
		\label{deh1} 
	\end{figure}

\underline{\textbf{Definición:}}\\
Sea la palabra $\beta \in B_{n}$. Llamaremos \textbf{generador principal} de $\beta$ al generador de $\beta$ de menor índice.\\

Por ejemplo, la palabra $\sigma2\sigma1^{-1}\sigma1\sigma3^{-1}$ tiene como generador principal el 1, mientras que la palabra $\sigma2\sigma4^{-1}$ tiene como generador principal el número 2. \\

\textbf{\underline{Definición:}}
Diremos que una palabra $\beta \in B_{n}$ es \textbf{reducida} si se cumple alguna de estas condiciones:
\begin{enumerate}

	\item $\beta$ es la palabra cadena vacía.
	\item El generador principal de $\beta$ se presenta sólo positiva o negativamente. 
\end{enumerate}
 
Por ejemplo, la palabra $\sigma2\sigma1^{-1}\sigma3^{-1}\sigma1$ no es reducida pues el generador principal (1) se encuentra en un cruce negativo y en un cruce positivo. La palabra $\sigma2\sigma1\sigma3^{-1}\sigma1$ si sería reducida pues el generador principal (1) se presenta sólo como cruces positivos.\\

Si tenemos una palabra no reducida con generador principal $ \sigma_{i} $, podemos considerar ocurrencias de la palabra de la forma $ \sigma_{i}^{\pm} ... \sigma_{i}^{\mp} $ que no contengan cruces $ \sigma_{i} $, es decir, entre los cruces de signo opuesto del generador $ \sigma_{i} $ sólo encontramos generadores $ \sigma_{k} $, $ k>i $. Podemos visualizar la idea en la figura \ref{h1}. En este caso diremos que la cadena i+1 forma a\textbf{ handle}.\\
\begin{figure}[h!]
	\centering
	\includegraphics[width=4.5cm]{itrenzas/h11.png}
	\caption{Main handle.}
	\label{h1} 
\end{figure}

\begin{pro}
	Una palabra reducida no vacía no puede ser equivalente a la cadena vacía.
\end{pro}

Por tanto, si encontramos un método para obtener la palabra reducida de una palabra, podremos resolver el problema de las palabras: \\
Si la palabra $\beta1$ reducida es equivalente a la palabra $\beta2$, entonces $\beta2$ es equivalente a la cadena vacía si y sólo si $\beta1$ es la cadena vacía. \\

\begin{pro}
	Cualquier palabra admite una palabra reducida. 
\end{pro}

Para ver una demostración de esta proposición, vamos ir viendo el método que realizaremos para transformar una palabra cualquiera en su palabra reducida.\\ 
Con este método tratamos de eliminar los handles de la trenza, pero para no entrar en bucles infinitos en el algoritmo, vamos a tener que considerar handles generados por cualquier generador y no sólo por el generador principal.\\

\underline{\textbf{Definición:}}\\
 Un \textbf{$ \sigma_{j} $-handle} de una palabra es una sub-palabra de la forma $ \sigma_{j}^{\pm} v \sigma_{j}^{\mp} $ donde la sub-palabra $v$ sólo contiene generadores $\sigma_{k}^{(-1)} $ con k $<$ j-1 o k $>$j. Si el generador $\sigma_{j}$ es el generador principal de la palabra, a dicho $ \sigma_{j} $-handle se conocerá como \textbf{main handle}.\\
 
 Podemos ver un $ \sigma_{j} $-handle en la figura \ref{h2}.\\
 \begin{figure}[h!]
 	\centering
 	\includegraphics[width=9cm]{itrenzas/h12.png}
 	\caption{A handle.}
 	\label{h2} 
 \end{figure}
 
 Supongamos que tenemos la palabra $ \sigma3\sigma1^{-1}\sigma2\sigma4^{-1}\sigma2^{-1}\sigma4^{-1}\sigma1 $. \\
 La sub-palabra $ \sigma1^{-1}\sigma2\sigma4^{-1}\sigma2^{-1}\sigma4^{-1}\sigma1 $ es un main handle mientras que $\sigma2\sigma4^{-1}\sigma2^{-1} $ es un $ \sigma2 $-handle.\\
 
\underline{\textbf{ Definición:}}\\
Sea $\sigma_{j}^{e}$ $v$ $\sigma_{j}^{-e} $ un $\sigma_{j}$-handle de una palabra $\beta$, donde $e \in$ $ \{-1,1\} $. En particular, podemos denotar dicho $\sigma_{j}$-handle como la siguiente sub-palabra:
\begin{center}
	$ \sigma_{j}^{e} $ $ v_{0} $ $\sigma_{j+1}^{d_{1}} $ $ v_{1}...v_{m-1} $ $\sigma_{j+1}^{d_{m}} $ $ v_{m} $ $ \sigma_{j}^{-e} $
\end{center} donde $v_{0},...,v_{m}$ no contienen generadores $\sigma_{k}^{(-1)}$ con $ j-1 \le k \le j+1 $, $d_{i} \in$ $ \{-1,1\} $\\
Podemos aplicarle una \textbf{reducción local} quedando la sub-palabra equivalente:
\begin{center}
 $ v_{0} $ $\sigma_{j+1}^{-e} \sigma_{j}^{d_{1}} \sigma_{j+1}^{e}$ $ v_{1}...v_{m-1} $ $\sigma_{j+1}^{-e} \sigma_{j}^{d_{m}} \sigma_{j+1}^{e}$ $ v_{m} $
\end{center}
En definitiva hemos aplicado el homomorfismo $\phi_{j,e}$ definido como:
\begin{center}
	$\sigma_{j}^{\pm 1} \rightarrow \epsilon$\\
	$\sigma_{j+1}^{\pm 1} \rightarrow \sigma_{j+1}^{-e} \sigma_{j}^{\pm 1} \sigma_{j+1}^{e}$\\
	$\sigma_{k}^{\pm 1} \rightarrow \sigma_{k}^{\pm 1},$ siendo k$ \neq $j,j+1.\\
\end{center}

Podemos ver esta transformación en la figura \ref{h3}.\\
\begin{figure}[h!]
	\centering
	\includegraphics[width=14cm]{itrenzas/h14.png}
	\caption{Reducción local.}
	\label{h3} 
\end{figure}

Es importante destacar que antes de realizar una reducción local a una palabra, tendremos que asegurarnos de que dicha palabra sea libremente reducida para evitar una mayor complejidad.\\

Sea la palabra $\sigma2\sigma1^{-1}\sigma2\sigma3\sigma1$. Podemos aplicar una reducción local al main handle $\sigma1^{-1}\sigma2\sigma3\sigma1$ quedando la palabra $\sigma2\sigma1\sigma2^{-1}\sigma3$. Esta sería su palabra reducida.\\

Pero antes de hacer una reducción local a un $\sigma_{j}$-handle, tendremos que asegurarnos que no tenemos ningún $\sigma_{j+1}$-handle en el $\sigma_{j}$-handle. En el caso de tenerlo, tendríamos que aplicar la reducción local a este $\sigma_{j+1}$-handle y posteriormente aplicar la reducción local al $\sigma_{j}$-handle. De este modo podremos evitar ciclos infinitos. Veámoslo con un ejemplo:\\

Supongamos que queremos obtener la palabra reducida de la palabra $\sigma1\sigma2\sigma3\sigma2^{-1}\sigma1^{-1}$. El generador principal es 1 y tenemos un main handle (que, en este caso, es toda la palabra en sí misma). Aplicamos una reducción local a este main handle quedando: \\
$\sigma2^{-1}\sigma1\sigma2\sigma3\sigma2^{-1}\sigma1^{-1}\sigma2$.
Vemos que volvemos a tener un main handle y entraremos en bucle infinito. Veamos cómo aplicaríamos la solución que hemos comentado:\\

Supongamos de nuevo que queremos obtener la palabra reducida de la palabra $\sigma1\sigma2\sigma3\sigma2^{-1}\sigma1^{-1}$. Tenemos un main handle pero dicho main handle consta de una sub-palabra $\sigma2$-handle. Aplicamos una reducción local a este $\sigma2$-handle quedando:\\
$\sigma1\sigma3^{-1}\sigma2\sigma3\sigma1^{-1}$. Ahora sí podemos aplicar una reducción local al main handle quedando la palabra reducida: \\
$\sigma3^{-1}\sigma2^{-1}\sigma1\sigma2\sigma3$.\\

\underline{\textbf{Definición:}} \\
Diremos que un  $\sigma_{j}$-handle está permitido si no contiene ningún $\sigma_{j+1}$-handle.\\

Por ejemplo el main handle $\sigma1\sigma2\sigma3\sigma2^{-1}\sigma1^{-1}$ no está permitido porque incluye al $\sigma2$-handle $\sigma2\sigma3\sigma2^{-1}$. Este 2-handle sí estaría permitido porque no incluye ningún $\sigma3$-handle. \\

Si un $\sigma_{j}$-handle no está permitido, le aplicaremos reducciones locales, como hemos hecho en el ejemplo, hasta que pase a estar permitido.\\

\underline{\textbf{Definición:}} \\
Diremos que una palabra $\beta'$ es deducida de una palabra $\beta$ usando una \textbf{reducción de handle} si $\beta'$ se obtiene a partir de una reducción local de un $\sigma_{j}$-handle permitido de $\beta$.\\

\begin{teo}
	Cualquier palabra puede ser reducida por una secuencia finita de reducciones de handle hasta llegar a su palabra reducida.
\end{teo}

\bigskip
Para poner en uso todas estas ideas vamos a ver un ejemplo en el que demostraremos que dos palabras dadas son equivalentes:\\
Supongamos que tenemos las palabras $\beta1=\sigma2\sigma3^{-1}\sigma1\sigma2^{-1}\sigma3$ y $\beta2=\sigma1\sigma2\sigma3^{-1}\sigma1\sigma2^{-1}$ que representan a las trenzas de la figura \ref{h4}.\\

\begin{figure}[h!]
	\centering
	\subfigure[$\beta1$]{\includegraphics[width=2.3cm]{itrenzas/deho2.png}}
	\subfigure[$\beta2$]{\includegraphics[width=2.2cm]{itrenzas/deho1.png}}
	\caption{Trenzas equivalentes.}
	\label{h4} 
\end{figure}

A simple vista es difícil saber si las palabras son o no equivalentes. Para ver que las palabras son equivalentes ($\beta1 \sim \beta2$), tenemos que ver que se verifica $\beta1 \beta2^{-1} \sim \epsilon$, donde $\epsilon$ representa a la cadena vacía.\\

En primer lugar vamos a considerar la palabra con la que vamos a trabajar:\\
$\beta1 \beta2^{-1} =\sigma2\sigma3^{-1}\sigma1\sigma2^{-1}\sigma3\sigma2\sigma1^{-1}\sigma3\sigma2^{-1}\sigma1^{-1}$.\\

Nos encontramos el main handle $ \sigma1\sigma2^{-1}\sigma3\sigma2\sigma1^{-1} $. Vamos a aplicarle una reducción local, pero vemos que contiene al $\sigma2$-handle $\sigma2^{-1}\sigma3\sigma2$. Por tanto aplicamos una reducción local a este $\sigma2$-handle:\\
$\sigma2^{-1}\sigma3\sigma2$ $\rightarrow$ $\sigma3\sigma2\sigma3^{-1}$, obteniendo la palabra:\\
$\beta1 \beta2^{-1} =\sigma2\sigma3^{-1}\sigma1\sigma3\sigma2\sigma3^{-1}\sigma1^{-1}\sigma3\sigma2^{-1}\sigma1^{-1}$.\\

Nos encontramos el main handle permitido $\sigma1\sigma3\sigma2\sigma3^{-1}\sigma1^{-1}$. Le aplicamos una reducción local:\\ 
$\sigma1\sigma3\sigma2\sigma3^{-1}\sigma1^{-1}$ $\rightarrow$ $\sigma3\sigma2^{-1}\sigma1\sigma2\sigma3^{-1}$, obteniendo la palabra:\\
$\beta1 \beta2^{-1} =\sigma2\sigma3^{-1}\sigma3\sigma2^{-1}\sigma1\sigma2\sigma3^{-1}\sigma3\sigma2^{-1}\sigma1^{-1}$.\\

Esta palabra no es libremente reducida porque nos encontramos un par de veces la sub-palabra $ \sigma3^{-1}\sigma3 $. Las eliminamos y obtenemos la palabra:\\
$\beta1 \beta2^{-1} =\sigma2\sigma2^{-1}\sigma1\sigma2\sigma2^{-1}\sigma1^{-1}$.\\

De nuevo nos encontramos con una palabra que no es libremente reducida porque nos encontramos un par de veces la sub-palabra $ \sigma2\sigma2^{-1} $. Las eliminamos y obtenemos la palabra:\\
$\beta1 \beta2^{-1} =\sigma1\sigma1^{-1}$.\\

Otra vez nos encontramos con una palabra no libremente reducida. Eliminamos la sub-palabra $\sigma1\sigma1^{-1}$ quedando la cadena vacía, es decir:\\
$\beta1 \beta2^{-1} =\epsilon$.\\

Concluimos que ambas palabras son equivalentes, luego las trenzas a las que representan son equivalentes. Sus trenzas cerradas serán equivalentes y por tanto los nudos a los que se está representando son también equivalentes.\\
\label{2sub4}
PONER CONCLUSIONES EN GENERAL Y TRABAJO FUTURO!!
\label{2sub5}

\chapter{Desarrollo informático.}
\label{ch3}
El objeto principal de estudio de este proyecto es la teoría de nudos. Sin embargo, hemos visto que esta teoría esta estrechamente relacionada con la teoría de trenzas.\\ En este capítulo realizaremos el desarrollo informático sobre teoría de trenzas y trabajaremos con trenzas cerradas, que se pueden ver como si fuesen nudos. El motivo que nos ha llevado a trabajar con teoría de trenzas en lugar de con teoría de nudos se puede resumir en el hecho de que la teoría de trenzas proporciona ciertos atributos con los que podemos desarrollar la teoría de forma más cómoda que en la teoría de nudos. \\

Antes de comenzar a realizar dicho desarrollo hemos investigado y hemos encontrado varios software que permiten trabajar computacionalmente con trenzas y nudos. Algunos de lo más destacados son los siguientes:
\begin{itemize}
	\item braidlab \cite{9}: se trata de un paquete Matlab que permite analizar datos usando trenzas. 
	\item knotilus \cite{8}: nos permite trabajar con nudos, visualizándolos y obteniendo sus notaciones de Dowker y Gauss entre otras herramientas.
	\item braid program \cite{7}: nos permite trabajar con trenzas. Lo más interesante de este programa es que nos permite obtener, a partir del algoritmo de Vogel, la trenza que genera cierto nudo, aunque no nos permite su visualización. 
\end{itemize}

Hemos implementado nuestro propio programa libre y gratuito porque estos programas ya existentes no proporcionan unas visualizaciones agradables para el usuario además de que la documentación de algunos de ellos no está muy completa y no resultan intuitivos de usar. Además, la implementación que hemos realizado ha ido pasando por distintas versiones hasta conseguir un diseño lo más simple posible, de modo que es fácil añadir nuevas funcionalidades e incluso mejoras. \\

Un aspecto importante a destacar es el cambio de notación que vamos a hacer sobre las trenzas para que al usuario le resulte mucho más cómodo trabajar:
\begin{itemize}
	\item A los cruces negativos de una trenza $ \sigma_{i}^{-1} $ les denotaremos como $ -\sigma_{i}$.
	\item A los cruces positivos de una trenza $ \sigma_{i}^{1} $, o simplemente $ \sigma_{i} $, les denotaremos como $ +\sigma_{i} $
\end{itemize}

Por ejemplo, la trenza $\beta = \sigma4^{-1}\sigma1^{-1}\sigma2$ la denotaremos como $\beta = -\sigma4-\sigma1+\sigma2$.\\ 


\newpage
\section{Teoría de trenzas con Matlab.}
Hemos implementado bajo Matlab R2015a todos los aspectos matemáticos que desarrollamos en el capítulo \ref{ch2}. Hemos creado métodos básicos para poder trabajar correctamente con el grupo de las trenzas. Además, hemos implementado un algoritmo muy básico para ver si una trenza dada es pura o no: diremos que una trenza es pura si su permutación se corresponde con la permutación identidad. También cabe comentar que sobre trenzas cerradas hemos implementado la notación de Dowker que vimos en el capítulo \ref{ch1} sobre teoría de nudos. \\

Para recopilar todo este conjunto de herramientas que nos permiten trabajar con trenzas hemos creado \textbf{toxtren}. Se trata de un toolbox, una librería de funciones Matlab, asociado al estudio de trenzas y trenzas cerradas \cite{6}.  \\ 

Para hacer uso de toxtren, el usuario simplemente tiene que disponer de Matlab e instalar toxtren.mltbx: seleccionamos como nuevo directorio de trabajo la carpeta que contiene el archivo .mltbx, hacemos click derecho sobre él e instalamos. \\

Para realizar el control de versiones de toxtren hemos utilizado git.\\

A la hora de implementar toxtren hemos seguido un diseño orientado a objetos ya que las trenzas se pueden representar de una forma cómoda haciendo uso de objetos y clases. En concreto, hemos construido dos clases: 
\begin{itemize}
	\item Clase trenza: mediante esta clase podremos crear cualquier trenza, visualizarla en 3D, obtener los invariantes que hemos explicado en el capítulo \ref{ch2}, estudiar su trivialidad, realizar operaciones básicas con distintas trenzas....
	\item Clase trenza\_cerrada: la clase trenza\_cerrada hereda de la clase trenza pues una trenza cerrada es una trenza en la que conectamos los extremos de las cadenas. Al hacer esta unión sobre los extremos de las cadenas podemos obtener varios enlaces: si únicamente tenemos un enlace, dicha trenza cerrada será equivalente a un nudo. \\
	En dicha clase heredamos los métodos que hemos comentado para la clase trenza y además estudiamos la notación Dowker de la trenza cerrada, el polinomio de Alexander, podemos realizar los movimientos de Markov...
\end{itemize} 

Podemos ver un esquema de la herencia de las clases con sus atributos y métodos en la figura \ref{escl}.
\newpage
\begin{figure}[h!]
	\centering
	\includegraphics[width=16cm]{img/Main.jpg}
	\caption{Estructura clases}
	\label{escl} 
\end{figure}


\newpage
\section{Pseudoalgoritmos.}
En esta sección vamos a ver algunos pseudoalgoritmos referentes a los algoritmos que hemos implementado. Nos vamos a centrar en aquellos que tienen mayor complejidad e interés, sin entrar en gran detalle. Es importante destacar que estos algoritmos no están establecidos como tales, sino que los hemos implementando intentando reflejar los conceptos matemáticos que hemos ido viendo. \\

En concreto, vamos a ver los pseudoalgoritmos para el algoritmo de Dehornoy (que descomponemos en varios algoritmos auxiliares), el algoritmo que nos permite comprobar si dos trenzas dadas (o sus cierres) son o no equivalentes entre sí y el algoritmo que nos permite comprobar si una trenza dada (o su cierre) es equivalente a la trenza trivial.\\

\begin{center}
	\textbf{Algoritmo Dehornoy para trenzas.}
\end{center} 
Vamos a ir viendo los algoritmos auxiliares y concluiremos con el propio algoritmo de Dehornoy. Recordemos que este algoritmo se explica con detalle en la sección \ref{deh}.\\

Creamos el método Simplifica mediante el cuál eliminamos ocurrencias de tipo $\sigma_{i}^{e}\sigma_{i}^{-e}$, siendo $ e \in$ \{-1,1\}, de una palabra que representa a cierta trenza. 

\begin{alg}
	\textbf{Algoritmo Simplifica}(indices\_braid)\\
	ENTRADA: indices\_braid (cadena de enteros que representa los cruces de una trenza)\\
	SALIDA: \hspace{0.4cm} braid\_aux (cadena auxiliar para mejor representacion visual) \\
    \hspace*{2.2cm} nueva\_braid (cadena de enteros que representa los cruces de la trenza tras eliminar dos cruces consecuetivos opuestos)\\
    \hspace*{2.2cm} encontrado (bool para indicar si se produce simplificación)
	
	\lstset{language=Matlab, breaklines=true, basicstyle=\ttfamily\small ,numbers=left, stepnumber=1, numberstyle=\color{blue}}
\begin{lstlisting}
   Recorro los cruces de indices_braid
   Si hay dos cruces seguidos con signos opuestos, creo una copia de indices_braid y elimino dichos cruces. 
\end{lstlisting}
\end{alg}

Haciendo uso del siguiente algoritmo podremos encontrar las posiciones que delimitan un $\sigma_{minimo}$-handle.
\begin{alg}
	\textbf{Algoritmo encuentra\_handle}(indices\_braid, minimo)\\
	ENTRADA: indices\_braid (cadena de enteros que representa los cruces de una trenza)\\
	\hspace*{2.2cm} minimo (generador de handle a encontrar) \\
	SALIDA: \hspace{0.4cm} pos1 (posición inicial del handle encontrado) \\
	\hspace*{2.2cm} pos2 (posición final del handle encontrado)
	
	\lstset{language=Matlab, breaklines=true, basicstyle=\ttfamily\small ,numbers=left, stepnumber=1, numberstyle=\color{blue}}
\begin{lstlisting}
   Recorro los cruces de indices_braid
   Si hay un cruce generado por el elemento minimo, me quedo con esa posicion como pos1 y su signo. Si no finalizo.
   Si encuentro un cruce generado por el elemento minimo con signo opuesto, me quedo con esa posicion como pos2.
\end{lstlisting}
\end{alg}

\bigskip
Haciendo uso del algoritmo reduccion\_base aplicamos una reducción local a la trenza representada por indices\_braid sobre el $\sigma_{minimo}$-handle que está situado entre las posiciones pos1 y pos2. Es importante destacar el hecho de que este $\sigma_{minimo}$-handle  no contiene $\sigma_{minimo+1}$-handle. Este detalle lo controlamos en el algoritmo Dehornoy que veremos posteriormente. \\

\begin{alg}
	\textbf{Algoritmo reduccion\_base}(indices\_braid, minimo, pos1, pos2 )\\
	ENTRADA: indices\_braid (cadena de enteros que representa los cruces de una trenza)\\
	\hspace*{2.2cm} minimo (generador de handle) \\
	\hspace*{2.2cm} pos1 (posición inicial del handle) \\
	\hspace*{2.2cm} pos2 (posición final del handle)\\
	SALIDA: \hspace{0.4cm} braid\_aux2 (cadena auxiliar para mejor representación visual) \\
	\hspace*{2.2cm} nuevo (cadena de enteros que representa los cruces de la trenza tras aplicar la reducción local al $\sigma_{minimo}$-handle entre pos1 y pos2)\\
	\hspace*{2.2cm} simplificado2 (bool auxiliar para mejor representación visual)
	
	\lstset{language=Matlab, breaklines=true, basicstyle=\ttfamily\small ,numbers=left, stepnumber=1, numberstyle=\color{blue}}
\begin{lstlisting}
   Creo vector_auxiliar
   Recorro los cruces de indices_braid desde pos1 hasta pos2
   Si hay un cruce generado por el elemento (minimo+1) aniado a vector_auxiliar los 3 cruces correspondientes del algoritmo. Si no, aniado el mismo cruce. 
   Creo vector_nuevo con los elementos desde inicio de indices_braid hasta pos1, vector_auxiliar, y los elementos desde pos2 hasta final de indices_braid.
   Si indices_braid y vector_auxiliar tienen distinto tamanio, asigno a braid_aux2 una cadena con ciertos ceros para mejor visualizacion.
\end{lstlisting}
\end{alg}


Estos algoritmos auxiliares no se proporcionan para uso directo al usuario ya que el realmente interesante es el algoritmo de Dehornoy, pero sí podrían ser usados. Veamos entonces el algoritmo de Dehornoy.\\

\begin{alg}
	\textbf{Algoritmo dehornoy}(br, N\_cortes, Radio, representar)\\
	ENTRADA: br (trenza)\\
	\hspace*{2.2cm} N\_cortes (numero de cortes de las cadenas de la trenza)\\
	\hspace*{2.2cm} Radio (radio de las cadenas de la trenza)\\
	\hspace*{2.2cm} representar (bool para representar las equivalencias de la trenza)\\
	SALIDA: \hspace{0.4cm} es\_trivial (bool que nos indica si la trenza dada es o no trivial) \\
	\hspace*{2.2cm} trenza\_final (cadena de enteros que representa a la trenza reducida equivalente a br)
	
	\lstset{language=Matlab, breaklines=true, basicstyle=\ttfamily\small ,numbers=left, numbersep=-2pt, numberstyle=\color{blue}}
\begin{lstlisting}
   Si numero_argumentos=1 -> N_cortes=20, Radio=0.5, representar=1.
   indices_braid = cadena de enteros que representa a la trenza
   Si br tiene cadenas a la derecha triviales, las eliminamos visualmente.
   Mientras queden handles en la trenza dada...
	 Obtenemos la palabra libremente reducida de indices_braid.
	 Si no se produce reduccion...
	   minimo = generador principal de indices_braid.
	   [pos1,pos2]=encuentra_handle(indices_braid,minimo)
	   Si pos1 y pos2 son posiciones validas....
		   Busco primer subhandle a realizar en el handle.
		   Actualizo pos1 y pos2
		   Aplico reduccion_base a dicho subhandle.
	 Creo matriz con secuencia de palabras generadas en el proceso
   Si representar, muestro las trenzas usando dicha matriz.			   
\end{lstlisting}
\end{alg}

Finalmente cabe comentar que el algoritmo de Dehornoy se podrá aplicar sobre trenzas cerradas. El proceso que se realizará internamente será exactamente el mismo que para trenzas, ya que el cerrar la trenza no aporta información nueva para el algoritmo de Dehornoy. Por este mismo motivo, a la hora de realizar la visualización del algoritmo de Dehornoy sobre una trenza cerrada, veremos las transformaciones sobre la trenza y no sobre la trenza cerrada. \\

\bigskip
\begin{center}
	\textbf{Algoritmo de equivalencia para trenzas.}
\end{center} 
Para ver si dos trenzas dadas son equivalentes o no entre sí, vamos a implementar algunos de los invariantes que vimos para trenzas. En concreto, vamos a estudiar los invariantes exponente y permutación. Si con estos invariantes no conseguimos analizar la equivalencia de las trenzas, vamos a aplicar el algoritmo de Dehornoy que hemos visto anteriormente. \\

\begin{alg}
	\textbf{Algoritmo equivalentes}(br1,br2,explicacion)\\
	ENTRADA: br1 (trenza1)\\
	\hspace*{2.2cm} br2 (trenza2)\\
	\hspace*{2.2cm} explicacion (bool para mostrar mensajes explicativos)\\
	SALIDA: \hspace{0.4cm} equi (bool para indicar si br1 y br2 son o no equivalentes)
	
	\lstset{language=Matlab, breaklines=true, basicstyle=\ttfamily\small ,numbers=left, stepnumber=1, numberstyle=\color{blue}}
\begin{lstlisting}
   Si numero_argumentos=2 -> explicacion=0.
   Obtengo exponente de ambas trenzas.
   Si son distintos -> No son equivalentes. Finalizo
   Obtengo permutacion de ambas trenzas.
   Si son distintas -> No son equivalentes. Finalizo
   Genero trenza_auxiliar = br1inver(br2).
   Aplico Dehornoy a trenza_auxiliar y obtengo final_braid.
   Si final_braid no es vacia -> No son equivalentes. 
   Si no -> Si explicacion=1 -> represento secuencia de trenzas de dehornoy.
    
\end{lstlisting}
\end{alg}

\bigskip
\begin{center}
	\textbf{Algoritmo de equivalencia para trenzas cerradas.}
\end{center} 
Para analizar la equivalencia de dos trenzas cerradas, vamos a apoyarnos en el algoritmo de equivalencia de las trenzas sin cerrar que hemos visto anteriormente. Además haremos uso de varios algoritmos auxiliares. En concreto, hemos implementado los movimientos de Markov que vimos en la sección \ref{Markov}.\\

Sabemos que el movimiento 1 de Markov puede eliminar o añadir cruces. Siempre intentaremos eliminar cruces, mientras que para añadir cruces se tiene que solicitar explícitamente. Lo hacemos así para que las trenzas no aumenten en número de cruces salvo que no quede más opción. \\ 

\begin{alg}
	\textbf{Algoritmo MV1}(br\_c, completo)\\
	ENTRADA: br\_c (trenza cerrada)\\
	\hspace*{2.2cm} completo (bool para indicar si se desean aniadir cruces)\\
	SALIDA: \hspace{0.4cm} a2 (trenza cerrada tras aplicar Movimiento1 eliminando de Markov.) 
	
	\lstset{language=Matlab, breaklines=true, basicstyle=\ttfamily\small ,numbers=left, stepnumber=1, numberstyle=\color{blue}}
\begin{lstlisting}
   Si numero_argumentos=1 -> completo=0.
   Si el primer cruce de br_c es opuesto al ultimo cruce de br_c -> a2 = br_c sin dichos cruces.
   Si no -> Si completo... 
     Obtengo cruce de mayor indice (indice m).
     Aniado cruces opuestos de indice m a principio y final de br_c.
      
\end{lstlisting}
\end{alg}

\begin{alg}
	\textbf{Algoritmo MV2}(br\_c)\\
	ENTRADA: br\_c (trenza cerrada)\\
	SALIDA: \hspace{0.4cm} a3 (trenza cerrada tras aplicar Movimiento2 de Markov.) 
	
	\lstset{language=Matlab, breaklines=true, basicstyle=\ttfamily\small ,numbers=left, stepnumber=1, numberstyle=\color{blue}}
\begin{lstlisting}
   Mientras br_c cambie de cruces...
      Si br_c tiene un solo cruce -> a3=trenza cerrada trivial. Finalizo.
      Obtengo cruce de mayor indice (indice m). 
      Si se encuentra solo una vez en br_c
         Si a izquerda o derecha del cruce no tenemos cruces con indice m-1 -> a3=br_c sin dicho cruce.
         
\end{lstlisting}
\end{alg}

Para implementar este algoritmo de equivalencia entre trenzas cerradas hemos analizado en primer lugar los invariantes básicos de ambas trenzas. Si no obtenemos una respuesta de equivalencia  o no equivalencia, pasamos a ver el polinomio de Alexander de ambas trenzas cerradas. Una vez analizados los invariantes que hemos estudiado, vamos a ir realizando transformaciones de equivalencia sobre la trenza cerrada generada por el producto de la trenza inicial y la inversa de la segunda trenza. \\

Ya sabemos que ir haciendo transformaciones de equivalencia sobre una trenza cerrada para ver si es o no equivalente a la trenza cerrada trivial puede conllevar a una serie indefinida de movimientos. \\

La idea que hemos llevado a cabo consiste en aplicar el Movimiento 1 de Markov, el algoritmo de Dehornoy a la trenza cerrada que obtenemos, el Movimiento 2 de Markov y de nuevo el algoritmo de Dehornoy. Si en alguna de estas transformaciones conseguimos saber si la trenza cerrada es o no equivalente a la trivial, finalizamos el proceso. En caso contrario, realizaremos el mismo proceso un número delimitado de veces. \\

\newpage
\begin{alg}
	\textbf{Algoritmo equivalentes}(br1,br2,explicacion)\\
	ENTRADA: br1 (trenza cerrada 1)\\
	\hspace*{2.2cm} br2 (trenza cerrada 2)\\
	\hspace*{2.2cm} explicacion (bool para mostrar mensajes explicativos)\\
	SALIDA: \hspace{0.4cm} equi (bool para indicar si br1 y br2 son o no equivalentes)
	
	\lstset{language=Matlab, breaklines=true, basicstyle=\ttfamily\small ,numbers=left, stepnumber=1, numberstyle=\color{blue}}
\begin{lstlisting}
   Si numero_argumentos=2 -> explicacion=0.
   Si el numero de enlaces de br1 y de br2 son distintos -> No son equivalentes. Finalizo
   Si equivalentes@trenza(br1,br2) son equivalentes -> sus cierres son equivalentes.
   Si no ->   
   	Obengo polinomio de Alexander de ambas trenzas.
	Si son iguales salvo signo -> No sabemos si son o no equivalentes. equi <- 2.
	Sino -> No son equivalentes.
   Si equi=2
    Creo trenza cerrada br = br1inver(br2)
    Mientras numero_iteraciones < limite(establecido a 3)
	    a1 = Aplico MV1 a br.
	    [es_trivial2,a2] = Aplico dehornoy a trenza cerrada a1. 
	    Si es_trivial2 -> Finalizo.
	    a3 = Aplico MV2 a trenza cerrada a2.
	    [es_trivial4,a4] = Aplico dehornoy a trenza cerrada a3. 
	    Si es_trivial4 -> Finalizo.
	    Si no ha cambiado br, a4 = Aplico MV1 completo a br. 
	   
\end{lstlisting}
\end{alg}

\begin{center}
	\textbf{Algoritmo para ver trivialidad de trenza.}
\end{center} 
Tanto para ver la trivialidad de una trenza como la trivialidad de una trenza cerrada vamos a usar la misma idea que será hacer uso de los algoritmos de equivalencia entre la trenza dada y la trenza trivial. Veámoslos:

\begin{alg}
	\textbf{Algoritmo es\_trivial}(br,explicacion)\\
	ENTRADA: br (trenza)\\
	\hspace*{2.2cm} explicacion (bool para mostrar mensajes explicativos)\\
	SALIDA: \hspace{0.4cm} equi (bool para indicar si br es o no equivalente a la trenza trivial)
	
	\lstset{language=Matlab, breaklines=true, basicstyle=\ttfamily\small ,numbers=left, stepnumber=1, numberstyle=\color{blue}}
\begin{lstlisting}
   Si numero_argumentos=1 -> explicacion=0.
   Creo br2 = trenza trivial. 
   Aplico algoritmo equivalentes para br y br2.
\end{lstlisting}
\end{alg}

\begin{center}
	\textbf{Algoritmo para ver trivialidad de trenza cerrada.}
\end{center} 

\begin{alg}
	\textbf{Algoritmo es\_trivial}(br\_c,explicacion)\\
	ENTRADA: br\_c (trenza cerrada)\\
	\hspace*{2.2cm} explicacion (bool para mostrar mensajes explicativos)\\
	SALIDA: \hspace{0.4cm} equi (bool para indicar si br\_c es o no equivalente a la trenza cerrada trivial)
	
	\lstset{language=Matlab, breaklines=true, basicstyle=\ttfamily\small ,numbers=left, stepnumber=1, numberstyle=\color{blue}}
\begin{lstlisting}
   Si numero_argumentos=1 -> explicacion=0.
   Creo br2_c = trenza cerrada trivial. 
   Aplico algoritmo equivalentes para br_c y br2_c.
\end{lstlisting}
\end{alg}
\section{Usando toxtren}
En esta sección vamos a hacer un recorrido por el toolbox toxtren, de modo que el usuario podrá ver fácilmente cómo trabajar con el mismo. Haremos un repaso sobre los métodos esenciales y más importantes para el manejo de trenzas y trenzas cerradas. \\
 

\subsection{Clase trenza.}

\begin{center}
	\textbf{Constructor.}
\end{center}
En primer lugar vamos a ver el constructor para dicha clase. Podemos crear una trenza indicando los cruces de la trenza y el número de cadenas de la misma. Por ejemplo, creamos la trenza $\sigma1\sigma2^{-1}\sigma1^{-1}$ con 5 cadenas del siguiente modo:
	\lstset{language=Matlab, breaklines=true, basicstyle=\ttfamily\small}
\begin{lstlisting}
>> a=trenza([1 -2 -1],5)
\end{lstlisting}
	\lstset{emph=[1]{trenza,trenza_cerrada},emphstyle=[1]\color{blue}}
\begin{lstlisting}
	a = trenza with properties:
	    indices_trenza: [1 -2 -1]
	    n_cadenas: 5
\end{lstlisting}

Si no indicamos el número de cadenas de la trenza, se establecerá al número mínimo de cadenas que tendrá que tener. 
   \lstset{emph=[1]{trenza,trenza_cerrada},emphstyle=[1]\color{black}}
\begin{lstlisting}
>> a=trenza([1 -2 -1])
\end{lstlisting}
\lstset{emph=[1]{trenza,trenza_cerrada},emphstyle=[1]\color{blue}}
\begin{lstlisting}
	a = trenza with properties:
	    indices_trenza: [1 -2 -1]
	    n_cadenas: 3
\end{lstlisting}

Además para ambos casos podemos indicar la trenza mediante una cadena tal y cómo hemos comentado al inicio del capítulo:

\lstset{emph=[1]{trenza,trenza_cerrada},emphstyle=[1]\color{black}}
\begin{lstlisting}
>> b=trenza('+s3-s2',7)
\end{lstlisting}
\lstset{emph=[1]{trenza,trenza_cerrada},emphstyle=[1]\color{blue}}
\begin{lstlisting}
b = trenza with properties:
indices_trenza: [3 -2]
n_cadenas: 7
\end{lstlisting}

Del mismo modo que antes, si no indicamos el número de cadenas de la trenza se establecerá como el número mínimo que ha de tener. 
\lstset{emph=[1]{trenza,trenza_cerrada},emphstyle=[1]\color{black}}
\begin{lstlisting}
>> b=trenza('+s3-s2')
\end{lstlisting}
\lstset{emph=[1]{trenza,trenza_cerrada},emphstyle=[1]\color{blue}}
\begin{lstlisting}
	b = trenza with properties:
	    indices_trenza: [3 -2]
	    n_cadenas: 4
\end{lstlisting}



\newpage
\begin{center}
	\textbf{Operaciones básicas.}
\end{center}
Obtenemos la trenza potencia de una trenza.
\begin{lstlisting}
>> pote(a,3)
	ans = trenza with properties:
	      indices_trenza: [1 -2 -1 1 -2 -1 1 -2 -1] 
	      n_cadenas: 3
\end{lstlisting}


Podemos obtener la trenza producto de un par de trenzas.
\begin{lstlisting}
>> c=producto(a,b)
	c = trenza with properties:
	    indices_trenza: [1 -2 -1 3 -2]
	    n_cadenas: 4
\end{lstlisting}

Obtenemos la trenza inversa de una trenza haciendo uso de la función inver:
\begin{lstlisting}
>> c.inver
	ans = trenza with properties:
	      indices_trenza: [2 -3 1 2 -1]
	      n_cadenas: 4
\end{lstlisting}

Podemos obtener el número de cruces de una trenza del siguiente modo:
\begin{lstlisting}
>> c.length
	ans =5
\end{lstlisting}

\begin{center}
	\textbf{Get/Set.}
\end{center}
Podemos obtener los cruces y el número de cadenas de una trenza con los siguientes get. 
\begin{lstlisting}
>> a.get_indices
ans = 1    -2    -1
\end{lstlisting}

\begin{lstlisting}
>> a.get_n
ans = 3
\end{lstlisting}

Cambiamos el número de cadenas de una trenza del siguiente modo:
\begin{lstlisting}
>> set_n(a,6)
>> a
a = trenza with properties:
indices_trenza: [1 -2 -1]
n_cadenas: 6
\end{lstlisting}

Finalmente cambiamos los cruces de la trenza:
\begin{lstlisting}
>> set_indices(a,[4 -3])
>> a
a = trenza with properties:
indices_trenza: [4 -3]
n_cadenas: 5
\end{lstlisting}
Si no indicamos un tercer parámetro que representa el número de cadenas de la trenza modificada, se establecerá al número de cadenas por defecto. 

\newpage
\begin{center}
	\textbf{Invariantes básicos.}
\end{center}
Para ver los invariantes básicos vamos a trabajar con la trenza que hemos creado anteriormente:
\begin{lstlisting}
>> a
	a = trenza with properties:
	    indices_trenza: [4 -3]
	    n_cadenas: 5
\end{lstlisting}

Obtenemos el invariante exponente mediante el método exp:
\begin{lstlisting}
>> a.exp
	ans = 0
\end{lstlisting}

Obtenemos la permutación de la trenza:
\begin{lstlisting}
>> a.perm	
	ans = 1     2     4     5     3
\end{lstlisting}

A partir de esta permutación podemos ver si la trenza es pura. Al obtener respuesta negativa, sabemos que la trenza no es pura. 
\begin{lstlisting}
>> a.pura	
	ans = 0
\end{lstlisting}

\begin{center}
	\textbf{Otros métodos destacados.}
\end{center}
Podemos ver la representación 3D de una trenza del siguiente modo:
\begin{lstlisting}
>> a.representar_trenza
\end{lstlisting}
\begin{figure}[h!]
	\centering
	\includegraphics[width=5cm]{img/infor1.png}
	\caption{Representación trenza Matlab.}
	\label{inf1} 
\end{figure}

\newpage
Para aplicar el algoritmo de Dehornoy sobre una trenza y ver su representación, vamos a crear una nueva trenza más compleja y haremos uso del siguiente comando:

\lstset{emph=[1]{trenza,trenza_cerrada},emphstyle=[1]\color{black}}
\begin{lstlisting}
>> a=trenza([1 -2 2 2 -1])
\end{lstlisting}
\lstset{emph=[1]{trenza,trenza_cerrada},emphstyle=[1]\color{blue}}
\begin{lstlisting}
	a = trenza with properties:
	    indices_trenza: [1 -2 2 2 -1]
	    n_cadenas: 3
>> [e_trivial,final]=a.dehornoy
	e_trivial = 0
	final = -2     1     2
\end{lstlisting}
Podemos ver el principio y final del proceso en la figura \ref{inf2}
\begin{figure}[h!]
	\centering
	\includegraphics[width=1.2cm]{img/infor2.png}
	\includegraphics[width=2cm]{img/infor3.png}
	\caption{Transformación Dehornoy Matlab.}
	\label{inf2} 
\end{figure}

Creamos una trenza a partir de los cruces que hemos obtenido al aplicar el algoritmo de Dehornoy a la trenza a:
\lstset{emph=[1]{trenza,trenza_cerrada},emphstyle=[1]\color{black}}
\begin{lstlisting}
>> f=trenza(final)
\end{lstlisting}
\lstset{emph=[1]{trenza,trenza_cerrada},emphstyle=[1]\color{blue}}
\begin{lstlisting}
	f = trenza with properties:
	    indices_trenza: [-2 1 2]
	    n_cadenas: 3
\end{lstlisting}

Obtivamente al aplicar el algoritmo de equivalencia entre las trenzas a y f tendremos que obtener respuesta afirmativa:
\begin{lstlisting}
>> equivalentes(a,f)
	ans = 1
\end{lstlisting}

Finalmente podemos comprobar si una trenza dada es o no trivial. En este caso la trenza a no es equivalente a la trenza trivial. 
\begin{lstlisting}
>> a.es_trivial
	ans = 0
\end{lstlisting}

\subsection{Clase trenza\_cerrada.}

\begin{center}
	\textbf{Constructor.}
\end{center}

\chapter{Conclusiones y vías futuras}
\label{ch4}
PONER CONCLUSIONES EN GENERAL Y TRABAJO FUTURO!!


\bibliographystyle{apalike}
\bibliography{tex/bibliografia}\label{bibliog}
\addcontentsline{toc}{chapter}{Bibliografía}

\end{document}
