
En este proyecto hemos mostrado las bases matemáticas de la teoría de nudos y teoría de trenzas, quedando una gran cantidad de aspectos por tratar pues son teorías bastante ricas. Sin embargo, sirve como una fuerte base de conocimiento que permitirá el estudio de las teorías a posibles investigadores. Tras el proyecto podemos concluir que disponemos de un amplio número de campos en los que ambas teorías pueden ser aplicadas. \\

Los objetivos propuestos inicialmente han sido cubiertos en su totalidad con el matriz de que el desarrollo informático lo hemos realizado en base a teoría de trenzas y no a teoría de nudos, pero ésto nos ha permitido realizar el estudio de ambas teorías matemáticas.\\

\bigskip
Por otra parte, existen diversas vías futuras a partir del trabajo realizado. 
\begin{itemize}
	\item\textbf{Posible mejora de la visualización del movimiento de trenzas equivalentes en el algoritmo de Dehornoy:}\\
	Hemos ido mejorando la visualización de las trenzas hasta conseguir que el movimiento de las mismas sea lo menos brusco posible para que el usuario vea que los movimientos son completamente válidos. Sin embargo, algunos casos pueden resultar algo confunsos de visualizar porque intervienen numerosas cadenas. Además, se podría modificar la velocidad de los movimientos en función de su dificultad visual para hacerlo aún más agradable para el usuario. 
	\item \textbf{Implementar el algoritmo de Yamada-Vogel para demostrar el teorema de Alexander:}\\
	Hemos visto la demostración de este algoritmo desde un punto formal matemático pero sería interesante implementarlo y obtener una representación visual de los movimientos que se van produciendo. Además, con dicho algoritmo obtendríamos una trenza cerrada que representa un nudo concreto y viceversa. 
	\item \textbf{Creación de una interfaz de usuario en Matlab. }\\
	Sería interesante desarrollar una interfaz de usuario que englobe todas las funcionalidades de toxtren para aquellos usuarios que prefieren trabajar en un entorno visual sencillo en lugar de trabajar con la ventana de comandos. 
	\newpage
	\item\textbf{Profundización en la relación entre teoría de nudos y ADN. }\\
	Al conocer el tipo de nudos que nos encontramos en las estructuras iniciales y finales de ADN, nos interesaría deducir la acción de la encima que actúa sobre la estructura. \\
	Otro aspecto interesante que podríamos realizar aquí sería el siguiente: tras analizar una base de datos real que contenga la estructuras iniciales y finales de ADN y obtener la acción de las encimas que han actuado, podremos construir modelos que nos permitan predecir la estructura final de una estructura inicial de ADN.
\end{itemize}