\section{Nudos y trenzas.}\label{t2sec1}
En la sección \ref{seccion7} dimos una idea general de lo que se entiende por una trenza. Veamos su definición más formal:\\

\underline{\textbf{Definición:}}\\
Consideremos el cubo $\mathds{D} = \{(x,y,z) / 0 \leq x,y,z \leq 1\}$ y situamos $A_{i}$ puntos en su cara superior y $B_{i}$ puntos en la cara inferior, siendo $i \in \mathds{N}$. Unimos cada punto $A_{i}$ con un cierto punto $B_{k}$, $i \geq k \in \mathds{N}$, mediante arcos simples $d_{i}$ de modo que:
\begin{enumerate}
	\item $ d_{1}, d_{2},...,d_{n} $ sean disjuntos.
	\item Los arcos $ d_{i} $ no pueden conectar puntos $A_{i}$ o $B_{i}$ entre sí.
	\item Al cortar por cualquier plano horizontal, cada arco $ d_{i} $ toca en un sólo punto al plano. 
\end{enumerate}
A cada uno de estos arcos simples $ d_{i} $ les llamaremos cadenas y al conjunto de las n-cadenas se le conoce como \textbf{trenza}.\\

Podemos ver algunos ejemplos de trenzas en la figura \ref{tren1}.\\
\begin{figure}[h!]
	\centering
	\subfigure[$-\sigma2+\sigma1-\sigma2+\sigma1+\sigma3$]{\includegraphics[width=5cm]{itrenzas/t1cubo.png}}
	\space
	\subfigure[$+\sigma1-\sigma2+\sigma1-\sigma2$]{\includegraphics[width=5cm]{itrenzas/t2cubo.png}}
	\caption{Ejemplos de trenzas}
	\label{tren1} 
\end{figure} 

Al igual que hacíamos con los nudos, podremos representar una trenza en el plano visualizando su proyección. En la figura \ref{tren2} se pueden ver las proyección de las trenzas representadas en la figura \ref{tren1}.\\
\begin{figure}[h!]
	\centering
	\subfigure[$-\sigma2+\sigma1-\sigma2+\sigma1+\sigma3$]{\includegraphics[width=3.5cm]{itrenzas/t1pro.png}}
	\space
	\subfigure[$+\sigma1-\sigma2+\sigma1-\sigma2$]{\includegraphics[width=3cm]{itrenzas/t2pro.png}}
	\caption{Proyección de trenzas}
	\label{tren2} 
\end{figure}  

Anteriormente vimos que a cada trenza le corresponde un nudo o un enlace particular. Se obtendrá uniendo los extremos superiores con los extremos inferiores de las cadenas en el mismo orden. A este nudo se le conocerá como \textbf{trenza cerrada}. \\

Denotaremos como $\mathscr{B}_{n}$ al conjunto de todas las trenzas de n cadenas.\\


\bigskip
\begin{center}
	\subsection{Notación de trenzas:}
\end{center}
Para poder trabajar de forma cómoda con las trenzas vamos a darle la siguiente notación:\\
Sean los segmentos que unen las posiciones $i$ con la $i+1$ y las posiciones $i+1$ con la $i$. Al producir un intercambio de posiciones de estos segmentos se producirá un \textbf{cruce}. Este cruce puede realizarse de dos formas: 
\begin{itemize}
	\item El segmento que parte de la posición $i$ cruza por delante al segmento que inicialmente parte en la posición $i+1$. En este caso el cruce se denota como $-\sigma(i)$ y se conoce como un cruce negativo.
	\item  El segmento que parte de la posición $i$ cruza por detrás al segmento que inicialmente parte en la posición $i+1$. En este caso el cruce se denota como $+\sigma(i)$ y se conoce como un cruce positivo.
\end{itemize}
Podemos verlo más claro en la figura \ref{tren4}.\\
\begin{figure}[h!]
	\centering
	\subfigure[$-\sigma(i)$]{\includegraphics[width=3.5cm]{itrenzas/t5.png}}
	\space
	\subfigure[$+\sigma(i)$]{\includegraphics[width=3.4cm]{itrenzas/t6.png}}
	\caption{Signo cruce.}
	\label{tren4} 
\end{figure}

La \textbf{n-trenza trivial} se define como la n-trenza que no realiza ningún cruce. La denotaremos como $1_{n}.$ \\

Cualquier trenza no trivial tendrá de una serie de cruces. En cada plano horizontal podremos tener como mucho un cruce. Notaremos a la trenza con la secuencia de cruces que tenga, empezando por la parte superior de la trenza. A esta secuencia se le conoce como \textbf{palabra} que representa a la trenza. Podemos ver un ejemplo en la figura \ref{tren5}.\\
\begin{figure}[h!]
	\centering
	\includegraphics[width=6cm]{itrenzas/t7.png}
	\caption{Trenza $+\sigma3-\sigma2+\sigma4$.}
	\label{tren5} 
\end{figure}


\bigskip
\begin{center}
	\subsection{Equivalencia de trenzas:}
\end{center}
Intuitivamente diremos que dos trenzas son equivalentes si podemos deformar las cadenas de las trenzas de forma que ambas trenzas se vean iguales. Las trenzas de la figura \ref{tren3} son equivalentes.\\
\begin{figure}[h!]
	\centering
	\subfigure[$-\sigma1+\sigma2-\sigma2$]{\includegraphics[width=3.9cm]{itrenzas/t3.png}}
	\space
	\subfigure[$-\sigma1-\sigma2+\sigma2$]{\includegraphics[width=3.9cm]{itrenzas/t4.png}}
	\caption{Trenzas equivalentes.}
	\label{tren3} 
\end{figure}

\textbf{\underline{Definición:}}\\
Consideremos la cadena $d$ de una trenza situada en el cubo $\mathds{D} = \{(x,y,z) / 0 \leq x,y,z \leq 1\}$. Sea AB un segmento de dicha cadena y C un punto en el cubo de forma que el triángulo $\triangle ABC$ no corta a ninguna otra cadena de la trenza y sólo toca a la cadena $d$ en el segmento AB. Supongamos además que los segmentos AC y CB cortar a cualquier plano horizontal del cubo en un sólo punto como mucho. Podemos quedarnos con una representación poligonal de las cadenas. Visualizamos estas condiciones en la primera imagen de la figura \ref{elem}.\\
Bajo estas condiciones definimos un \textbf{movimiento elemental} como la operación $ \Omega $ que intercambia el segmento AB por los segmentos AC $ \cup $ CB.\\

La operación inversa $ \Omega^{-1} $ que intercambia los segmentos AC $\cup$  CB, que formen parte de una cadena, por el segmento AB de forma que el triángulo $\triangle ABC$ no corte a ninguna otra cadena, también es considerada un movimiento elemental. \\
Podemos ver la representación de ambos movimientos en la figura \ref{elem}.\\
\begin{figure}[h!]
	\centering
	\includegraphics[width=6.5cm]{itrenzas/elemental.png}
	\caption{}
	\label{elem} 
\end{figure}


\textbf{\underline{Definición 2.1:}}\label{defequi}\\
Sean dos trenzas $\beta$, $\beta'$. Diremos que son \textbf{equivalentes} ($\beta \sim \beta'$) si existe una cadena finita de trenzas $ \beta = \beta_{0}$, $\beta_{1},...,\beta_{m}=\beta'$ tal que cada par de trenzas $ \beta_{i}, \beta_{i+1}, i=0,..,m-1, $ está relacionado por un movimiento elemental. A esta cadena de trenzas equivalentes la representaremos del siguiente modo:
\begin{center}
	$ \beta = \beta_{0} \rightarrow \beta_{1} \rightarrow ... \rightarrow \beta_{m}=\beta'$
\end{center}

Si dos trenzas $\beta$, $\beta'$ no son equivalentes, lo denotaremos como $\beta \not \sim \beta'$.\\

Denotaremos como $ \textbf{B}_{n} $ al conjunto de todas las trenzas de n cadenas no equivalentes entre sí. Es decir, ${B}_{n}$ = $\mathscr{B}_{n}$/$ \sim $.\\