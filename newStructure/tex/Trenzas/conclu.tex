\newpage
\section{Conclusiones}
Para estudiar si dos nudos dados son o no equivalentes, tratamos de estudiar si las palabras que representan a las trenzas cerradas que generan dichos nudos son equivalentes. \\

El método que acabamos de ver (al que nos referiremos como método de Dehornoy) nos permite confirmar si dos palabras dadas representan a una misma trenza, y por tanto los nudos que generan sus trenzas cerradas serán equivalentes.\\

El problema es que podemos tener dos palabras de trenzas que no sean equivalentes, pero puede ser que sus trenzas cerradas sí que sean equivalentes dando lugar a nudos equivalentes. En este caso, podremos estudiar el polinomio de Alexander para comprobar si las dos palabras dadas generan trenzas cerradas que no son equivalentes o que sí podrían serlo (que no quiere decir que lo sean).\\

Por tanto, nos queda una serie de palabras que generan trenzas cerradas que no sabremos si son equivalentes entre sí. Se trata de un problema que sigue abierto. \\

El uso de invariantes como el exponente o la permutación de una palabra nos van a permitir determinar si dos palabras dadas representan a trenzas que no son equivalentes de una forma más rápida que usando método de Dehornoy, pero con el uso de estos invariantes no podremos confirmar que dos palabras sean equivalentes.\\ 

