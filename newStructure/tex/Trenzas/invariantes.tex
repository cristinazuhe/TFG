\bigskip
\section{Algunos invariantes.}
Ya conocemos cuáles son los movimientos básicos que nos permiten determinar si dos trenzas dadas son equivalentes. Pero al igual que ocurría con los nudos, llevar esta idea a la práctica no es factible.\\

En esta sección vamos a ver algunos invariantes de trenzas que nos permitirán determinar de forma relativamente rápida si dos trenzas no son equivalentes.\\

\underline{\textbf{Definición:}} \\
Un \textbf{invariante} de una trenza es una propiedad que no cambia cuando la trenza sufre deformaciones en el espacio. \\

Vamos a ver algunos invariantes de trenzas que usaremos posteriormente en la práctica.

\bigskip
\subsection{Exponente:}\label{invtren1}
\textbf{\underline{Definición:}}\\
Sea $\beta \in B_{n}$ representada como $\beta $= $\sigma_{i_{1}}^{e_{1}} \sigma_{i_{2}}^{e_{2}} ... \sigma_{i_{m}}^{e_{m}}$ donde $e_{i} \in \{-1,1\}$, 1 $\le i_{1}, i_{2},..,i_{m} \le$ n-1.\\
Llamamos \textbf{exponente} de $\beta$ al entero $ \sum_{i=1}^{m} e_{i} $. Se denotará como exp($\beta$).\\

El exponente de la trenza $\beta = \sigma3\sigma2^{-1}\sigma3^{-1}$ de la figura \ref{exp1} es $+1-1-1=-1$.\\
   \begin{figure}[h!]
   	\centering
   	\includegraphics[width=4.3cm]{itrenzas/4c3.png}
   	\caption{$\sigma3\sigma2^{-1}\sigma3^{-1}$}
   	\label{exp1} 
   \end{figure}

\begin{pro}
	El exponente de una trenza $\beta \in B_{n}$ es un invariante. 
	\begin{proof}
		Veamos que dadas las trenzas $\beta1,\beta2 \in B_{n}$ tales que $\beta1 \sim \beta2$, se verifica que exp($\beta1$)=exp($\beta2$).\\
		
		Al ser $\beta1 \sim \beta2$, por la definición \ref{defequi}, sabemos que existe la secuencia de trenzas equivalentes tal que: 
		\begin{center}
			$ \beta1 = \beta1_{0} \rightarrow \beta1_{1} \rightarrow ... \rightarrow \beta1_{m}=\beta2$ 
		\end{center}
	    donde cada par de trenzas $ \beta_{j}, \beta_{j+1}, j=0,..,m-1, $ está relacionado por un movimiento elemental. Estos movimientos elementales se pueden reducir a las relaciones las  \ref{rel1} y \ref{rel2} que definen al grupo $ B_{n} $ . Por tanto,para verificar que el exponente de cada par $ \beta_{j}, \beta_{j+1}$ es igual, tenemos que ver que estas relaciones de igualdad tienen el mismo exponente:
	    
	    \begin{enumerate}
	    \item $\sigma_{i+1}\sigma_{i}\sigma_{i+1} =\sigma_{i}\sigma_{i+1}\sigma_{i} $ siendo $i=2,..,n-2 $:\\
	    exp($ \sigma_{i+1}\sigma_{i}\sigma_{i+1})$ = exp$(\sigma_{i}\sigma_{i+1}\sigma_{i} $).
	    \item $\sigma_{i}\sigma_{j}=\sigma_{j}\sigma_{i}$ siendo $1 \le i < j \le n-1 $, $j-i \geq 2$:\\
	    exp($\sigma_{i}\sigma_{j})$ = exp$(\sigma_{j}\sigma_{i}$).  	
	    \end{enumerate}
	\end{proof}
\end{pro}

Haciendo uso de este invariante podremos ver muy fácilmente si dos trenzas dadas tienen distintos exponente y por tanto no son equivalentes. Podemos ver un ejemplo de dos trenzas no equivalentes en la figura \ref{exp2}. La trenza $\beta1 = \sigma2\sigma1\sigma2^{-1}$ tiene exponente +1, mientras que la trenza $\beta1 = \sigma2^{-1}\sigma1^{-1}\sigma2$ tiene exponente -1.\\ 
	\begin{figure}[h!]
		\centering
		\subfigure[exp($ \beta1 $) = +1]{\includegraphics[width=3.5cm]{itrenzas/6c1.png}}
		\subfigure[exp($ \beta2 $) = -1]{\includegraphics[width=3.5cm]{itrenzas/6c4.png}}
		\caption{Trenzas no equivalentes.}
		\label{exp2} 
	\end{figure}
	
Si tuviesen igual exponente tendríamos que estudiar otros invariantes para ver si las trenzas son iguales. Por ejemplo, en la figura \ref{exp3} vemos que las trenzas $\beta1 = \sigma2\sigma2\sigma3^{-1}$ y $\beta2 = \sigma3^{-1}\sigma2\sigma1$ tienen el mismo exponente (+1) luego no podremos saber si son o no equivalentes. Al estudiar el invariante de la seguiente sección veremos que no lo son.  \\
	\begin{figure}[h!]
		\centering
		\subfigure[exp($ \beta1 $) = +1]{\includegraphics[width=3.5cm]{itrenzas/7c1.png}}
		\subfigure[exp($ \beta2 $) = +1]{\includegraphics[width=3.5cm]{itrenzas/7c2.png}}
		\caption{Trenzas con mismo exponente.}
		\label{exp3} 
	\end{figure}

\bigskip
\subsection{Permutación:}\label{invtren2}
\textbf{\underline{Definición:}}\\
Sea $\beta \in B_{n}$. Consideramos la cadena i que conecta el punto inicial $ A_{i}$ con el punto final $B_{j_{i}}$. Llamaremos \textbf{permutación} asociada a la trenza a la matriz:\\
\[\pi(\beta)=\begin{bmatrix}
1 & 2 & .. & n\\
j_{1} & j_{2} & .. & j_{n} \\
\end{bmatrix}\]
Por simplicidad podremos denotar la permutación de la trenza como el vector $j_{1} j_{2} ... j_{n}$.\\

La permutación de la trenza $\beta = \sigma3\sigma2^{-1}\sigma3^{-1}$ de la figura \ref{perm1} es \[\pi(\beta)=\begin{bmatrix}
1 & 2 & 3 & 4\\
1 & 4 & 3 & 2 \\
\end{bmatrix}\] o bien directamente la permutación: 1 4 3 2.\\
\begin{figure}[h!]
	\centering
	\includegraphics[width=5.5cm]{itrenzas/4c3.png}
	\caption{$\sigma3\sigma2^{-1}\sigma3^{-1}$}
	\label{perm1} 
\end{figure}

\begin{pro}
    La permutación de una trenza $\beta \in B_{n}$ es un invariante. 
    \begin{proof}
    	Veamos que dadas las trenzas $\beta1,\beta2 \in B_{n}$ tales que $\beta1 \sim \beta2$, se verifica que $\pi(\beta1$)=$\pi(\beta2$).\\
    	
    	Por la definición \ref{defequi}, existirá la secuencia de trenzas equivalentes tal que: 
    	\begin{center}
    		$ \beta1 = \beta1_{0} \rightarrow \beta1_{1} \rightarrow ... \rightarrow \beta1_{m}=\beta2$ 
    	\end{center}
    	donde cada par de trenzas $ \beta_{j}, \beta_{j+1}, j=0,..,m-1, $ está relacionado por un movimiento elemental. Los movimientos elementales, por definición, no permiten intercambiar los extremos a los que están sujetas las cuerdas de las trenzas. Por este motivo las permutaciones tienen que ser iguales. \\
    \end{proof}
\end{pro}

Haciendo uso de este invariante podemos ver un ejemplo de dos trenzas que no son equivalentes en la figura \ref{perm2}: la trenza $\beta1 = \sigma2\sigma2\sigma3^{-1}$ tiene permutación 1 2 4 3 mientras que la trenza $\beta2 = \sigma3^{-1}\sigma2\sigma1$ tiene permutación 2 3 4 1.\\ 
	\begin{figure}[h!]
		\centering
		\subfigure[$\pi( \beta1 $) = 1 2 4 3]{\includegraphics[width=4.4cm]{itrenzas/7c1.png}}
		\subfigure[$\pi( \beta2 $) = 2 3 4 1]{\includegraphics[width=4.8cm]{itrenzas/7c2.png}}
		\caption{Trenzas no equivalentes.}
		\label{perm2} 
	\end{figure}


\bigskip
\subsection{Polinomio de Alexander:}\label{invtren3}
En la sección \ref{alenudos} estudiamos el polinomio de Alexander como un invariante para nudos. Por el teorema \ref{teoMarkov}, podremos hacer uso de las trenzas como una base para obtener los invariantes de nudos, en concreto, para estudiar este invariante que implementaremos posteriormente. \\

Para poder obtener el polinomio de Alexander de una trenza, tendremos que ver unos conceptos previos, \cite{3}:

\textbf{\underline{Definición:}}\\
Sea la trenza $\beta $= $\sigma_{i_{1}}^{e_{1}} \sigma_{i_{2}}^{e_{2}} ... \sigma_{i_{m}}^{e_{m}}$ donde $e_{i} \in \{-1,1\}$, 1 $\le i_{1}, i_{2},..,i_{m} \le$ n-1. Podemos definir el homomorfismo
\begin{center}
	 $ \phi_{n} : B_{n}  \rightarrow  M(n,\mathds{Z}[t,t^{-1}])$	 
\end{center}
\[ \phi_{n} ( \sigma_{i}) = \begin{bmatrix}
I_{i-1} &  &  & \\
 & 1-t & t &  \\
 & 1 & 0 &  \\
 &  &  & I_{n-i-1} \\
\end{bmatrix}\]
donde $ I_{i} $ representa a la matriz $ ixi $ identidad y los espacios en blanco representan matrices nulas. LLamaremos a esta representación como \textbf{representación de Burau}.\\

En consecuencia tenemos 
	\[ \phi_{n} ( \sigma_{i}^{-1}) = \phi_{n} ( \sigma_{i})^{-1}= \begin{bmatrix}
	I_{i-1} &  &  & \\
	& 0 & 1 &  \\
	& t^{-1} & 1-t^{-1} &  \\	
	&  &  & I_{n-i-1} \\
	\end{bmatrix}\]


\bigskip
Veamos la matriz de Burau de la trenza $\beta \in B_{4}$ representada como $\beta = \sigma2\sigma3^{-1}$.\\
Matriz Burau de $ \sigma2 $:
	\[ \phi_{4} ( \sigma2) = \begin{bmatrix}
	1 & 0 & 0 & 0 \\
	0 & 1-t & t & 0  \\
	0 & 1 & 0 & 0 \\
	0 & 0 & 0 & 1 \\
	\end{bmatrix}\]

Matriz Burau de $ \sigma3^{-1} $:
\[ \phi_{4} ( \sigma3^{-1}) = \begin{bmatrix}
	1 & 0 & 0 & 0 \\
	0 & 1 & 0 & 0 \\
	0 & 0 & 0 & 1  \\	
	0 & 0 & t^{-1} & 1-t^{-1} \\
\end{bmatrix}\]
 
 Por tanto, la matriz Burau de $\beta = \sigma2\sigma3^{-1}$ es:
 \[ \phi_{4} (\sigma2\sigma3^{-1}) = \phi_{4} (\sigma2) \phi_{4}(\sigma3^{-1}) = \begin{bmatrix}
 1 & 0 & 0 & 0 \\
 0 & 1-t & 0 & t \\
 0 & 1 & 0 & 0  \\	
 0 & 0 & t^{-1} & 1-t^{-1} \\
 \end{bmatrix}\]\\
 
 
\begin{teo}\label{teoalex}
	Sea la trenza $\beta \in B_{n}$ con representación de Burau $\phi_{n}(\beta)$. Supongamos que su trenza cerrada genera al nudo K.\\
    Entonces $\exists k \in \mathds{Z}$ tal que el polinomio ($ \pm $ $ t^{k} )det[\phi_{n}(\beta)$ - $ I_{n} $$ ]_{1,1} $ es un invariante de K. A este polinomio, conocido como polinomio de Alexander, se le denota como $ \triangle_{k} $(t).
\end{teo}

Nota: Sea la matriz cuadrada A de dimensión n y A' la matriz de dimensión n-1 que se obtiene al suprimir en la matriz A la fila i y la columna i. 
Se tiene $ det[A]_{i,i}=det[A'] $.\\ 

Gracias a este teorema podemos afirmar que si dos trenzas $\beta1 \in B_{n}, \beta2 \in B_{m}$ cerradas generan el mismo nudo entonces se verificará la igualdad:\\
det[$\phi_{n}(\beta1)$ - $ I_{n} $$ ]_{1,1}$ = $ \pm t^{k} $det[$\phi_{m}(\beta2)$ - $ I_{m} $$ ]_{1,1}$.\\

Para ver la demostración necesitamos un lema previo y los siguientes conceptos:\\

\newpage
Sea $\phi_{n}(\beta)$ = $M$ = $||a_{ij}||$ $1 \le i,j \le n$, donde $\beta \in B_{n}$. Definimos las siguientes matrices de dimensión nxn:\\
  \[ S = \begin{bmatrix}
  1 & 1 & 1 &... & 1 & 1 \\
  0 & 1 & 1 &... & 1 & 1\\
  0 & 0 & 1 &... & 1 & 1 \\
  ... & ... & ...& ... & 1 & 1 \\	
  0 & 0 & 0 & ... & 1 & 1\\	
  0 & 0 & 0 & ... & 0 & 1\\
  \end{bmatrix} 
  S^{-1} = \begin{bmatrix}
  1 & -1 & 0 & ... & 0 & 0 \\
  0 & 1 & -1 & ... & 0 & 0\\
  0 & 0 & 1 & ... & 0  & 0\\
  ... & ... & ...& ... & 0 & 0  \\	
  0 & 0 & 0 &...& 1 & -1  \\	
  0 & 0 & 0 & ... & 0 & 1 \\
  \end{bmatrix}\]\\
  
 Podemos considerar el producto de matrices $S^{-1}MS$
   \[ M S = \begin{bmatrix}
   a_{1,1} & a_{1,2} & a_{1,3} &... & a_{1,n} \\
   a_{2,1} & a_{2,2} & a_{2,3} &...  & a_{2,n}\\
   ... & ... & ...& ... & ... \\		
   a_{n,1} & a_{n,2} & a_{n,3} & ... & a_{n,n}\\
   \end{bmatrix} 
   \begin{bmatrix}
   1 & 1 & 1 &... & 1  \\
   0 & 1 & 1 &... & 1 \\
   ... & ... & ...& ... & ... \\	
   0 & 0 & 0 & ... & 1\\
   \end{bmatrix}\]
   
   \[= \begin{bmatrix}
   a_{1,1} & a_{1,1}+a_{1,2} & ... & a_{1,1} + a_{1,2}+...+a_{1,n} = 1  \\
   a_{2,1} & a_{2,1}+a_{2,2} &... & a_{2,1} + a_{2,2}+...+a_{2,n} = 1  \\
   ... & ... & ... & ... \\	
   a_{n,1} & a_{n,1}+a_{n,2} & ... & a_{n,1} + a_{n,2}+...+a_{n,n} = 1 \\
   \end{bmatrix}\]\\
  
  Luego la matriz  $S^{-1}MS$ tendrá la siguiente forma:\\
     \[ S^{-1}MS = \begin{bmatrix}
     1 & -1 & 0 &... & 0 \\
     0 & 1 & -1 &...  & 0\\
     ... & ... & ...& ... & ... \\		
     0 & 0 & 0 & ... & 1\\
     \end{bmatrix} 
     \begin{bmatrix}
     a_{1,1} & a_{1,1}+a_{1,2} & ... & a_{1,1} + a_{1,2}+...+a_{1,n} = 1  \\
     a_{2,1} & a_{2,1}+a_{2,2} &... & a_{2,1} + a_{2,2}+...+a_{2,n} = 1  \\
     ... & ... & ... & ... \\	
     a_{n,1} & a_{n,1}+a_{n,2} & ... & a_{n,1} + a_{n,2}+...+a_{n,n} = 1 \\
     \end{bmatrix}\]
     
     \[= \begin{bmatrix}
     a_{1,1}-a_{2,1} & a_{1,1}+a_{1,2}-a_{2,1}-a_{2,2} & ... & 1-1 = 0  \\
     a_{2,1}-a_{3,1} & a_{2,1}+a_{2,2}-a_{3,1}-a_{3,2}&... & 1 - 1=0  \\
     ... & ... & ... & ... \\	
     a_{n,1} & a_{n,1}+a_{n,2} & ... & 1 \\
     \end{bmatrix}\]\\
     
     De un modo más simple, vamos a denotar a dicha matriz del siguiente modo:\\
     \[ S^{-1}MS = \left[\begin{array}{r|r}
     \Lambda (t) & 0 \\ \hline	
     * ... * & 1\\
     \end{array}\right]\]\\
  Nota:  Es claro que $ det[S^{-1}(M-I_{n})S]_{n,n} = det(\Lambda(t)-I_{n-1}) $.\\
  
  \newpage
 \begin{lem}
 	Se verifica la siguiente igualdad:
	 \begin{center}
 		det $(\Lambda(t) - I_{n-1})$ = (1+t+..+$ t^{n-1} $) det$[ M - I_{n}] _{1,1} $
	 \end{center}
	 \begin{proof}
	 	Consideramos la matriz de dimensión (n-1)x(n-1):
	            \begin{center}
	            	$ \Lambda(t) - I_{n-1}= $ 
	            \end{center}	           
	           \[\begin{bmatrix}
	           a_{1,1}-a_{2,1}-1 & a_{1,1}+a_{1,2}-a_{2,1}-a_{2,2} & ... &  a_{1,1}+...+a_{1,n-1}-a_{2,1}-...-a_{2,n-1} \\
	           a_{2,1}-a_{3,1} & a_{2,1}+a_{2,2}-a_{3,1}-a_{3,2}-1&... & a_{2,1}+...+a_{2,n-1}-a_{3,1}-...-a_{3,n-1}  \\
	           ... & ... & ... & ... \\	
	           a_{n-1,1}-a_{n,1} & a_{n-1,1}+a_{n-1,2}-a_{n,1}-a_{n,2} & ... & a_{n-1,1}+...+a_{n-1,n-1}-a_{n,1}-...-a_{n,n-1} \\
	           \end{bmatrix}\]\\
	           
	           Es claro que det $(\Lambda(t) - I_{n-1})$ = det$(S^{T}(\Lambda(t) - I_{n-1})S^{-1})$ donde $S^{T}$ es la matriz traspuesta de dimensión (n-1)x(n-1) de la matriz $S$. Por tanto, vamos a trabajar con la matriz $(S^{T}(\Lambda(t) - I_{n-1})S^{-1})$:
	           
	           \[ (\Lambda(t) - I_{n-1})S^{-1} = (\Lambda(t) - I_{n-1}) \begin{bmatrix}
	           1 & -1 & 0 &... & 0 \\
	           0 & 1 & -1 &...  & 0\\
	           ... & ... & ...& ... & ... \\		
	           0 & 0 & 0 & ... & 1\\
	           \end{bmatrix}=\]
	                
	           \[= \begin{bmatrix}
	           a_{1,1}-a_{2,1}-1 & a_{1,2}-a_{2,2}+1 & ... & a_{1,n-1}-a_{2,n-1}  \\
	           a_{2,1}-a_{3,1} & a_{2,2}-a_{3,2}-1&... &  a_{2,n-1}-a_{3,n-1} \\
	           ... & ... & ... & ... \\	
	           a_{n-1,1}-a_{n,1} & a_{n-1,2}-a_{n,2}&... &  a_{n-1,n-1}-a_{n,n-1} \\
	           \end{bmatrix}\]\\
	           
	           Luego:	          
	           \begin{center}
	            	$  S^{T}(\Lambda(t) - I_{n-1})S^{-1} =$
	           \end{center}
	           \[ = \begin{bmatrix}
	           1 & 0 & 0 &... & 0 \\
	           1 & 1 & 0 &...  & 0\\
	           ... & ... & ...& ... & ... \\		
	           1 & 1 & 1 & ... & 1\\
	           \end{bmatrix}
	           \begin{bmatrix}
	           a_{1,1}-a_{2,1}-1 & a_{1,2}-a_{2,2}+1 & ... & a_{1,n-1}-a_{2,n-1}  \\
	           a_{2,1}-a_{3,1} & a_{2,2}-a_{3,2}-1&... &  a_{2,n-1}-a_{3,n-1} \\
	           ... & ... & ... & ... \\	
	           a_{n-1,1}-a_{n,1} & a_{n-1,2}-a_{n,2}&... &  a_{n-1,n-1}-a_{n,n-1} \\
	           \end{bmatrix}=\]\\
	           
	           \[ = \begin{bmatrix}
	           a_{1,1}-a_{2,1}-1 & a_{1,2}-a_{2,2}+1 & ... & a_{1,n-1}-a_{2,n-1}  \\
	           a_{1,1}-a_{3,1}-1 & a_{1,2}-a_{3,2}&... &  a_{1,n-1}-a_{3,n-1} \\
	           ... & ... & ... & ... \\	
	           a_{1,1}-a_{n,1}-1 & a_{1,2}-a_{n,2}&... &  a_{1,n-1}-a_{n,n-1} \\
	           \end{bmatrix} =
	           \begin{bmatrix}
	           A_{1} - A_{2} \\
	           A_{1} - A_{3} \\
	           ... \\	
	           A_{1} - A_{n} \\
	           \end{bmatrix}\]\\
	           
	           Donde los vectores $A_{i} $ $ 1 \le i \le n $ son las filas de la matriz $[M-I_{n}]_{0,n}$, es decir, tienen la siguiente forma:
			   \begin{center}
		           $ A_{1} = [a_{1,1}-1, a_{1,2},...,a_{1,n-1}] $\\
	               $ A_{2} = [a_{2,1}, a_{2,2}-1,...,a_{2,n-1}] $\\
	               ...\\
	               $ A_{n} = [a_{n,1}, a_{n,2},...,a_{n,n-1}] $\\
		       \end{center}
	           
	           Por tanto det $(\Lambda(t) - I_{n-1})$ = det$(S^{T}(\Lambda(t) - I_{n-1})S^{-1})$ = 
	           \[=det\begin{bmatrix}
	           A_{1} - A_{2} \\
	           A_{1} - A_{3} \\
	           ... \\	
	           A_{1} - A_{n} \\
	           \end{bmatrix}=\]
	           \[=det\begin{bmatrix}
	           A_{1} \\
	           A_{1} - A_{3} \\
	           ... \\	
	           A_{1} - A_{n} \\
	           \end{bmatrix} + det\begin{bmatrix}
	           - A_{2} \\
	           A_{1} - A_{3} \\
	           ... \\	
	           A_{1} - A_{n} \\
	           \end{bmatrix} =det\begin{bmatrix}
	           A_{1} \\
	           - A_{3} \\
	           ... \\	
	           - A_{n} \\
	           \end{bmatrix} + det\begin{bmatrix}
	           - A_{2} \\
	           A_{1} - A_{3} \\
	           ... \\	
	           A_{1} - A_{n} \\
	           \end{bmatrix}=...=\]
	           
	           \[=det\begin{bmatrix}
	           A_{1}  \\
	           - A_{3} \\
	           ... \\	
	           - A_{n} \\
	           \end{bmatrix}+det\begin{bmatrix}
	           -A_{2}  \\
	           A_{1} \\
	           ... \\	
	           - A_{n} \\
	           \end{bmatrix}+...+det\begin{bmatrix}
	           -A_{2}  \\
	           -A_{3} \\
	           ... \\
	           -A_{k}\\
	           A_{1}\\
	           -A_{k+2}
	           ...\\	
	           - A_{n} \\
	           \end{bmatrix}+...+det\begin{bmatrix}
	           -A_{2}  \\
	           - A_{3} \\
	           ... \\	
	           - A_{n-1} \\
	           A_{1}\\
	           \end{bmatrix}+det\begin{bmatrix}
	           - A_{2}  \\
	           - A_{3} \\
	           ... \\	
	           - A_{n} \\
	           \end{bmatrix}\]
	           
	           Para obtener el valor de estos determinantes vamos a hacer uso de la siguiente igualdad:
	           \begin{center}
	           	$ det[M-I_{n}]_{p,q} = (-1)^{p-q}t^{p-1}det[M-I_{n}]_{1,1}	 $ $1\le p,q \le n$  
	           \end{center}    
	           
	           De modo que tomando $ p=k+1 $ y $ q=n $, se tiene:     
	           
	           \[det\begin{bmatrix}
	           -A_{2}  \\
	           -A_{3} \\
	           ... \\
	           -A_{k}\\
	           A_{1}\\
	           -A_{k+2}
	           ...\\	
	           - A_{n} \\
	           \end{bmatrix}=(-1)^{n-k-1}det[M-I_{n}]_{k+1,n} = t^{k}det[M-I_{n}]_{1,1}\]
	           
	           Por tanto, det $(\Lambda(t) - I_{n-1})$ = (1+t+..+$ t^{n-1} $) det$[ M - I_{n}] _{1,1} $.
	 \end{proof}
 \end{lem}
 

 \begin{cor}\label{corlem}
 		Se verifica la siguiente igualdad:
 		\begin{center}
 			$ det[S^{-1}(M-I_{n})S]_{n,n} = det(\Lambda(t)-I_{n-1}) $ = (1+t+..+$ t^{n-1} $) det$[ M - I_{n}] _{1,1} $
 		\end{center}
 \end{cor}
 
 \bigskip
 Ya estamos en condiciones de demostrar el teorema \ref{teoalex}:
 \begin{proof}
 	Por el teorema \ref{teoMarkov} sabemos que es suficiente con probar las siguientes igualdades, donde $ \gamma, \beta \in B_{n} $ :
 	\begin{itemize}
 		\item Movimiento elemental M1: \\
 		$ det[\phi_{n}(+\sigma_{i}-\sigma_{i}) - I_{n}]_{1,1} = det[\phi_{n}(-\sigma_{i}+\sigma_{i}) - I_{n}]_{1,1}$, siendo $i<n$
 		\item Movimiento elemental M2: \\
 		$ det[\phi_{n}(\sigma_{i}\sigma_{i+1}\sigma_{i}) - I_{n}]_{1,1} = det[\phi_{n}(\sigma_{i+1}\sigma_{i}\sigma_{i+1}) - I_{n}]_{1,1}$, siendo $ i+1<n $.
 		\item Movimiento elemental M3:\\
 		$ det[\phi_{n}(\sigma_{i}\sigma_{j}) - I_{n}]_{1,1} = det[\phi_{n}(\sigma_{j}\sigma_{i}) - I_{n}]_{1,1}$, siendo $i<n, |i-j| > 1$
 		\item Verificando conjugación Mv1: $ det[\phi_{n}(\gamma\beta\gamma^{-1}) - I_{n}]_{1,1} = det[\phi_{n}(\beta) - I_{n}]_{1,1}$ 
 		\item Verificando estabilización Mv2: $ det[\phi_{n+1}(\beta\sigma_{n}) - I_{n+1}]_{1,1} = det[\phi_{n}(\beta) - I_{n}]_{1,1}$   
  	\end{itemize}
	 Para demostrar las igualdades referentes a los movimientos elementales basta con probar las siguientes igualdades:
	 \begin{itemize}
	 	 \item Para el movimiento elemental M1: \\
	 	 $ \phi_{n}(+\sigma_{i}-\sigma_{i}) = \phi_{n}(-\sigma_{i}+\sigma_{i})$, siendo $i<n$.\\
	 	 Esta igualdad es clara pues por definición $-\sigma_{i}$ es la matriz inversa de $\sigma_{i}$  luego $ \phi_{n}(+\sigma_{i}-\sigma_{i}) = \phi_{n}(+\sigma_{i})\phi_n(-\sigma_{i}) = \phi_{n}(-\sigma_{i})\phi_n(+\sigma_{i}) = \phi_{n}(-\sigma_{i}+\sigma_{i})$
	 	 
	 	 \item Para el movimiento elemental M2: \\
	  	 $ \phi_{n}(\sigma_{i}\sigma_{i+1}\sigma_{i}) = \phi_{n}(\sigma_{i+1}\sigma_{i}\sigma_{i+1})$, siendo $ i+1<n $.\\
	  	 Sin pérdida de generalidad podemos suponer $i=1, n=3$ de modo que:
	  	 	\[ \phi_{3}(\sigma_{1}\sigma_{2}\sigma_{1}) = 
	  	 	\begin{bmatrix}
	  	 	  1-t & t & 0  \\
	  	 	  1 & 0 & 0 \\
	  	 	  0 & 0 & 1 \\
	  	 	\end{bmatrix}\begin{bmatrix}
	  	 	1 & 0 & 0 \\
	  	 	0 & 1-t & t \\
	  	 	0 & 1 & 0 \\
	  	 	\end{bmatrix}\begin{bmatrix}
	  	 	1-t & t & 0  \\
	  	 	1 & 0 & 0 \\
	  	 	0 & 0 & 1 \\
	  	 	\end{bmatrix}=\]
	  	 	
	  	 	\[ = 
	  	 	\begin{bmatrix}
	  	 	1-t & t & 0  \\
	  	 	1 & 0 & 0 \\
	  	 	0 & 0 & 1 \\
	  	 	\end{bmatrix}\begin{bmatrix}
	  	 	1-t & t & 0  \\
	  	 	1-t & 0 & t \\
	  	 	1 & 0 & 0 \\
	  	 	\end{bmatrix}=
	  	 	\begin{bmatrix}
	  	 	(1-t)^{2}+(1-t)t = 1-t & (1-t)t & t^{2}  \\
	  	 	1-t & t & 0 \\
	  	 	1 & 0 & 0 \\
	  	 	\end{bmatrix}=\]
	  	 	
	  	 	\[\begin{bmatrix}
	  	 	1-t& (1-t)t & t^{2}  \\
	  	 	1-t & t & 0 \\
	  	 	1 & 0 & 0 \\
	  	 	\end{bmatrix} = 
	  	 	\begin{bmatrix}
	  	 	1 & 0 & 0 \\
	  	 	0 & 1-t & t \\
	  	 	0 & 1 & 0 \\
	  	 	\end{bmatrix}\begin{bmatrix}
	  	 	1-t & (1-t)t & t^{2}  \\
	  	 	1 & 0 & 0 \\
	  	 	0 & 1 & 0 \\
	  	 	\end{bmatrix}= \]
	  	 	\[\begin{bmatrix}
	  	 	1 & 0 & 0 \\
	  	 	0 & 1-t & t \\
	  	 	0 & 1 & 0 \\
	  	 	\end{bmatrix}\begin{bmatrix}
	  	 	1-t & t & 0  \\
	  	 	1 & 0 & 0 \\
	  	 	0 & 0 & 1 \\
	  	 	\end{bmatrix}\begin{bmatrix}
	  	 	1 & 0 & 0 \\
	  	 	0 & 1-t & t \\
	  	 	0 & 1 & 0 \\
	  	 	\end{bmatrix}= \phi_{3}(\sigma_{2}\sigma_{1}\sigma_{2})\]
	  	 	
	  	 
	 	 \item Para el movimiento elemental M3:\\
	 	 $ \phi_{n}(\sigma_{i}\sigma_{j}) = \phi_{n}(\sigma_{j}\sigma_{i})$, siendo $i<n, |i-j| > 1$\\
	 	 Sin pérdida de generalidad podemos suponer $ i+1<j $ de modo que:
	 	 \[ \phi_{n} (\sigma_{i}) \phi_{n} (\sigma_{j}) = \begin{bmatrix}
	 	 I_{i-1} &  &  & \\
	 	 & 1-t & t &  \\
	 	 & 1 & 0 &  \\
	 	 &  &  & I_{j-i-2} \\
	 	 & & & & 1-t & t &  \\
	 	 & & & & 1 & 0 &  \\
	 	 & & & & & & I_{n-j-1}  \\
	 	 \end{bmatrix}= \phi_{n} (\sigma_{j}) \phi_{n} (\sigma_{i})\]
 	 \end{itemize}
 	 
 	 Veamos ahora que se verifican las igualdades referentes a los movimientos de Markov.
 	 \begin{itemize}
 	 	\item 
 	 Veamos que se verifica la igualdad para el primer movimiento de Markov, es decir, veamos que se verifica $ det[\phi_{n}(\gamma\beta\gamma^{-1}) - I_{n}]_{1,1} = det[\phi_{n}(\beta) - I_{n}]_{1,1}$:\\
 	 Consideramos las matrices $ S $ y $ S^{-1} $ que definimos anteriormente y definimos los productos de matrices     
 	 \[S^{-1}\phi_{n}(\gamma)S = \left[\begin{array}{r|r}
 	 \Lambda (\gamma) & 0 \\ \hline	
 	 * ... * & 1\\
 	 \end{array}\right];
 	 S^{-1}\phi_{n}(\beta)S = \left[\begin{array}{r|r}
 	 \Lambda (\beta) & 0 \\ \hline	
 	 * ... * & 1\\
 	 \end{array}\right];
 	 S^{-1}\phi_{n}(\gamma^{-1})S = \left[\begin{array}{r|r}
 	 \Lambda (\gamma)^{-1} & 0 \\ \hline	
 	 * ... * & 1\\
 	 \end{array}\right]\]\\
 	 
 	 De modo que  	 
 	 \[ S^{-1}\phi_{n}(\gamma\beta\gamma^{-1})S = \left[\begin{array}{r|r}
 	 \Lambda (\gamma)\Lambda (\beta)\Lambda (\gamma)^{-1} & 0 \\ \hline	
 	 * ...................... * & 1\\
 	 \end{array}\right]\]\\
	 Por el corolario \ref{corlem} sabemos que \\
	 $ (1+t+..+ t^{n-1} ) det[ \phi_{n}(\gamma\beta\gamma^{-1}) - I_{n}] _{1,1} = det[S^{-1}(\phi_{n}(\gamma\beta\gamma^{-1}))S-I_{n}]_{n,n}$\\
	 
	 Desarrollando este segundo término tenemos:
	 \[det[S^{-1}(\phi_{n}(\gamma\beta\gamma^{-1}))S-I_{n}]_{n,n} = det[\Lambda (\gamma)\Lambda (\beta)\Lambda (\gamma)^{-1}-I_{n-1}] =\]
	 \[
	 det[\Lambda (\gamma)(\Lambda (\beta)-I_{n-1})\Lambda (\gamma)^{-1}] =
	 det[\Lambda (\beta)-I_{n-1}] \]
	 
	 Aplicando de nuevo el corolario \ref{corlem} tenemos la siguiente igualdad\\
	 $det[\Lambda (\beta)-I_{n-1}]  = (1+t+..+ t^{n-1} ) det[ \phi_{n}(\beta) - I_{n}] _{1,1} $\\
	 
	 Luego obtenemos la igualdad:\\
	  $ (1+t+..+ t^{n-1} ) det[ \phi_{n}(\gamma\beta\gamma^{-1}) - I_{n}] _{1,1} = (1+t+..+ t^{n-1} ) det[ \phi_{n}(\beta) - I_{n}] _{1,1} $\\
	  
	  y podemos concluir
	  $ det[ \phi_{n}(\gamma\beta\gamma^{-1}) - I_{n}] _{1,1} = det[ \phi_{n}(\beta) - I_{n}] _{1,1} $\\
	  
	  \item 
	  Por último vamos a ver que se verifica la igualdad para el segundo movimiento de Markov, es decir, veamos que se verifica $ det[\phi_{n+1}(\beta\sigma_{n}) - I_{n+1}]_{1,1} = det[\phi_{n}(\beta) - I_{n}]_{1,1}$ .\\
	  
	  Llamemos $ M = \phi_{n}(\beta) $ de modo que 
 	 \[ \phi_{n+1}(\beta) = \left[\begin{array}{r|r}
 	  M & 0 \\ \hline	
 	  0 & 1\\
 	 \end{array}\right]\]\\	  
 	 
 	 Por otra parte sabemos que 
 	 \[ \phi_{n+1} (\sigma_{n}) = \begin{bmatrix}
 	 I_{n-1} &  &  \\
 	 & 1-t & t  \\
 	 & 1 & 0  \\
 	 \end{bmatrix}\]\\
 	 
 	 Por tanto, 
 	 \[ \phi_{n+1} (\beta\sigma_{n}) = \begin{bmatrix}
 	 & & & a_{1,n}(1-t) & a_{1,n}t\\
 	 & M_{n-1,n-1} & & ... & ...\\
 	 & & & a_{n-1,n}(1-t) & a_{n-1,n}t\\
 	 a_{n,1}& ... & a_{n,n-1} & a_{n,n}(1-t) & a_{n,n}t\\
 	 & & & 1 & 0\\
 	 \end{bmatrix}\]\\
	  
	 De este modo se tiene 
 	 \[det[\phi_{n+1}(\beta\sigma_{n}) - I_{n+1}]_{n,n} = det \left[\begin{array}{r|r}
    	M_{n-1,n-1} - I_{n-1} & \\ \hline	
 	    & -1\\
 	 \end{array}\right]\] 
 	 \begin{center}
 	 	$ = -det[M_{n-1,n-1}-I_{n-1}] = -det[\phi_{n}(\beta)-I_{n}]_{n,n} $ 
 	 \end{center} 
 	 
 	 Obteniendo la igualdad
 	  	 \begin{center}
 	  	 	$ det[\phi_{n+1}(\beta\sigma_{n}) - I_{n+1}]_{1,1} = det[\phi_{n}(\beta)-I_{n}]_{1,1} $ 
 	  	 \end{center}
      \end{itemize}
 \end{proof}


De este modo, el polinomio de Alexander de la trenza $\beta1 = \sigma1$ se obtendría del siguiente modo:\\
Por definición tenemos que  
\[ \phi_{2} (\beta1) = \begin{bmatrix}
1-t & t  \\
1 & 0 \\
\end{bmatrix}\]
luego 
\[ \phi_{2} (\beta1) - I_{2}= \begin{bmatrix}
-t & t  \\
1 & -1 \\
\end{bmatrix}\]
Finalmente tenemos 
\[ det(\phi_{2} (\beta1) - I_{2})_{1,1} = det(\begin{bmatrix}
-1 \\
\end{bmatrix}) = -1\].

Veamos ahora el polinomio de Alexander de una trenza más compleja, en concreto de la trenza $\beta2 = \sigma2\sigma3^{-1}$. Ya sabemos que 
 \[ \phi_{4} (\beta2) = \begin{bmatrix}
 1 & 0 & 0 & 0 \\
 0 & 1-t & 0 & t \\
 0 & 1 & 0 & 0  \\	
 0 & 0 & t^{-1} & 1-t^{-1} \\
 \end{bmatrix}\]
 luego
  \[ \phi_{4} (\beta2) - I_{4} = \begin{bmatrix}
  0 & 0 & 0 & 0 \\
  0 & -t & 0 & t \\
  0 & 1 & -1 & 0  \\	
  0 & 0 & t^{-1} & t^{-1} \\
  \end{bmatrix}\].
  
  Finalmente tenemos 
    \[ det(\phi_{4} (\beta2) - I_{n})_{1,1} = det(\begin{bmatrix}
    -t & 0 & t \\
     1 & -1 & 0  \\	
     0 & t^{-1} & t^{-1} \\
    \end{bmatrix}) = -1+1 = 0.\].
    
Por tanto tenemos que las trenzas $\beta1$ y $\beta2$ no pueden generar nudos equivalentes pues sus polinomios de Alexander son distintos.\\
    
