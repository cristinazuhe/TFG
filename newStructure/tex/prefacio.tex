
\cleardoublepage
\thispagestyle{empty}

\begin{center}
{\large\bfseries Entendiendo la teoría de nudos\\
	 mediante la simulación y la informática gráfica.}\\
\end{center}
\begin{center}
Cristina Zuheros Montes.\\
\end{center}

%\vspace{0.7cm}
\noindent{\textbf{Palabras clave}: nudo, enlace, nudo trivial, invariante, trenza, trenza cerrada, palabra, cruce, handle, toolbox}\\

\vspace{0.7cm}
\noindent{\textbf{Resumen}}\\
En este proyecto comenzamos analizando la teoría de nudos: explicamos qué se entiende por un nudo, vemos un breve recorrido por la historia de dicha teoría mostrando sus aplicaciones más destacadas, estudiamos la composición de nudos, analizamos una de las cuestiones que más nos interesan como es la equivalencia entre nudos, mostramos algunos invariantes de nudos, las notaciones más usuales de los mismos y concluimos relacionando dicha teoría con la teoría de grafos y la teoría de trenzas.\\

Como ya hemos comentado, uno de los problemas clave que tratamos en este proyecto es la equivalencia de nudos. Encontramos solución usando invariantes de nudos, pero esta solución no es completa en el sentido de que no siempre vamos a obtener respuesta usando invariantes. Por este motivo, entre otros que iremos viendo, vamos a orientar nuestro trabajo al estudio de las trenzas, donde una relación directa entre nudos y trenzas cerradas nos permitirá plantear el problema de la equivalencia de nudos como el problema de la equivalencia de trenzas.\\ 

Abordamos el estudio general de las trenzas con la definición formal de las mismas. Vemos que el conjunto de las n-trenzas tiene estructura de grupo no abeliano, introducimos la equivalencia de trenzas y de trenzas cerradas, analizamos algunos invariantes para trenzas y trenzas cerradas y estudiamos el problema de las palabras que nos permite determinar si dos trenzas dadas serán o no equivalentes. \\

El desarrollo práctico para determinar si dos trenzas son o no equivalentes conlleva más trabajo del que puede parecer a simple vista. El análisis de una gran cantidad de trenzas se convierte en un trabajo muy tedioso y repetitivo. Es por eso que interesa desarrollar software que nos permita trabajar con trenzas. \\

Por este motivo, finalizamos presentando toxtren: se trata de un toolbox que hemos creado en Matlab para trabajar tanto con trenzas como con trenzas cerradas. Mostramos cómo podemos instalar toxtren de una forma realmente simple. Además, presentamos los pseudo algoritmos más destacados para trenzas y trenzas cerradas.\\
Por último, presentamos una guía en la que mostramos los comandos necesarios para trabajar con toxtren. \\

En conclusión, hemos estudiado dos teorías relevantes en la rama de la topología como son la teoría de nudos y la teoría de trenzas, hemos visto la relación que hay entre ambas y el impacto que tienen hoy en día en diversas áreas de conocimiento. Finalmente, hemos desarrollado toxtren, toolbox en el que implementamos todos los conceptos y algoritmos que hemos analizado en la teoría de trenzas. 

 
\cleardoublepage


\thispagestyle{empty}

\selectlanguage{USenglish}
\begin{center}
{\large\bfseries Understanding knot theory \\
	through simulation and computer graphics.}\\
\end{center}
\begin{center}
Cristina Zuheros Montes.\\
\end{center}

%\vspace{0.7cm}
\noindent{\textbf{Keywords}: knot, link, trivial knot, invariant, braid, closed braid, word, crossing, toolbox, handle.}\\

\vspace{0.7cm}
\noindent{\textbf{Abstract}}\\
Within mathematical science we find a branch, known as topology, which is in charge of the study of the properties of those objects (known as topological spaces) that, by means of continuous transformations on the objects, remain invariant. Although an exact date of the origin of the topology is not known, it is usually established in 1735 with the resolution, by Leonhard Euler, of the problem of the seven bridges of Königsberg.\\


Within topology arises the theory of knots that emerged about 200 years ago. We can say that it is a fairly recent theory and the most interesting results have been discovered in the last decades. This is one of the reasons why knot theory is very striking for mathematical researchers as well as for biologists, physicists, chemists, sailors, survival experts and a large number of scientists who find in this theory a response to their approaches. The attractiveness of this theory is that, being relatively recent, it still has many open questions on which one can work.\\






At the beginning of this project, the theory of knots is explained from a mathematical point of view. We present a formal definition of knot and link.  Straightaway, we visualize some examples of knots in the three-dimensional space and its projection.\\


Next, we see a brief but interesting tour on the history of this theory and we show some of its most outstanding applications. It may seem surprising that we are able to connect knot theory to DNA structure, but nowadays in many laboratories, it is essential to have an expert in knot theory in order to understand what has been the process that has been happened to obtain a final DNA structure.\\


Once we know the notion of knot and we have some motivation to work with them, we study the composition of knots. Also at this point, we see what is meant by a prime or compound knot, we define an oriented knot and we examine when a knot is invertible. We show some projections of prime knots having less than 8 crossings.\\


Next we are concerned with one of the most striking aspects of this theory: the equivalence of knots. We consider the basic movements, known as Reidemeister movements, that allow us to modify the projection of the knot without modifying the knot itself. We expose Reidemeister's theorem, which allows us to determine if two projections represent the same knot. However, this process is really complicated and we have no guarantee that this algorithm ends in reasonable finite time.\\


For this reason, we analyze some invariants for knots. We present a formal definition of the invariant concept and we study the number of components of a link, the crossing number of a knot, tricolorability, the unknotting number and Alexander's polynomial. In order to understand all these concepts, we apply the different invariants over some projections of knots.\\ 


Afterwards, we see some of the most common knots’s notations. In particular, we analyze Dowker's notation, Gauss's notation, and Conway's notation. This last notation, by the specific way of projecting knots, has a special interest in the study of the structure of DNA.\\


Finally, we connect this theory to other great theories such as graph theory and braid theory. We have previously commented that topology arises after the resolution of the problem of the bridges of Königsberg. It is interesting to comment the fact that the resolution of this problem lead to the graph theory. Therefore, it is not surprising that knot theory can be related to graph theory. We prove how we can construct a graph from a knot and how we could construct a knot from a graph.\\


Concerning braid theory, we justify the relationship a bit more deeply. In a relatively simple way we can get a knot from a given braid. To obtain the braid representing a knot we use the Alexander's theorem.\\








As we have previously discussed, one of the key problems that we are dealing in this project is the equivalence of knots, but with the algorithm provided in the Reidemeister's theorem we can not always obtain an answer in an acceptable time. We found a solution that uses invariants of knots, but this solution is not complete in the sense that we will not always get response using invariants. For this reason, we focus on braid theory. Having a direct relationship between knots and closed braids, we can consider the problem of knot equivalence as the problem of the equivalence of closed braids. In addition, braid theory has direct application to cryptography and fluid mechanics, among other subjects.\\








Regarding braid theory, we begin showing a formal definition of braid, the strings of a braid and a closed braid. In addition, we display some examples of braids in three-dimensional space.\\


As we do in knot theory, we study the notion of braid equivalence. Now we only give an overview of equivalence based on the elementary movements. We show a really simple example so that you can intuitively have the idea in your mind. Subsequently, we study this concept in more detail but it is important to have an initial idea to be able to obtain a projection of a braid that is in accordance with the definition that we give, which will be the easiest projection to manipulate braids. In addition, we show the notation for braids that we use defining the positive and negative crossings. We define the trivial n-braid, the word of a braid and we see an example of braid with its pertinent notation.\\










Below we study one of the most important aspect to consider in braid theory: by providing the set of all braids equivalent to each other of the product of braids we find a non-abelian group structure. We define the product of braids and we check the properties of the non-abelian group with some examples ir order to make proofs as intuitive as possible.\\

Once we have defined the group structure of the braids, we examine the notion of equivalence of braids in a more formal way. We define the group of equivalent braids by equivalence relations or basic movements. Clearly if two braids are equivalent, the braid closures will also be equivalent. The problem that we find is that if two braids are not equivalent, their closures maybe be equivalent. That is, the closed braids could be equivalent even though the respective braids are not equivalent. We study this idea through the Markov’s movements: conjugation and stabilization.\\


As we commented in the knot theory, this notion of equivalence of braids and Markov-equivalent braids is not feasible to put into practice because we would have to apply a number of possible movements that might not finish in a reasonable time. Therefore, we return to study some of the most essential invariants. In this case, we study the exponent and the permutation of a braid and Alexander's polynomial of a closed braid.\\


However, these invariants will not let us always to know if two braids are or are not equivalent to each other. The problem of finding some method to know if the words of braids are equivalent or not is known as the problem of words. In this project we study the method of Patrick Dehornoy to solve this problem. The problem of analyzing whether or not two braids are equivalent can be seen as the problem of determining whether a given braid is equivalent to the trivial braid. This idea is essential in Dehornoy's method: we can determine in a relatively simple way if a word of a given braid is equivalent to the trivial braid. At this point, we use also the notion of handle.\\








Analyzing braids by hand implies more work than it may seem to the naked eye. In addition, if we would like to analyze a large number of braids, it becomes a really tedious and repetitive work. Because of this, it is interesting to develop software that allows us to work with braids.\\








In this project, we end up presenting toxtren: this is a toolbox that we have implemented in Matlab to work with braids and closed braids. We apply an object-oriented design by creating a class for each type of object. We show how we can install toxtren in a really simple way.\\


In addition, we present the most outstanding pseudo algorithms for both classes. In particular, we present the Dehornoy algorithm for braids, the algorithm of equivalences for closed braids and braids and the algorithm that allows us to know if a braid or a closed braid given is equivalent to the trivial braid. It is important to remember that for closed braids we will not always be able to obtain a positive or negative answer for these two last algorithms because we do not have mathematical algorithms that allow it. It is an issue that remains open and captures the attention of many researchers.\\ Next, we present some algorithms that we have created to represent closed braids and braids in the three-dimensional space.\\




Finally, we present a guide in which we show the basics commands necessary to work with toxtren. We have allowed the user to introduce braids in different ways to make it as comfortable as possible.\\

\newpage
In conclusion, we have studied two relevant theories in the branch of topology such as knot theory and braid theory, we have seen the relationship between both theories and the impact that they have nowadays in different areas of knowledge. Finally, we have developed toxtren, a toolbox in which we implement all this concepts and algorithms that we have analyzed in braid theory.\\







\selectlanguage{spanish}

\chapter*{}
\thispagestyle{empty}

\noindent\rule[-1ex]{\textwidth}{2pt}\\[4.5ex]

Yo, \textbf{Cristina Zuheros Montes}, alumna de la titulación Doble Grado en
Ingeniería Informática y Matemáticas de la \textbf{Escuela Técnica Superior
	de Ingenierías Informática y de Telecomunicación} y de la \textbf{Facultad de Ciencias}
de la \textbf{Universidad de Granada}, con DNI XXXXXXXXX, autorizo la
ubicación de la siguiente copia de mi Trabajo Fin de Grado en la biblioteca de
ambos centros para que pueda ser consultada por las personas que lo deseen.

\vspace{6cm}

\noindent Fdo: Cristina Zuheros Montes.

\vspace{2cm}

\begin{flushright}
	Granada a 13 de diciembre de 2016.
\end{flushright}



\chapter*{}
\thispagestyle{empty}

\noindent\rule[-1ex]{\textwidth}{2pt}\\[4.5ex]

D.\textbf{Alejandro J. León Salas}, Profesor del Departamento de Lenguajes y Sistemas Informáticos de la Universidad de Granada.\\

D.\textbf{Antonio Martínez López}, Profesor del Departamento de Geometría y Topología de la Universidad de Granada.\\

\vspace{0.5cm}

\textbf{Informan:}

\vspace{0.5cm}

Que el presente trabajo, titulado \textit{\textbf{Entendiendo la teoría de nudos mediante la simulación y la informática gráfica}},
ha sido realizado bajo su supervisión por \textbf{Cristina Zuheros Montes}, y autoriza la defensa de dicho trabajo ante el tribunal
que corresponda.

\vspace{0.5cm}

Y para que conste, expide y firma el presente informe en Granada a 13 de diciembre de 2016.

\vspace{1cm}

\textbf{Los directores:}

\vspace{5cm}

\noindent \textbf{Alejandro J. León Salas} \hspace{7cm} \textbf{Antonio Martínez López}

\chapter*{Agradecimientos}
\thispagestyle{empty}

\vspace{0.5cm}
 
 Quiero mostrar mi agradecimiento a la Universidad de Granada y a sus profesores por la oportunidad que me han brindado para formarme simultáneamente en dos grandes áreas como son matemáticas e informática. \\
 
 A mis tutores, Antonio Martínez y Alejandro León, por la ayuda, atención y apoyo constante que me han proporcionado desde el primer momento.\\
 
 A mi familia y amigos por la comprensión y enseñanzas que me han proporcionado. 
 
