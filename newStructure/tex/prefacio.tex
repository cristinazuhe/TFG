
\cleardoublepage
\thispagestyle{empty}

\begin{center}
{\large\bfseries Entendiendo la teoría de nudos mediante la simulación y la informática gráfica.}\\
\end{center}
\begin{center}
Cristina Zuheros Montes.\\
\end{center}

%\vspace{0.7cm}
\noindent{\textbf{Palabras clave}: nudo, enlace, nudo trivial, invariante, trenza, trenza cerrada, palabra, cruce, handle, toolbox}\\

\vspace{0.7cm}
\noindent{\textbf{Resumen}}\\

EN VERDAD NO SE SI ESTO ES LO QUE QUIERO PONER COMO RESUMEN EN ESPAÑOL Y SI QUEDARÍA MEJOR EN LA INTRODUCCION.....

Dentro de la ciencia matemática nos encontramos una rama, conocida como topología, que se encarga del estudio de las propiedades de aquellos cuerpos que, mediante transformaciones continuas sobre los cuerpos, permanecen inalteradas. Aunque no se conoce una fecha exacta del origen de la topología, se suele establecer en 1735 con la resolución, por parte de Leonhard Euler, del problema de los 7 puentes de Königsberg.\\

Dentro de la topología surge la teoría de nudos que sólo tiene algo más de unos 200 años. Se puede decir que es una teoría bastante reciente y cuyos resultados más interesantes se han ido descubriendo en las últimas décadas. Es por esto que la teoría de nudos es muy llamativa tanto para investigadores matemáticos como para biólogos, físicos, químicos, marineros, expertos en supervivencia y una gran cantidad de científicos que encuentra en esta teoría una respuesta a sus planteamientos. Lo atractivo de dicha teoría es que, al ser relativamente reciente, aún tiene muchas cuestiones abiertas en las que se puede trabajar. \\

Al comienzo de este proyecto analizamos la teoría de nudos: explicamos qué se entiende por un nudo, vemos un breve recorrido por la historia de los nudos viendo sus aplicaciones más destacadas, estudiamos la composición de nudos, analizamos una de las cuestiones que más nos interesan como es la equivalencia entre nudos, mostramos algunos invariantes de nudos, las notaciones más usuales de los mismos y concluimos relacionando dicha teoría con la teoría de grafos y la teoría de trenzas.\\

Uno de los problemas clave que tratamos en este proyecto es la equivalencia de nudos, pero con el algoritmo que proporcionamos mediante el teorema de Reidemeister no podemos obtener siempre respuesta en un tiempo aceptable. Encontramos solución usando invariantes para nudos, pero esta solución no es completa en el sentido de que no siempre vamos a obtener respuesta usando invariantes. Por este motivo, nos enfocamos en la teoría de trenzas. Al tener una relación directa entre nudos y trenzas cerradas, podemos plantearnos el problema de la equivalencia de nudos como el problema de la equivalencia de trenzas cerradas. Además, la teoría de trenzas tiene aplicación directa en criptografía y en mecánica de fluidos, entre otras  materias.\\ 

En cuanto a teoría de trenzas vemos una definición formal del concepto, vemos que tiene estructura de grupo no abeliano, estudiamos la equivalencia de trenzas y de trenzas cerradas, analizamos algunos invariantes para trenzas y trenzas cerradas y estudiamos el problema de las palabras que nos permite determinar si dos trenzas dadas serán o no equivalentes. \\

Analizar las trenzas a mano conllevan más trabajo del que puede parecer a simple vista, además si es necesario analizar una gran cantidad de trenzas se convierte en un trabajo muy tedioso y repetitivo. Es por eso que interesa desarrollar software que nos permita trabajar con trenzas. \\


En este proyecto, finalizamos presentando toxtren: se trata de un toolbox que hemos creado en Matlab para trabajar tanto con trenzas como con trenzas cerradas. Trabajamos con ellas siguiendo un diseño orientado a objetos creando una clase para cada tipo de objeto. Mostramos cómo podemos instalar toxtren de una forma realmente simple. Además, presentamos los pseudo algoritmos más destacados para ambas clases.\\
Por último, presentamos una guía en la que mostramos los comandos necesarios para trabajar con toxtren. Hemos permitido que el usuario pueda introducir las trenzas de distintas formas para que le resulte lo más cómodo posible. \\

En conclusión, hemos estudiado dos teorías relevantes en la rama de la topología como son la teoría de nudos y la teoría de trenzas, hemos visto la relación que hay entre ambas y el impacto que tienen hoy en día en diversas áreas de conocimiento. Finalmente, hemos desarrollado toxtren, toolbox en el que implementamos todos los conceptos y algoritmos que hemos analizado en la teoría de trenzas. 

 
\cleardoublepage


\thispagestyle{empty}

\selectlanguage{USenglish}
\begin{center}
{\large\bfseries PONER TITULO EN INGLES!!!}\\
\end{center}
\begin{center}
Cristina Zuheros Montes.\\
\end{center}

%\vspace{0.7cm}
\noindent{\textbf{Keywords}: knot, braid, handle....PONER KEYWORDS!!!!!}\\

\vspace{0.7cm}
\noindent{\textbf{Abstract}}\\

//PONER RESUMEN 1500 MIN INGLES.!!!!!!\\

\selectlanguage{spanish}



