\bigskip
\section{Sobre su historia y aplicaciones.}
En el siglo XIX, ciertos físicos escoceses se preguntaban por la estructura de los átomos.\\
Estos científicos tomaron como base la teoría de Descartes, que afirmaba que el \textit{éter} era un fluido que ocupaba todo el espacio y transmitía la luz (éter lumínico), para desarrollar su modelo del átomo. \\
Aunque dichos físicos conocían la existencia de los elementos y que estaban formados por átomos, no conocían la propia estructura de los átomos. \\

Científicos como Peter Guthrie Tait y Willian Thomson llegaron a la teoría de que los átomos se concebían como vórtices, que podríamos ver como remolinos tubulares, en dicho fluido. Estos vórtices se encontraban anudados y en función del tipo de anudamiento darían lugar a un tipo de elemento u otro.\\
De este modo se plantearon que los diferentes nudos corresponderían a los diferentes elementos de la naturaleza. De acuerdo con la teoría, si conociésemos todos los nudos posibles, crearíamos la tabla de elementos que reemplazaría la tabla periódica actual. \\

Para hacernos una idea más clara, para Willian Thomson el nudo trébol podría corresponder con el átomo de helio, el nudo de ocho con el átomo de oxígeno....\\

Numerosos científicos contribuyeron a dicha teoría intentando crear la tabla de nudos pero a finales de este mismo siglo, Michelson-Morley demostró que el éter lumínico no existía y por tanto la teoría de los átomos de vórtice fue descartada. \\

Tras este hecho, la teoría de nudos perdió su interés hasta que fue objeto de estudio en Topología a principios del siglo XX.\\

Posteriormente esta rama de la topología de baja dimensión destacó por su gran interés en áreas como:
  \begin{enumerate}
  	\item Química: ya hemos visto que la teoría de nudos nace en este área.
  	\item Biología: se estudia la teoría de nudos en la estructura de ADN.\\
  	Conocemos como ácido desoxirribonucleico (ADN) a aquella molécula que se encuentra en el núcleo de nuestras células y, por tanto, que contiene nuestro código genético. Se trata de un elemento esencial para la vida. Es muy conocida su forma: se puede ver como dos cuerdas enrolladas formando un doble hélice. \\
  	
  	Su forma de doble hélice puede encontrarse cerrada por los extremos de forma que nos encontraríamos con la propia forma de un nudo. Las ideas que veremos en este proyecto, junto con algunas más, se pueden aplicar a estas estructuras de ADN.\\
  	
  	Además, esta estructura de ADN puede sufrir ciertas alteraciones producidas por la encima topoisomerasa. Lo interesante es que estas alteraciones se corresponden con algunos de los movimientos que veremos para los nudos.\\
  	
  	
  	\item Criptografía \cite{12}: 
  	La fuerte relación que veremos entre la teoría de nudos y la teoría de trenzas, hace que podamos establecer cierta relación entre la teoría de nudos y la criptografía. \\
  	
  	Haciendo uso de la criptografía tratamos de encontrar métodos para que el intercambio de información no sea comprensible por terceras personas.\\
  	
  	Algunos métodos para encriptar dicha información están inspirados en la teoría de trenzas. En concreto el problema de la conjugación de trenzas, que trata de estudiar la igualdad de dos trenzas haciendo uso de una tercera trenza, nos permitirá encriptar la información de forma segura. \\
  \end{enumerate}
