\section{Contextualización.}
Dentro de la ciencia matemática nos encontramos una rama, conocida como topología, que se encarga del estudio de las propiedades de aquellos cuerpos que, mediante transformaciones continuas sobre los cuerpos, permanecen inalteradas. Aunque no se conoce una fecha exacta del origen de la topología, se suele establecer en 1735 con la resolución, por parte de Leonhard Euler, del problema de los 7 puentes de Königsberg.\\

Dentro de la topología surge la teoría de nudos que sólo tiene algo más de unos 200 años. Se puede decir que es una teoría bastante reciente y cuyos resultados más interesantes se han ido descubriendo en las últimas décadas. Es por esto que la teoría de nudos es muy llamativa tanto para investigadores matemáticos como para biólogos, físicos, químicos, marineros, expertos en supervivencia y una gran cantidad de científicos que encuentra en esta teoría una respuesta a sus planteamientos. Lo atractivo de dicha teoría es que, al ser relativamente reciente, aún tiene muchas cuestiones abiertas en las que se puede trabajar. \\

//COMPLETAR CON ALGO DE LA PARTE DE HISTORIA PONIÉNDOLO CON OTRAS PALABRAS Y TAL.....



\section{Problema abordado.}

\section{Técnicas y áreas matemáticas.}

\section{Contenido de la memoria.}

\section{Fuentes principales.}

\section{Objetivos.}

En la propuesta inicial se propusieron los siguientes objetivos:

\begin{enumerate}
\item Estudiar...//PONER LOS OBJETIVOS INICIALES
\end{enumerate}

//PONER DONDE SE CUBREN LOS OBJETIVOS